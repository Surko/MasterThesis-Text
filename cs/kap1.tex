\newcommand{\infogain}{\mbox{Info\_zisk}}
\newcommand{\entropy}{\mbox{Entropia}}
\newcommand{\ginindex}{\mbox{Gini\_zisk}}
\newcommand{\gini}{\mbox{Gini}}
\newcommand{\gainratio}{\mbox{Koef\_zisku}}
\newcommand{\twoing}{\mbox{Twoing}}

\chapter{Úvod k rozhodovacím stromom}
V tejto kapitole zadefinujeme základné pojmy a koncepty, ktoré budú obsiahnuté v ďalších častiach tejto práce. V nadchádzajúcich oddieloch vysvetlíme postupne všetky dôležité pojmy. V oddieli \ref{kap1:2.1:Data} pomenovanom Dáta zadefinujeme čo sú to dáta, zavedieme pojem atribút a  vysvetlíme základné pojmy, s ktorými budeme pracovať. Dobývanie znalostí, ktoré predstavuje oddiel \ref{kap1:2.2:DataMining}, popíše tento obor so zaradením rozhodovacích stromov v ňom. V oddieli \ref{kap1:2.3:DT} zavedieme základný model rozhodovacích stromov, pre ktorý bude nutné uviesť pár pojmov z teórie grafov. Oddiel \ref{kap1:2.4:DTTypes} zoznamuje s rôznymi typmi rozhodovacích stromov. V oddieli \ref{kap1:2.5:DTSplitCriterias} predstavíme deliace kritéria využívané pri tvorbe stromov. Metriky na vyhodnotenie kvality stromov zavedieme v oddieli \ref{kap1:2.6:DTEvaluation}. V oddieli \ref{kap1:2.7:DTTechniques} zhrnieme techniky nazývané indukčné algoritmy na tvorbu stromov.
Nakoniec v oddieli \ref{kap1:2.8:DTUsage} ukážeme využitie rozhodovacích stromov pri predikcii dát.

\section{Dáta}\label{kap1:2.1:Data}
V oddieli popíšeme čo sú to dáta. Dozvieme sa tie najzákladnejšie pojmy s ktorými sa stretneme pri práci s nimi. Informácie sme čerpali z voľne dostupnej knihy  \cite{kap1-DataMiningAndAnalysis} a online zdrojov \cite{wiki-Data,online-Data}.

V prvej časti \ref{kap1:2.1:2.1.1:DataRepresentation} popíšeme, čo sú to dáta a ako ich môžme reprezentovať. Popíšeme základné pojmy, ako je atribút a príznakový vektor. V časti \ref{kap1:2.1:2.1.2:DataAttributes} rozdelíme atribúty na dva druhy - kategoriálne a numerické. Aj napriek jednoduchosti týchto pojmov je ich nutné zaviesť, pretože ich budeme využívať v neskorších kapitolách.

\subsection{Dáta a ich reprezentácia}\label{kap1:2.1:2.1.1:DataRepresentation}
Dáta sú množiny hodnôt, ktoré predstavujú kusy informácií.
Môžu byť zozbierané, odmerané, analyzované a následne vizualizované.
Zozberané dáta môžme nájsť v relačných, tabuľkových databázach. Tento typ dát nazývame štrukturované. V skutočnosti sa však vo väčšine prípadov stretávame presne s ich opakom, a to s neštrukturovanými dátami.
Podľa \cite{kap1-DataMiningAndAnalysis} si dáta možme predstaviť ako maticu $n \times d$, kde $n$ predstavuje počet riadkov a $d$ počet stĺpcov tejto matice. Riadky sú v tomto prípade popisujú objekty a stĺpce sú tvorené hodnotami danej vlastnosti. Matica je daná takto

\begin{center}
Data = 
$\begin{array}{c | c c c c}
& X_{1} & X_{2} & \ldots & X_{d} \\ \hline
\mathbf{x}_{1} & x_{11} & x_{12} & \ldots & x_{1d} \\
\mathbf{x}_{2} & x_{21} & x_{22} & \ldots & x_{2d} \\
\vdots & \vdots & \vdots & \ddots & \vdots \\
\mathbf{x}_{n} & x_{n1} & x_{n2} & \ldots & x_{nd} \\
\end{array}$
\end{center}

$x_{i}$ predstavuje $i$-tý riadok, ktorý obsahuje $d$ hodnôt.
Jeden riadok matice taktiež nazývame príznakový vektor, entita, objekt, transakcia alebo inštancia.
\begin{center}
$\mathbf{x}_{i} = (x_{i1},x_{i2},\ldots,x_{id})$
\end{center}

$X_{j}$ predstavuje $j$-tý stĺpec, ktorý obsahuje $n$ hodnôt.
Na druhú stranu, stĺpec matice označujeme aj ako príznak, vlastnosť, atribút.
\begin{center}
$X_{j} = (x_{1j},x_{2j},\ldots,x_{nj})$
\end{center}

Pri metódach učenia s učiteľom uvažujeme ešte jeden stĺpec. Jeho hodnoty odpovedajú klasifikácii alebo regresii jednotlivých inštancií. Tieto hodnoty sú z domény množiny označovanej $C$. Kvôli konzistencii budeme tento posledný stĺpec ďalej nazývať ako výstupný atribút.
\begin{align}
C &= (c_{1},c_{2},\ldots,c_{m}) \nonumber
\end{align}

Premennú \textit{n} nazývame veľkosť dát a premennú \textit{d} zase dimenzionalita alebo rozmer dát.

Ako sme už spomínali, štrukturované dáta sú tie, ktoré máme uchované v riadkovo-stĺpcových databázach a teda ich ľahko prevedieme do maticového zápisu. Pre komplexnejšie dátové množiny (neštrukturované), ktoré sa vyskytujú napr. v bioinformatike (DNA, proteínové sekvencie,...), je buď nutné použiť iný zápis alebo využiť techniku extrakcie príznakov. Medzi ďaľšie neštrukturované dáta zaraďujeme obrázky, text, audio ale aj video nahrávky.

\subsection{Atribúty}\label{kap1:2.1:2.1.2:DataAttributes}
Doména atribútu je množina hodnôt, ktoré môže atribút nadobudnúť. Atribúty môžme rozdeliť na 2 typy podľa ich domény:
\begin{itemize}
\item \textbf{Kategoriálne} -- doména týchto atribútov sa skladá z konečnej množiny symbolov. Takýmto atribútom môže byť napríklad pohlavie (muž, žena), rodinný stav (slobodný, ženatý), ale aj komplikovanejšie ako je študijný program (informatika, fyzika, matematika, ...). Tieto atribúty je možné rozdeliť ešte do dvoch skupín:
\begin{itemize}
\item \textbf{Nominálne}, keď hodnoty v doméne nie sú usporiadané. Zmysluplné je iba porovnanie na zhodu. Príkladom môže byť pohlavie.
\item \textbf{Usporiadané}, keď sa hodnoty v doméne dajú porovnávať. Takéto hodnoty môžeme porovnávať nie len na zhodu, ale aj porovnávať (hodnota je väčšia, menšia). Ako príklad môžeme uviesť vzdelanie (základné, stredné, vysokoškolské).
\end{itemize}
\item \textbf{Numerické} sú také, ktorých doména je založená na celých alebo reálnych číslach. Jedným z takýchto atribútov môže byť príjem, zostatok na účte, \ldots. 
\end{itemize}

\section{Dobývanie znalostí}\label{kap1:2.2:DataMining}
V tomto oddieli zhrnieme vedomosti z dobývania znalostí, ktorého základ je dôležitý pri zaradení rozhodovacích stromov do správneho oboru.
Znalosti sme čerpali z kníh, ktoré sú vhodné ako úvod do tohoto oboru \cite{kap1-DataMiningAndAnalysis,kap1-DataMiningForMasses,kap1-DataMiningForTrees,kap1-StatisticLearn}. Pomimo týchto kníh sú informácie získané aj z ďaľších zdrojov ako je Wikipédia \cite{wiki-DataMining} ale aj menej známa HTML stránka od Dr. Saed Sayada \cite{online-DataMining}, ktorá stojí za povšimnutie. Všetky materiály sú voľne dostupné online.
 
V časti \ref{kap1:2.2:2.2.1:KDD} popíšeme proces dobývania znalostí z databází, anglicky knowledge discovery in databases (KDD). Hlavné informácie o samotnom dobývaní znalostí zhrnieme v časti \ref{kap1:2.2:2.2.2:DataMineProcess}. Tu ho rozdelíme do štyroch vrstiev. Každú vrstvu vysvetlíme a pri tom do tejto schémy zaradíme rozhodovacie stromy. V časti \ref{kap1:2.2:2.2.3:Taxonomy} predstavíme taxonómiu metód s nadväznosťou na zavedený 4-vrstvový model. Ďalší oddiel \ref{kap1:2.2:2.2.4:Supervised} vysvetlí, čo sú to metódy učenia s učiteľom, rozlíšime jeho dve obmeny a spomenieme tu aj učenie bez učiteľa.
Nakoniec v oddieli \ref{kap1:2.2:2.2.5:Tools} zhrnieme a uvedieme pár základných nástrojov, ktoré môžeme použiť pri práci v oboru dobývania znalostí. 

\subsection{Dobývanie znalostí z databází}\label{kap1:2.2:2.2.1:KDD}
Tiež známe pod skratkou KDD je komplikovaný proces pozostávajúci z viacerých fáz, ktorý ma za úlohu identifikovať nové, využiteľné vzory v dátach. KDD zahŕňa aj samotné dobývanie znalostí, ktoré je pravdepodobne jeho najvýznamnejšou súčasťou. Kvôli prominentnému postaveniu dobývania znalostí v rámci KDD, ich mnoho odborníkov stotožňuje.

\begin{figure}[h]
\centering
\centerline{\mbox{\includegraphics[width=400pt]{../img/kap1/DM-KDD.pdf}}}
\caption{Štrukturovaný KDD proces (prekreslené a preložené z \cite{kap1-DataMiningForTrees}).}\label{fig:dataMineKDD}
\end{figure}

Jednou z úloh v minulosti bolo formalizovať a štandardizovať prístup k dobývaniu znalostí. Na Obrázku \ref{fig:dataMineKDD} sú jednotlivé časti sedem-stupňového modelu. Toto rozdelenie ale nie je pevné. Mnoho iných prác a spoločností navrhlo svoje obmeny. Veľké korporácie ako Daimler-Benz, poskytovateľ poistenia OHRA, vývojári softwaru a hardwaru NCR Corp. sa v roku 1999 spojili a vytvorili svoj vlastný model. Týmto vznikol známy CRISP-DM, ktorého štruktúru je možné vidieť na Obrázku~\ref{fig:dataMineCRISP} \cite{kap1-DataMiningForMasses}. Ďalším známym modelom je model 5A alebo SEMMA.

\begin{figure}[h]
\centering
\centerline{\mbox{\includegraphics{../img/kap1/DM-CRISP.pdf}}}
\caption{CRISP-DM model (prekreslené a preložené z \cite{kap1-DataMiningForMasses})}\label{fig:dataMineCRISP}
\end{figure}

\subsection{Popis dobývania znalostí}\label{kap1:2.2:2.2.2:DataMineProcess}
Dobývanie znalostí je technika z KDD, ktorá je priamo zodpovedná za vytváranie poznatkov o existujúcich dátach, tak isto ako za predikciu určitých vlastnosti dát. Je to obor, ktorý kombinuje viaceré poznatky zo štatistiky, umelej inteligencie, strojového učenia a čiastočnej znalostí databázových strojov (Obrázok \ref{fig:dataMineComb}).

% IMAGE 1
\begin{figure}[h]
\centering
\centerline{\mbox{\includegraphics[width=225pt]{../img/kap1/DM-comb.pdf}}}
\caption{Dobývanie znalostí ako kombinácia rôznych oborov (prekreslené a preložené z \cite{online-DataMining})}\label{fig:dataMineComb}
\end{figure}

Celý proces dobývania znalostí je možné popísať v štyroch vrstvách. Tento model je znázornený na Obrázku \ref{fig:layerModel}. Každá z nižších vrstiev je dôležitá pre vrstvu vyššie.

% IMAGE 2
\begin{figure}[h]
\centering
\centerline{\mbox{\includegraphics[width=400pt]{../img/kap1/DM-layer.pdf}}}
\caption{Proces dobývania znalostí v 4 vrstvách (prekreslené a preložené z \cite[s. 26]{kap1-DataMiningForTrees})}\label{fig:layerModel}
\end{figure}

Prvá vrstva sa zaoberá už konečnou, výstupnou aplikáciou, pri tvorbe ktorej sme využili všetky predchádzajúce vrstvy. Tieto majú široký záber využitia v podnikateľských sférach. Medzi najvyužívanejšie patrí ohodnocovanie zákazníkov podľa ich príjmu, tak ako aj detekcia fraudov\footnote{podvod alebo nepoctivý trik, ktorým je poškodený zainteresovaný uživateľ}. Na vytvorenie aplikácie použijeme jednu alebo viacero techník, ktoré na našom obrázku predstavujú druhú vrstvu. Táto vrstva sa zaoberá úlohami strojového učenia ako je regresia, zhlukovanie a mnoho iných. Tieto úlohy využívajú rôzne, už konkrétne modely (tretia vrstva), kde patria už spomínané rozhodovacie stromy, neurónové siete, kohonenove mapy, \ldots. Posledná vrstva sa zaoberá tvorbou týchto modelov, v ktorej sa nachádzajú aj indukčné algoritmy tvorby rozhodovacích stromov.


\subsection{Taxonómia metód}\label{kap1:2.2:2.2.3:Taxonomy}
Z predchádzajúceho oddielu vieme, že časti dobývania znalostí sú umiestnené v nejakých vrstvách. V tomto oddieli nás budú zaujímať hlavne vzťahy v rámci druhej a tretej vrstvy. Vertikálne rozlíšenie je uvedené v predchádzajúcom oddieli. Na zistenie nejakých ďalších vzťahov je nutné zaradiť metódy do širších jednotiek ako na Obrázku \ref{fig:dataMineParad}. Jeden taký pohľad na rozdelenie metód môžeme nájsť aj v \cite{online-DataMining}. V tomto druhom prípade je taxonómia v konečnom dôsledku funkcionálne rovnaká (niektoré názvy sú iné), ale líšia sa štruktúrou. 

V úlohách je v prvom rade nutné rozlišovať medzi dvoma typmi dobývacích techník, verifikačné a objaviteľské. Každý z nich má svoju metodológiu. Verifikačný sa zaoberá testovaním hypotéz, analýzou rozptylu a ďalšími praktikami. Objaviteľské zase automaticky hľadajú nové pravidlá a vzory.

Verifikačné metódy spoliehajú na určitú predom známu hypotézu, ktorá býva vyhodnotená expertom. Tento prístup nie je až tak spojený s dobývaním znalostí, ako to je pri objaviteľských metódach. Najvýznamnejším dôvodom je ten, že techniky dobývania znalostí by mali vyberať hypotézu z množiny vhodných hypotéz (identifikácia modelu), než aby sa spoliehali na nejakú dopredu známu (odhadnutie modelu).

Medzi takéto objaviteľské techniky patria ako predikčné, tak aj popisné metódy (zhlukovanie, vizualizácia). Popisnými sa hlavne snažíme pochopiť ako a prečo sú dáta umiestnené tak, ako sú nám prezentované. Predikčné zase dokážu predpovedať hodnoty jednotlivých atribútov daného objektu. Indukčné algoritmy sú typickou technikou, ktoré sa používajú pri vytváraní týchto modelov.

\begin{figure}[h]
\centering
\centerline{\mbox{\includegraphics[width=400pt]{../img/kap1/DM-paradigm.pdf}}}
\caption{Taxonómia modelov, prekreslené a preložené z \cite{kap1-DataMiningForTrees})}\label{fig:dataMineParad}
\end{figure}

\subsection{Učenie s učiteľom a bez učiteľa}\label{kap1:2.2:2.2.4:Supervised}
Učenie s učiteľom je oblasťou dobývacích techník, ktoré na vytvorenie modelu potrebuje trénovaciu množinu. Z taxonómie zavedenej v predchádzajúcom oddieli sem patria predikčné metódy. Pri tomto type sa teda snažíme nájsť vzťahy v trénovacích dátach medzi vstupnými atribútmi (nazývanými aj nezávislé premenné) a výstupným, klasifikačným atribútom (tiež nazývaný závislá premenná). Nájdený vzťah v dátach je reprezentovaný určitou štruktúrou a je to v konečnom dôsledku náš model. Takýto model je vytváraný už nejakou konkrétnou metódou. Medzi najzákladnejšie patria rozhodovacie stromy, neurónové siete, SVM, \ldots. Poskytnutím nových inštancií môžeme model využiť na klasifikáciu výstupného atribútu z tých vstupných.
Využitie je veľmi bohaté v rôznych sférach (financie, obchod, služby, \ldots). 

Hlavným cieľom učenia je zlepšenie modelu na nejakej úlohe pomocou predom daných dát. K tomuto cieľu je nutné poznať tri jeho komponenty:
\begin{itemize}
\item Úloha $U$, ktorú chceme učením vylepšiť
\item Dáta $D$, ktoré použijeme pri učení
\item Meradlo výkonu $M$, ktoré je použité pri meraní miery zlepšenia.
\end{itemize} 
Pre lepšie pochopenie môžeme použiť známy problém identifikácie emailov do spamu\footnote{Spamová správa - nevyžiadaná správa, ktorú užívateľovi prišla do emailovej schránky.}. Pre túto vyzerajú komponenty takto:
\begin{itemize}
\item Úlohou $U$ je identifikovať nevyžiadané emaily.
\item Dáta $D$ sú v tomto prípade množiny emailov, v ktorých sa nachádzajú správne aj nesprávne emaily.
\item Meradlom výkonu $M$ je percento nevyžiadaných emailov, ktoré boli klasifikované správne a percento správnych emailov, ktoré boli klasifikované nesprávne.
\end{itemize} 

Trénovacie dáta sú v tomto prípade v trochu odlišnejšom dátovom zápise, ako sme definovali v časti \ref{kap1:2.1:Data}. Nová matica by vyzerala ako $TrainData = (Data | Y)$, kde znak $|$ znamená pripojenie stĺpca $Y$ za maticu $Data$. Iným zápisom môže byť aj $TrainData = \{(\mathbf{x}_{k},y_{k})\}_{k=1}^{n}$, kde $y_{k}$ je hodnota klasifikácie/regresie pre vektor $\mathbf{x}_{k}$. Hodnota $y_{k}$ je z domény množiny $C$ 

Pri modeloch učenia s učiteľom je nutné rozlišovať medzi dvoma základnými typmi podľa typu výstupného atribútu:
\begin{itemize}
\item \textbf{Klasifikačné} mapujú vstupný vektor na dopredu známe triedy. Príkladom môže byť predikovanie rizika rakoviny (žiadne, nízke, vysoké) pacienta podľa jeho lekárskej anamnézy.
\item \textbf{Regresné} mapujú vstupný vektor na reálne čísla. Takýto model dokáže napríklad určiť výšku pôžičky, ktorú môže banka poskytnúť z informácii o žiadateľovi.
\end{itemize} 

Učenie bez učiteľa je presným opakom od učenia s učiteľom. Tento typ vytvára model bez trénovacej množiny. Podľa taxonómie sem môžme zase zaradiť popisné metódy (nie vizualizáciu) z objaviteľských techník. 

\subsection{Nástroje pre DM}\label{kap1:2.2:2.2.5:Tools}
Existuje veľké množstvo nástrojov, ktoré boli vytvorené aby zvládali úlohy z dobývania znalostí. Medzi nimi existuje plno profesionálnych nástrojov určených priamo pre takéto úlohy (Weka, RapidMiner). Iné druhy sú vo forme knižníc do stávajúcich jazykov (Python, Matlab, ...). Stavané sú tak, aby dokázali využiť techniky umelej inteligencie, strojového učenia, štatistiky a iné. Veľa z nich je ale platených, niektoré zase komplikované na inštaláciu, konfiguráciu alebo použitie. Pre učenie, ľahšiu využiteľnosť a hlavne pre účel našej práce sú preto nevhodné. 

\begin{figure}[h]
\centering
\centerline{\mbox{\includegraphics[width=300pt]{../img/kap1/DM-Weka.png}}}
\caption{Náhľad na užívateľské prostredie nástroja Weka}\label{fig:dataMineWeka}
\end{figure}

Najvhodnejšími kandidátmi sú:
\begin{itemize}
\item \textbf{RapidMiner} je napísaný v jazyku Java a poskytuje pokročilé analytické techniky pomocou framework-u založeného na šablónach. Výhodou tohto nástroja je aj to, že používateľ nie je nútený vytvárať žiaden kód. Ponú\-kaný je skorej ako služba, než kus lokálneho softwaru. Popri riešení úloh dobývania znalostí dokáže aj predspracovanie, vizualizáciu dát, štatistické modelovanie a vyhodnocovanie. RapidMiner je jedným z najpoužívanejších a najlepších nástrojov na úlohy z dobývania znalostí. Sila nástroja narastá vďaka využiteľnosti schém, modelov a algoritmov z Weky a R-skriptov. Distribuovaný je pod AGPL licenciou a je voľne stiahnuteľný zo stránky SourceForge.
\item \textbf{Weka} je kolekciou algoritmov pre strojové učenie a dobývanie znalostí. Algoritmy môžu byť volané priamo z užívateľského prostredia programu (Obrázok \ref{fig:dataMineWeka}) alebo z vlastne vytvorenej aplikácie bežiacej pod Java prostredím. Obsahuje nástroje na predspracovanie, klasifikáciu, regresiu, asociačné pravidlá a vizualizáciu. Taktiež je použiteľná na vytváranie vlastných schém (prebrané z \cite{online-DataMiningWeka}). Aplikáciu si je možné stiahnuť zadarmo z hlavnej stránky pod GNU General Public licenciou. Tento typ licencie je jedna z najdôležitejších výhod Weky oproti RapidMiner-u. Vďaka nej je možná ľubovoľná úprava nástroja podľa našich požiadavkov, tak ako aj pridanie úplne nových algoritmov.
\end{itemize}

Ďalší kandidáti nie sú samostatnými aplikáciami, ale moduly pre dobývanie znalostí do nejakého dynamického jazyka. Tieto možnosti môžeme považovať za pokročilejšie, keďže na ich využitie je nutné spísať vlastný kód vo forme skriptov.
\begin{itemize}
\item \textbf{Matlab} je jazyk vysokej úrovne a interaktívne prostredie používané širokou skupinou vedcov a technikov \cite{online-DataMiningMatlab}. Príjemné prostredie tohoto jazyka a možnosť interaktívneho ladenia svojich programov zjednodušuje prácu. Rozširujúce moduly nazývané toolboxy\footnote{panel nástrojov, pomenovanie modulov do Matlab-u}, majú široké spektrum zamerania. Medzi najvýznamnejšie moduly na tvorbu modelov a dátovú analýzu pat\-ria štatistický modul (Statistics and Machine Learning Toolbox), modul pre neu\-rónové siete (Neural Network Toolbox) a mnoho ďalších. Tieto moduly sa dajú často ovládať z takzvaných pomocných nástrojov\footnote{pomocný nástroj/wizard je skupina po sebe nasledujúcich obrazoviek, ktoré prevedú užívateľa celým procesom tvorby požadovaného programu, bez ďalšej znalosti vnútornej funkcionality}.
Nevýhodou tohoto prístupu je buď nutnosť vlastniť Matlab, ktorý je spoplatnený, alebo mať alternatívny Octave, ktorým zase strácame príjemné užívateľské prostredie s ďalšími výhodami modulov, ako sú spomenuté pomocné nástroje.
\item \textbf{Python} je vysoko úrovňový dynamický jazyk podobne ako to je u Matlab-u \cite{wiki-Python}. Je to obľúbeným jazykom vedcov a jeho hlavnou filozofiou je zaručiť čitateľnosť kódu. Python podporuje mnoho programátorských paradigmát (objektovo orientovaný, imperatívny, funkcionálny, ...). Taktiež k nemu existuje mnoho knižníc na zjednodušenie práce v širokej škále oborov. Najoblúbenejší nástroj pre štatistické a strojové učenie je obsiahnutý v balíčku scikit-learn \cite{kap1-Scikit}. Tento balík v sebe viaže efektívne a jednoduché nástroje pre dátovú analýzu aj dobývanie znalostí. Stavia na obľúbených knižniciach NumPy, SciPy a matplotlib. Jazyk Python je k tomu voľne dostupný a scikit-learn je so svojou BSD licenciou skvelým kandidátom aj pre obchodné účely. 
\end{itemize}

\section{Základné pojmy rozhodovacích stromov}\label{kap1:2.3:DT}
V tomto oddieli vysvetlíme základné pojmy, ktoré sa používajú v spojitosti s rozhodovacími stromami. Najprv zadefinujeme čo je strom a čo znamenajú jednotlivé jeho časti v rozhodovacom strome. Informácií o tejto metóde existuje veľa, no základné pojmy sme hlavne čerpali z kníh \cite{kap1-DataMiningForTrees,kap1-DecisionTree} a \cite[s. 481-498]{kap1-DataMiningAndAnalysis}. Ďalšími použitými zdrojmi bola Wikipédia \cite{wiki-DecisionTree} a scikit-learn stránka \cite{online-DecisionTreeScikit}, na ktorej sa nachádzajú zaujímavé príklady.

V časti \ref{kap1:2.3:2.3.1} popíšeme čo je strom. Rozhodovacie stromy potom na základe grafovej štruktúry stromu zavedieme v časti \ref{kap1:2.3:2.3.2}.

\subsection{Strom}\label{kap1:2.3:2.3.1}
Na bližšie zadefinovanie rozhodovacieho stromu je potrebný aspoň minimálny základ z teórie grafov. Základné pojmy sme čerpali z knihy \cite{kap1-KapitolyDiskretka}.
\begin{def-sk}[Graf]\label{kap1:2.3:2.3.1:graf}
Graf je usporiadaná dvojica (V,E), kde V je nejaká neprázdna množina a E je množina dvojbodových podmnožín množiny V. Prvky množiny V sa nazývajú vrcholy grafu G a prvky množiny E sú hrany grafu G.
\end{def-sk}

\begin{figure}[h]
\centering
\centerline{\mbox{\includegraphics[width=300pt]{../img/kap1/DT-graphs.pdf}}} 
\caption{Príklady neorientovaných grafov. Čierne body predstavujú vrcholy a čiary medzi vrcholmi reprezentujú hrany z $E$.}\label{fig:decisionTreeGraphs}
\end{figure}

\begin{def-sk}[Orientovaný graf]\label{kap1:2.3:2.3.1:orient-graf}
Orientovaný graf G je dvojica (V,E), kde V je nejaká neprázdna množina a E je podmnožina kartézskeho súčinu $V \times V$. Prvky množiny V sa nazývajú vrcholy grafu G a prvky množiny E nazývame orientované hrany. Orientovaná hrana má tvar (x,y). Hovoríme, že orientovaná hrana vychádza z x a končí v y.
\end{def-sk}

Príklady obyčajných(neorientovaných) a orientovaných grafov je možné vidieť na Obrázku \ref{fig:decisionTreeGraphs} a \ref{fig:decisionTreeOrientGraphs}.

\begin{figure}[h]
\centering
\centerline{\mbox{\includegraphics[width=300pt]{../img/kap1/DT-orientgraphs.pdf}}}
\caption{Príklady orientovaných grafov. Čierne body predstavujú vrcholy a čiary medzi vrcholmi reprezentujú orientované hrany z $E$. Šípka určuje orientáciu hrany.}\label{fig:decisionTreeOrientGraphs}
\end{figure}

\begin{def-sk}[Symetrizácia]\label{kap1:2.3:2.3.1:symetrizacia}
Každému orientovanému grafu $G = (V,E)$ môžeme priradiť neorientovaný graf $sym(G) = (V,E')$, kde $E' = \{\{x,y\}; (x,y) \in E$ alebo $(y,x) \in E\}$. Graf $sym(G)$ nazývame symetrizáciou grafu G.
\end{def-sk}

\begin{def-sk}[Súvislosť]\label{kap1:2.3:2.3.1:suvislost}
Hovoríme, že graf G je súvislý, keď pre každé dva jeho vrcholy x a y v ňom existuje cesta z x do y. Orientovaný graf je súvislý, keď je súvislá jeho symetrizácia (takáto súvislosť sa nazýva slabá).
\end{def-sk}

Pre orientované hrany existuje ešte druhý typ súvislosti a to silná. Avšak táto súvislosť je pre naše účely a hlavne definíciu rozhodovacích stromov nepodstatná.

\begin{def-sk}[Kružnica]\label{kap1:2.3:2.3.1:kruznica}
Graf $C_{n}$, kde n > 3, nazývame kružnicou, keď $ V = \{1,2,...,n\}$ a $E = \{\{i,i+1\};i=1,...,n-1\} \cup \{\{1,n\}\}$.
\end{def-sk}

Na Obrázku \ref{fig:decisionTreeCycles} sú príklady štyroch druhov kružníc (tých najmenších).

\begin{figure}[h]
\centering
\centerline{\mbox{\includegraphics[width=300pt]{../img/kap1/DT-cycles.pdf}}}
\caption{Pár príkladov kružníc. Čierne body predstavujú vrcholy a čiary medzi vrcholmi reprezentujú hrany z $E$.}\label{fig:decisionTreeCycles}
\end{figure}

\begin{def-sk}[Strom]\label{kap1:2.3:2.3.1:strom}
Strom je súvislý graf neobsahujúci kružnicu. Orientovaný strom je súvislý orientovaný graf, ktorý po symetrizácii neobsahuje žiadnu kružnicu.
\end{def-sk}

Podrobnejší výklad o stromoch môže čitateľ nájsť v knihe \cite{kap1-KapitolyDiskretka}. 

\begin{def-sk}[Zakorenený strom]\label{kap1:2.3:2.3.1:korenovystrom}
Zakorenený strom je orientovaným stromom, v ktorom je jeden význačný vrchol nazývaný koreň stromu. Hrany takéhoto stromu sú jednoznačne orientované a vedú smerom od koreňa. Pre zakorenený strom ďalej definujeme tieto pojmy:
\begin{itemize}
\item \textbf{potomok} vrcholu $V$ je každý vrchol, do ktorého je vedená orientovaná hrana z vrcholu $V$,
\item \textbf{list} je vrchol, ktorý nemá ďalšieho potomka.
\end{itemize} 
\end{def-sk}

\begin{figure}[h]
\centering
\centerline{\mbox{\includegraphics[width=200pt]{../img/kap1/DT-parentchild.pdf}}}
\caption{Popis jednotlivých druhov vrcholov vzhľadom k danému vrcholu (červený). Modrá -- nasledovníci, žltá -- predchodcovia, zelená -- potomkovia, fialová -- rodič)}\label{fig:decisionTreeParentChild}
\end{figure}

\begin{def-sk}[Vzťahy v strome]\label{kap1:2.3:2.3.1:naslednik}
Nech $G = (V,E)$ je zakorenený strom (Obrázok \ref{fig:decisionTreeParentChild}) a $k$ je koreň stromu, potom uvažujme:
\begin{itemize}
\item \textbf{predchodca} vrcholu $v$ je každý vrchol na orientovanej ceste z koreňa $k$ do vrcholu $v$,
\item \textbf{nasledovník} vrcholu $v$ je každý vrchol, kde jedným z predchodcov tohoto vrcholu je vrchol $v$,
\item \textbf{rodič} vrcholu $v$ je vrchol, z ktorého vedie orientovaná hrana do vrcholu $v$. Inak aj bezprostredný predchodca vrcholu $v$,
\item \textbf{potomok} vrcholu $v$ je vrchol, do ktorého vedie orientovaná hrana z vrcholu $v$. Inak aj bezprostredný nasledovník vrcholu $v$.
\end{itemize} 
\end{def-sk}

\subsection{Rozhodovací strom}\label{kap1:2.3:2.3.2}
Rozhodovací strom je prediktívny model, ktorý môžeme použiť pri riešení klasifikačných, ale aj regresných úloh. Z časti \ref{kap1:2.2:2.2.4:Supervised} vieme, že táto technika patrí pod metódy učenia s učiteľom. Z tejto časti taktiež vieme, že trénovacie dáta majú mierne pozmenený zápis od toho zadefinovaného v oddieli \ref{kap1:2.1:2.1.1:DataRepresentation}.

\begin{def-sk}[Trénovacie dáta]\label{kap1:2.3:2.3.2:traindata}
Trénovacie dáta sú n prvkovou množinou dvojíc $D = \{(\mathbf{x}_{k},y_{k})\}_{k=1}^{n}$, kde $\mathbf{x}_{k}$ je $d$-rozmerný príznakový vektor $d \geq 1$ a $y_{k}$ je jeho klasifikácia/regresia vybraná z množiny $C$. Pre klasifikáciu je doménou $C$ konečná množina symbolov, $C = \{c_{1},c_{2},\ldots,c_{m}\}$. Pri regresii je doména $C$ nejaká nekonečná množina, ako napríklad interval reálnych čísel $\mathbb{R}$, podmnožina celých čísel, $\ldots$.
\end{def-sk}

\begin{def-sk}[Podmnožina dát podľa hodnoty atribútu]\label{kap1:2.3:2.3.2:subsetValueData}
Nech $D = \{(\mathbf{x}_{i},y_{i})\}_{i=1}^{n}$ sú trénovacie dáta. Ďalej nech $D_{X_{i} = t}$ je podmnožina $D$ taká, že obsahuje práve všetky dvojice $(\mathbf{x},y) \in D$, pre ktoré platí $\mathbf{x} = (x_{1}, \ldots, x_{d})$ a $x_{i} = t$. Podobne $D_{c_{i}}$ je podmnožina dátovej množiny $D$ skladajúca sa z dvojíc $(\mathbf{x},y) \in D$, pre ktoré platí $y = c_{i}$.
\end{def-sk}

Na Obrázku \ref{fig:decisionPlaneDataa} sú zobrazené štyri množiny dát, ktoré dokopy tvoria celú množinu $D$ ($=D_{1} \cup D_{2} \cup D_{3} \cup D_{4}$). Farba jednotlivých útvarov určuje ich klasifikáciu.
 
\begin{figure}[h]
\centering
\begin{subfigure}[b]{0.45\textwidth}
\includegraphics[width=\textwidth]{../img/kap1/DT-planedata.pdf}
\caption{}\label{fig:decisionPlaneDataa}
\end{subfigure}
\qquad
\begin{subfigure}[b]{0.45\textwidth}
\includegraphics[width=\textwidth]{../img/kap1/DT-planedatasep.pdf}
\caption{}\label{fig:decisionPlaneDatab}
\end{subfigure}
\caption{Umiestnenie dátových množín $D = D_{1} \cup D_{2} \cup D_{3} \cup D_{4}$ v rovine. Farby určujú klasifikáciu/regresiu pre množinu. V prípade b) sú vyznačené aj deliace nadroviny.}\label{fig:decisionPlaneData}
\end{figure}

\begin{def-sk}[Zobrazenie na atribúty]\label{kap1:2.3:2.3.2:mapovanievrcholov}
Nech P je množina príznakových vektorov v daných dátach, $G = (V,E)$ je zakorenený strom a L je množina všetkých listov tohoto stromu. Potom zobrazením $\alpha:(V \setminus L) \times P \rightarrow V$ priraďujeme nelistovému vrcholu $u \in V \setminus L$ a špecifickému príznakovému vektoru $\mathbf{x} \in P$ vrchol $v \in V$ (listový alebo nelistový), taký že $(u,v)$ je orientovaná hrana z $E$.
\end{def-sk}

\begin{def-sk}[Zobrazenie na výstupný atribút]\label{kap1:2.3:2.3.2:mapovanielistov}
Nech C je výstupný atribút a dom(C) jeho obor hodnôt v daných dátach, $G$ je koreňový strom a L je množina všetkých listov tohoto stromu. Potom zobrazenie $\beta:L \rightarrow dom(C)$ priraďuje každému listu triedu alebo reálnu hodnotu.
\end{def-sk}

\begin{def-sk}[Rozhodovací strom]\label{kap1:2.3:2.3.2:DT}
Majme zakorenený strom $G = (V,E)$ a zobrazenia $\alpha$ a $\beta$, ako sú uvedené v Definíciach \ref{kap1:2.3:2.3.2:mapovanielistov} a \ref{kap1:2.3:2.3.2:mapovanievrcholov}. Rozhodovací strom pre trénovacie dáta $D$ definujeme ako trojicu $(G,\alpha,\beta)$.
\end{def-sk}

Pri takejto definícii klasifikácia/regresia funguje nasledovne. Daný príznakový vektor je testovaný na nelistovom vrchole pomocou funkcie $\alpha$, ktorou získame ďalší vrchol na otestovanie. Tento proces začína v koreni stromu a pokračuje až dokým nedostaneme nejaký listový vrchol. V tomto vrchole, za použitia funkcie $\beta$ dostávame požadovanú klasifikáciu/regresiu.
Testovanie väčšinou prebieha porovnávaním určitého atribútu alebo skupiny atribútov vektora s nejakou hodnotou.

\begin{def-sk}[Deliaca nadrovina]\label{kap1:2.3:2.3.2:axisHyperplanes}
Deliacu nadrovinu $h$, určenú vektorom $\mathbf{w}$ a číslom $b$ definujeme ako množinu všetkých bodov $\mathbf{x}$, pre ktoré platí $h:\mathbf{w^{T}x} + b = 0$, kde $\mathbf{w^{T}}$ nazývame normálový váhový vektor nadroviny $h$ a $b$ je vzdialenosť nadroviny $h$ od počiatku súradnicovej sústavy.
\end{def-sk}

V modeli rozhodovacích stromov sú uvažované oddeľujúce nadroviny, ktoré sú kolmé na jednu z hlavných osí. Z Definície \ref{kap1:2.3:2.3.2:axisHyperplanes} teda vyplýva, že normálový váhový vektor $\mathbf{w}$ takejto nadroviny musí byť taktiež rovnobežný s niektorou z hlavných osí. Preto je možné tento vektor obmedziť iba na jeden z množiny $\{\mathbf{e}_1,\mathbf{e}_2,\ldots,\mathbf{e}_d\}$, kde každý vektor $\mathbf{e}_i \in \mathbb{R}^{d}$ a obsahuje, okrem jedničky na $i$-tom mieste, samé nuly. Potom pre každý bod $\mathbf{x} = \{x_{1},x_{2},\ldots,x_{d}\}$ nadroviny $h$ platí:
\begin{align}
h:\mathbf{e}_i\mathbf{x} + b &= 0 \nonumber\\
h:x_{i} + b &= 0 \label{kap1:2.3:2.3.2:eq1}
\end{align}

\begin{def-sk}[Deliaci bod]\label{kap1:2.3:2.3.2:splitPoints}
Deliaci bod zodpovedajúci rozhodnutiu a definovaný nadrovinou rozdeľuje dátový priestor $\mathcal{R}$ na dva podpriestory $\mathcal{R}_{p}$ a $\mathcal{R}_{n}$. 
\end{def-sk}

\begin{remark-sk}\label{kap1:2.3:2.3.2:remarkSplitPoints}
Takže všetky body $\mathbf{x}$, pre ktoré platí $h(\mathbf{x}) < 0$, sú na jednej strane nadroviny, pričom $h(\mathbf{x}) \geq 0$ implikuje opačnú stranu. V prípade $h(\mathbf{x}) < 0$ podľa (\ref{kap1:2.3:2.3.2:eq1}) platí, že $x_{i} + b < 0$ a teda $x_{i} < -b$, pričom $x_{i}$ nadobúda nejakú hodnotu z domény atribútu $X_{i}$. Deliaci bod v rozhodovacom strome je preto potom uvádzaný vo formáte $X_{i} < v$, kde $v = -b$.
\end{remark-sk}

Rozhodovacie stromy teda obsahujú vrcholy, ktoré reprezentujú rozhodnutia zodpovedajúce deliacim bodom asociovanými s deliacimi nadrovinami.

\begin{def-sk}[Binárny rozhodovací strom]\label{kap1:2.3:2.3.2:binarnyDT}
Binárny rozhodovací strom je rozhodovací strom, pre ktorý platí, že počet potomkov každého vrcholu tohoto stromu je rovný 2 alebo 0 (len v prípade listov).
\end{def-sk}

\begin{figure}[h]
\centering
\centerline{\mbox{\includegraphics[width=350pt]{../img/kap1/DT-stump.pdf}}}
\caption{Príklady rozhodovacích koreňov. Vo vnútorných uzloch sú atribúty podľa ktorých sa rozhodujeme. V listoch sú klasifikované triedy. V prípade a) je atribút $z$ kategoriálny. V prípade b) je atribút $x$ numerický.}\label{fig:DT-stump}
\end{figure}

\begin{def-sk}[Rozhodovací koreň]\label{kap1:2.3:2.3.2:stumpDT}
Rozhodovací koreň je rozhodovací strom, ktorý obsahuje koreň a jeho nasledovníci sú iba listy. 
\end{def-sk}

Na Obrázku \ref{fig:decisionPlaneDatab} sú nakreslené 4 množiny bodov zo štyroch tried spolu s deliacimi nadrovinami. Rozhodovací strom, v konečnom dôsledku, svojou vnútornou stavbou pokladá tieto deliace čiary.

\section{Typy rozhodovacích stromov}\label{kap1:2.4:DTTypes}
Z časti taxonómia metód \ref{kap1:2.2:2.2.3:Taxonomy} vieme, že rozhodovací strom je predikčný model.
Uvádzali sme, že predikčné modely sú dvoch typov, klasifikačné a regresné. Delenie rozhodovacích stromov bude v tomto ohľade také isté. Informácie použité v tejto časti sme čerpali už zo spomínaných kníh \cite{kap1-DataMiningForTrees,kap1-DataMiningForMasses}, článku \cite{kap1-DecisionTreesTypes} a z online zdroja \cite{online-DecisionTreeTypes}.

Čo je to klasifikačný strom spolu s jeho dôležitými vlastnosťami zhrnieme v časti \ref{kap1:2.4:2.4.1:DTClassification}. Poznatky o regresných stromoch budú uvedené v časti \ref{kap1:2.4:2.4.1:DTRegression}. V obidvoch častiach ukážeme prevod rozhodovacieho stromu na súbor jednoduchých pravidiel pre ľahšiu predikciu.
\subsection{Klasifikačný strom}\label{kap1:2.4:2.4.1:DTClassification}
Rozhodovací strom, ako už vieme z predchádzajúcej kapitoly je hierarchická štruktúra, ktorá pomocou rozhodnutí dovedie zadaný vektor k určitej výstupnej hodnote. V obore dobývania znalostí je rozhodovací strom prediktívny model, ktorý môže reprezentovať ako klasifikačný, tak aj regresný model. Keď je strom použitý na klasifikačné úlohy, tak ho nazývame klasifikačný.

Klasifikačný strom používame vtedy, keď chceme klasifikovať objekt do triedy vybranej z konečnej, predom definovanej množiny tried podľa jeho príznakov. Klasifikačné stromy sú použiteľné najmä ako objaviteľská technika. Široko sú využívané v oblasti financií (pre ich zrozumiteľnosť), medicíny, vo vedeckých okruhoch, ale aj priemyselných procesoch. Aj napriek svojim výhodám nenahradzujú iné tradičné techniky predikcie ako sú Bayesovské alebo neurónové siete. Pri klasifikačných stromoch si je taktiež nutné uvedomiť, že atribúty vo vrcholoch môžu byť ako numerické, tak aj kategorické.

Na Obrázku \ref{fig:TypesClassify} je nakreslený typický klasifikačný strom. Tento príklad vznikol priamym prepisom Obrázku \ref{fig:decisionPlaneDatab} spolu s uvedenými deliacimi nadrovinami. Uvažujme, že každej definovanej množine $D_{1},\ldots,D_{4}$ z \ref{fig:decisionPlaneDatab} priradíme triedu $c_{1},\ldots,c_{4}$. Podľa pozorovania \ref{kap1:2.3:2.3.2:remarkSplitPoints} z predchádzajúcej časti je vidieť, že testy atribútov vo vnútorných uzloch stromu zodpovedajú deliacim bodom a konečné klasifikácie jednotlivým dátovým množinám na tomto obrázku. Vstupnú inštanciu, ktorú chceme klasifikovať označme $I$. Pravidlá tohoto klasifikačného stromu vyzerajú takto:
\begin{align}
Y(I) < 3.5 \wedge X(I) < 3.5 &\Rightarrow c_{2} \nonumber \\
Y(I) < 3.5 \wedge X(I) \geq 3.5 &\Rightarrow c_{3} \nonumber \\
Y(I) \geq 3.5 \wedge X(I) > 2 &\Rightarrow c_{4} \nonumber \\
Y(I) \geq 3.5 \wedge X(I) \leq 2 &\Rightarrow c_{1} \nonumber
\end{align}
\begin{figure}[h]
\centering
\centerline{\mbox{\includegraphics{../img/kap1/DT-binary.pdf}}}
\caption{Príklad klasifikačného stromu (binárneho), ktorý popisuje deliace nadroviny z obrázku \ref{fig:decisionPlaneDatab}.}\label{fig:TypesClassify}
\end{figure}
Na tvorbu klasifikačných stromov sa využíva mnoho rôznych techník. 
Prvý publikovaný algoritmus sa nazýva THeta Automatic Interaction Detection (THAID). Neskôr vznikli vylepšené techniky ako Iterative Dichotomiser 3 (ID3), C4.5 (zlepšený ID3), CHi-squared Automatic Interaction Detection (CHAID) a Classification And Regression Tree (CART), ktorý sa dá použiť pri tvorbe klasifikačných aj regresných stromov. Každý z nich používa iný typ hľadania deliacich bodov. Medzi ďalšie, menej známe techniky môžeme uviesť CRUISE, GUIDE, QUEST, o ktorých si je možné prečítať v článkoch \cite{kap1-DecisionTreesUnused1,kap1-DecisionTreesUnused2,kap1-DecisionTreesUnused3}.
Konkrétnejšie budú niektoré z nich popísané v oddieli \ref{kap1:2.7:DTTechniques}. 

\subsection{Regresný strom}\label{kap1:2.4:2.4.1:DTRegression}
Regresný strom je variantou rozhodovacieho stromu navrhnutý na aproximáciu reálnych funkcií na rozdiel od klasifikácie. Strom je veľmi podobný tomu klasifikačnému. Jediným rozdielom je výstupný atribút, ktorého doména je v tomto prípade nekonečná množina ($\mathbb{R},\mathbb{N},\ldots$).

\begin{figure}[h]
\centering
\centerline{\mbox{\includegraphics{../img/kap1/DT-regression.pdf}}}
\caption{Príklad regresneho stromu (binárneho), ktorý predikuje požičky pre žiadateľov.}\label{fig:TypesRegression}
\end{figure}

Tento druh stromov teda používame vtedy, keď chceme predikovať vlastnosť objektu, ktorá je z nekonečnej množiny. Využitie je úplne rovnaké, ako pri klasifikačných stromoch. Regresné stromy majú ale bohatšie spôsoby pri predikcii. Týka sa to hlavne ďalšieho využitia iných druhov modelov v koncových listoch. Takto vytvorené stromy bývajú ale o niečo náročnejšie na konštrukciu kvôli ich komplikovanejšej štruktúre. Vďaka zložitejšej štruktúre stromy dávajú pri regresii zvyčajne lepšie výsledky než iné jednoduchšie metódy.

Na Obrázku \ref{fig:TypesRegression} si je možné všimnúť typický regresný strom. V tomto prípade predikuje výšku priradenej pôžičky žiadateľom podľa ich veku a sociálneho zaradenia (príklad nevychádza z reálnych údajov). Označme žiadateľa/entitu o pôžičku ako $E$. Jednotlivé pravidlá pre tento strom by vyzerali takto:
\begin{align}
vek(E) < 25 \wedge soc.zar(E) = \check{s}tudent &\Rightarrow schv\acute{a}len\acute{a}\  po\check{z}i\check{c}ka = 30000, \nonumber \\
vek(E) < 25 \wedge soc.zar(E) \neq \check{s}tudent &\Rightarrow schv\acute{a}len\acute{a}\  po\check{z}i\check{c}ka = 15000, \nonumber \\
vek(E) \geq 25 \wedge soc.zar(E) = zamestnan\acute{y} &\Rightarrow schv\acute{a}len\acute{a}\  po\check{z}i\check{c}ka = 60000, \nonumber \\
vek(E) \geq 25 \wedge soc.zar(E) = nezamestnan\acute{y} &\Rightarrow schv\acute{a}len\acute{a}\ po\check{z}i\check{c}ka = 5000. \nonumber 
\end{align}
Regresné stromy sa konštruujú rôznymi technikami. Historicky prvým algoritmom na konštrukciu regresných stromov bol algoritmus Automatic Interaction Detection (AID), ktorý sa objavil pár rokov pred THAID. Ďalším, veľmi obľúbeným algoritmom je už spomínaný CART, ktorý dokáže vytvoriť popri regresných stromov aj tie klasifikačné. K menej známym technikám patria M5' a GUIDE.

\section{Kritéria delenia}\label{kap1:2.5:DTSplitCriterias}
Kritéria delenia sú funkcie používané pri konštrukcii stromov. Tieto kritéria majú za úlohu zvoliť vhodný atribút do vnútorného vrcholu stromu.
V rozhodovacích stromoch majú tieto vnútorné vrcholy väčšinou jediný atribút, na ktorom porovnávame jednotlivé inštancie z dát. Kritéria, ktoré vytvárajú takýto typ atribútov nazývame jednorozmerné.

Jednorozmerné kritéria môžu byť ďalej charakterizované podľa
\begin{itemize}
\item pôvodu (teória informácii, závislosť, vzdialenosť),
\item štruktúry, kam patria binárne kritéria, kritéria založené na miere neusporiadanosti (impurity-based) a ich normalizované verzie.
\end{itemize}

Viac príkladov kritérií, doplnené informácie o spomínaných kritériách a podrobnejšie vysvetlenie deliacich kritérií je možné nájsť v knihách \cite{kap1-DataMiningForTrees,kap1-StatisticLearn}. 

V časti \ref{kap1:2.5:2.5.1:Impurity} zavedieme pojem neusporiadanosti, ktorý je nutné poznať pre porozumenie deliacich kritérií. Známe kritérium informačného zisku predvedieme v časti \ref{kap1:2.5:2.5.2:InfoGain}. Kritérium gini predstavíme v časti \ref{kap1:2.5:2.5.3:GiniIndex}. Ďalej v časti \ref{kap1:2.5:2.5.4:GainRatio} zavedieme koeficient zisku. Nakoniec v časti \ref{kap1:2.5:2.5.5:Twoing} zadefinujeme twoing kritérium použité pri tvorbe regresných stromov.
\subsection{Neusporiadanosť}\label{kap1:2.5:2.5.1:Impurity}
Na bližšie zadefinovanie kritérií založených na neusporiadanosti potrebujeme vedieť pár pojmov z teórie informácii.  
\begin{def-sk}[Miera neusporiadanosti]\label{kap1:2.5:2.5.1:ImpurityFunction}
Nech $k \in \mathbb{N}$. Miera/funkcia neusporiadanosti $\gamma: [0,1]^{k}  \rightarrow \mathbb{R}$ je funkcia definovaná pre $k$-tice $P = (p_{1},\ldots,p_{k})$, spĺňajúca $\sum_{i=1}^{k}p_{i} = 1, p_{i} \geq 0$ taká, že:
\begin{itemize}
\item $\gamma(P) \geq 0$,
\item $\gamma(P)$ dosahuje minimum v bodoch, pre ktoré platí, že $p_{i} = 1$, pre nejaké $i$, $1 \leq i \leq k$,
\item $\gamma(P)$ dosahuje maxima v bode $(1/k,\ldots,1/k)$,
\item $\gamma(P)$ je symetrická vzhľadom k hodnotám $p_{1},\ldots,p_{k}$,
\item $\gamma(P)$ je všade diferencovateľná.
\end{itemize}
\end{def-sk}

\begin{def-sk}[Neusporiadanosť množiny]\label{kap1:2.5:2.5.1:ImpuritySet}
Nech $D$ je dátová množina a $C = (c_{1},\ldots,c_{k})$ je množina klasifikačných tried. Ďalej nech $D_{c_{i}}$ je podmnožina dátovej množiny $D$ tvorená vektormi z triedy $c_{i}$, pre $i$=$1,\ldots,k$ a $\gamma: [0,1]^{k}  \rightarrow \mathbb{R}$ je funkcia neusporiadanosti. Potom neusporiadanosť dátovej množiny $D$, označovaná $\gamma(D)$, je definovaná ako
\begin{equation}
\gamma(D) = \gamma
	\left(
	\dfrac
		{\lvert D_{c_{1}}\lvert}
		{\lvert D \lvert},		
	\ldots,
	\dfrac
		{\lvert D_{c_{k}}\lvert}
		{\lvert D \lvert}				
	\right).
\end{equation}
\end{def-sk}

\begin{def-sk}[Zníženie neusporiadanosti množiny]
Nech $X_{i}$ je daný atribút a $D_{1}, \ldots, D_{n}$ sú podmnožiny dátovej množiny $D$, ktoré vznikli rozdelením $D$ podľa atribútu $X_{i}$. Potom výsledné zníženie neusporiadanosti označené ako $\Delta\gamma(X_{i},D) = \gamma(D) - \sum_{i=1}^{s}\dfrac{\lvert D_{i} \lvert}{\lvert D \lvert} \gamma(D_{i})$.
\end{def-sk}

\begin{figure}[h]
\centering
\centerline{\mbox{\includegraphics[width=400pt]{../img/kap1/DT-splitcrit.pdf}}}
\caption{Príklad, ako môžu kritéria delenia rozdeliť dátovú množinu so štyrmi triedami. Farby určujú pomery dát s danou triedou. Obrázok a) oproti b) je vhodnejší, pretože minimalizuje neusporiadanosť v listových vrcholoch týchto stromov.}\label{fig:SplitCriterias}
\end{figure}

Pri indukcii rozhodovacích stromov zvyčajne využívame nejakú hladnú stratégiu, ktorá sa snaží čo najviac znížiť neusporiadanosť celého rozhodovacieho stromu. Na Obrázku \ref{fig:SplitCriterias} môžeme vidieť rozdelenie stromu v jednom z jeho vrcholov. V tomto obrázku preferujeme prípad a), pretože minimalizuje neusporiadanosť podmnožín $D$ v listových vrcholoch.
\subsection{Informačný zisk}\label{kap1:2.5:2.5.2:InfoGain} 
Informačný zisk je jedno z kritérií delenia založené na neusporiadanosti. Definícia kritéria vychádza z predchádzajúcej časti \ref{kap1:2.5:2.5.1:Impurity}, kde za funkciu neusporiadanosti $\gamma$ je dosadená entropia (z teórie informácii). Vzorec kritéria je potom definovaný ako ($D_{X_{i}}$ je v Definícii \ref{kap1:2.3:2.3.2:subsetValueData})

\begin{align}
\infogain(& X_{i},D) = \nonumber \\
& \entropy(C,D) -
\sum_{t \in dom(X_{i})}^{}
\dfrac{\lvert D_{X_{i} = t}\lvert}{\lvert D \lvert} 
\entropy(C,D_{X_{i} = t}). \label{kap1:2.5:2.5.2:InfoGainDef} 
\end{align}
kde
\begin{equation}
\entropy(C,D) = \sum_{c_{i} \in C}^{} \left( -
\dfrac{\lvert D_{c_{i}}\lvert}{\lvert D \lvert} \log
\dfrac{\lvert D_{c_{i}}\lvert}{\lvert D \lvert}\right). 
\end{equation}

Informačný zisk blízko súvisí s metódou maximálnej vierohodnosti. Táto metóda je populárnou hlavne v oblasti štatistiky pri odhadovaní parametrov nejakého štatistického modelu.

Ďalšie informácie o entropii je možné nájsť v knihách spomenutých v týchto častiach. Pre naše potreby nám stačí použiť Wikipédiu \cite{wiki-Entropy}.
\subsection{Gini Index}\label{kap1:2.5:2.5.3:GiniIndex}
Gini index je ďalšie z kritérii založené na neusporiadanosti. Používa sa ako miera diverzifikácie hodnôt vo výstupnom atribúte. Tak ako aj ostatné miery neusporiadanosti nadobúda hodnoty od 0 po 1, kde 0 znamená dokonalú rovnosť tried výstupu a 1 úplnú nerovnosť. Kritérium využívajúce Gini index ako funkciu neusporiadanosti $\gamma$ potom definujeme ako
\begin{align}
\ginindex(X_{i},D) = \gini(C,D) -
\sum_{t \in dom(X_{i})}^{}
\dfrac{\lvert D_{X_{i} = t}\lvert}{\lvert D \lvert} 
\gini(C,D_{X_{i} = t}). \label{kap1:2.5:2.5.3:GiniIndexDef}
\end{align}
kde
\begin{equation}
\gini(C,D) = 1 - \sum_{c_{i} \in C}^{} 
\left(
\dfrac{\lvert D_{c{i}}\lvert}{\lvert D \lvert} 
\right) ^ 2. 
\end{equation}
Taktiež je možné nahliadnuť, že pre binárny prípad (počet tried je rovný dvom) je výpočet kritéria jednoduchší.
\begin{equation}
\gini(C,D) = 2p_{c_{1}}p_{c_{2}}, 
\end{equation}

kde $p_{c_{i}}$ je relatívny počet inštancií, ktorých výstupný atribút má hodnotu $c_{i}$.

\subsection{Koeficient zisku}\label{kap1:2.5:2.5.4:GainRatio} 
Koeficient zisku je ďalším známym kritériom, ktoré pre svoj výpočet využíva informačný zisk. Patrí medzi normalizované kritéria neusporiadanosti. Definované je ako
\begin{align}
\gainratio(X_{i},D) =
\dfrac{\infogain(X_{i},D)}{\entropy(X_{i},D)} \label{kap1:2.5:2.5.4:GainRatioDef} 
\end{align}

Z definície je vidieť, že kritérium nie je definované, keď je entropia rovná nule. Taktiež je kritérium náchylné k atribútom, pre ktoré je entropia veľmi malá. V minulosti bolo ukázané, že toto kritérium dokázalo prekonať normálny informačný zisk v presnosti aj v zložitosti.

\subsection{Twoing kritérium}\label{kap1:2.5:2.5.5:Twoing} 
Medzi menej známe kritéria patrí Twoing kritérium \cite[s.88]{kap1-DataMiningForTrees}. Je to binárne kritérium, ktoré je ako zbytok binárnych kritérii založené na rozdelení domény nejakého atribútu na dve podmnožiny. Zavedené bolo hlavne kvôli Gini indexu a jeho problému s príliš veľkou doménou výstupného atribútu. Kritérium je definované nasledovne 
\begin{align}
\twoing(X_{i},& dom_{1}(X_{i}),dom_{2}(X_{i}),D) = \nonumber \\
& 0.25 
\dfrac{\lvert D_{X_{i} = t \in dom_{1}(X_{i})}\lvert}{\lvert D\lvert}
\dfrac{\lvert D_{X_{i} = t \in dom_{2}(X_{i})}\lvert}{\lvert D\lvert} \nonumber \\
& \left(
\sum_{c_{i} \in C}^{}
\left| 
\dfrac{
\lvert D_{X_{i} = t \in dom_{1}(X_{i})} \cap D_{c_{i}}\lvert }{\lvert D_{X_{i} = t \in dom_{1}(X_{i})}\lvert} -
\dfrac{
\lvert D_{X_{i} = t \in dom_{2}(X_{i})} \cap D_{c_{i}}\lvert }{\lvert D_{X_{i} = t \in dom_{2}(X_{i})}\lvert}
\right|
\right)^2, \label{kap1:2.5:2.5.5:TwoingDef} 
\end{align}

kde $dom_{1}(X_{i})$ a $dom_{2}(X_{i})$ sú podmnožiny domény atribútu $X_{i}$. Keď je výstupný atribút binárny, tak Gini Index a Twoing kritérium sú totožné. Toto kritérium v nebinárnom prípade preferuje atribúty s podobným rozdelením dátovej množiny. Algoritmus CART používa toto deliace kritérium pri indukcii stromov.


\section{Vyhodnotenie kvality stromov}\label{kap1:2.6:DTEvaluation}
Tak ako pri vytváraní modelov metódami strojového učenia, tak aj pri tvorbe stromov je dôležité vyhodnotiť kvalitu vytvoreného modelu. Z predchádzajúcich oddielov vieme, že pri tvorbe stromu používame trénovaciu množinu. Vytvorený strom následne slúži na klasifikovanie nových inštancií. Takto vytvorený strom je možné ohodnotiť podľa určitých kritérií.

Najrozumnejším kritériom sa zdá byť presnosť modelu, no existuje mnoho ďalších. Ďalšími kritériami vyhodnotenia môžu byť zložitosť stromu (výška, počet vrcholov stromu, a pod.) ale aj výpočtová zložitosť budovania stromu, úprav stromu či klasifikovania nových inštancií. Existujú aj také kritéria, ktoré sa zaoberajú robustnosťou\footnote{Schopnosť modelu správne predikovať inštancie, ktoré obsahujú šum, alebo ktorým chýbajú niektoré atribúty.} a stabilitou\footnote{Schopnosť modelu vytvárať podobné/rovnaké stromy pri pozmenení niektorých dát z trénovacej množiny} stromov.

Informácie o rôznych kritériách na vyhodnotenie stromov sú v knihách \cite[s. 364]{kap1-DecisionTree} a \cite[s. 31]{kap1-DataMiningForMasses}. Jednoduché vysvetlenie generalizačnej chyby existuje na Wikipédii \cite{wiki-GeneralizationError}.

Najprv v časti \ref{kap1:2.6:2.6.1:Generalize} zadefinujeme generalizačnú chybu a ukážeme ako súvisí táto chyba s presnosťou modelu. Následne v časti \ref{kap1:2.6:2.6.2:Alternatives} predstavíme niektoré alternatívne metriky pre hodnotenie kvality stromov.

\subsection{Generalizačná chyba}\label{kap1:2.6:2.6.1:Generalize}
Pred zadefinovaním generalizačnej chyby najprv zavedieme pojmy trénovacia chyba a testovacia chyba.
Trénovacia chyba $err$ modelu $M$ nad trénovaciou množinou $D$ je definovaná ako:
\begin{equation}
err(M,D) = \dfrac{1}{\lvert D \lvert}\sum_{(x,y) \in D} L(y, M(x)),
\end{equation}
a testovacia chyba $Err$ modelu $M$ nad testovacou množinou $T$ je definovaná ako:
\begin{equation}
Err(M,T) = \dfrac{1}{\lvert T \lvert}\sum_{(x,y) \in T} L(y, M(x)),
\end{equation}
kde v oboch prípadoch je $M(x)$ klasifikácia inštancie $x$ pomocou modelu $M$ a $L$ je stratová funkcia. V prípade kategoriálneho výstupného atribútu môže byť $L(y,\tilde{y})$ 0-1 stratová funkcia
\begin{equation}
L(y,\tilde{y}) = 
\begin{cases}
1 & \quad \text{ak } y \ne \tilde{y},\\
0 & \quad \text{ak } y = \tilde{y}.\\
\end{cases}
\end{equation}
V prípade spojitého výstupného atribútu môže byť stratová funkcia definovaná ako kvadratická odchýlka:
\begin{equation}
L(y,\tilde{y}) = (y - \tilde{y})^2, \\
\end{equation}
alebo absolútna chyba
\begin{equation}
L(y,\tilde{y}) = \lvert y - \tilde{y} \lvert.
\end{equation}
Generalizačná chyba $GE$ pre určitý model $M$ je definovaná ako pravdepodobnosť, že model nesprávne klasifikuje novú inštanciu (meria ako dobre model klasifikuje nové dáta). Klasifikačná presnosť modelu $ACC$, ďalej označovaná len ako presnosť modelu, je potom definovaná:
\begin{equation}
ACC(M) = 1 - GE(M).
\end{equation}

Presnosť modelu je najdôležitejším kritériom ohodnotenia modelov. Vo väčšine prípadov sa ale nedá použiť, pretože nepoznáme generalizačnú chybu. Namiesto presnej hodnoty generalizačnej chyby sa preto používajú odhady (empirické alebo teoretické). Jednoduchým, ale nesprávnym odhadom generalizačnej chyby je hodnota trénovacej chyby. V prípade nízkej hodnoty trénovacej chyby môže model pôsobiť na prvý pohľad idylicky, no jeho generalizačné schopnosti môžu byť veľmi nízke. Tento nepriaznivý stav sa nazýva preučenie modelu. Na Obrázku \ref{fig:overfit} je vidieť vývoj generalizačnej a trénovacej chyby vzhľadom k veľkosti stromu. Lepším odhadom generalizačnej chyby je testovacia chyba, viď nižšie (empirický odhad).

\begin{figure}[h]
\centering
\centerline{\mbox{\includegraphics{../img/kap1/DT-overfit.pdf}}}
\caption{Vývoj generalizačnej a trénovacej chyby vzhľadom k veľkosti stromu. Toto je typický graf pre preučený model.}\label{fig:overfit}
\end{figure}

Medzi najčastejšie využívané odhady generalizačnej chyby patria tie empirické. Pri tomto type odhadov sa vstupná množina rozdelí na trénovaciu a testovaciu podmnožinu. Rozdelenie býva najčastejšie náhodné a v pomere 2:1. Väčšia množina je trénovacia a menšia zase testovacia. 
Model, ako už vieme, vytvárame pomocou trénovacej množiny. Testovaciu množinu využívame na vyhodnotenie kvality stromu a na získanie testovacej chyby $Err$. Táto testovacia chyba býva lepším odhadom generalizačnej chyby, než je tá trénovacia. Nevýhodou tohoto prístupu je menšia veľkosť trénovacej množiny. Pri veľmi malej vzorke vstupných dát sa trénovacia aj testovacia podmnožina vytvára pomocou výberu s opakovaním. Na Obrázku \ref{fig:testtrain} je zobrazený vývoj trénovacej a testovacej chyby. Prerušovaná čiara v minime testovacej chyby predstavuje najlepší strom, ktorý môžeme získať pre vstupné dáta (nepreučený model s čo najlepšími generalizačnými schopnosťami).

\begin{figure}[h]
\centering
\centerline{\mbox{\includegraphics{../img/kap1/DT-testtrain.pdf}}}
\caption{Vývoj testovacej a trénovacej chyby vzhľadom k veľkosti stromu. Prerušovaná čiara predstavuje najlepší vytvorený strom vzhľadom k dostupným dátam.}\label{fig:testtrain}
\end{figure}

Ďalším, veľmi známym postupom na odhad generalizačnej chyby je $n$-násobná krížová validácia. Pri tejto technike vytvárame $n$ modelov rovnakého typu. Dáta sú rozdelené na $n$ navzájom disjunktných podmnožín, najlepšie rovnakých veľkostí. Model testujeme vždy na jednej z $n$ častí. Ostatných $n-1$ častí slúži ako trénovacia množina (rozdelenie množín je teda v pomere $n-1:1$). Testovacie chyby $Err_{i}$, kde $i \in \{1,\ldots,n\}$, získané z každého z $n$ modelov môžeme následne použiť na aproximáciu generalizačnej chyby:
\begin{equation}
GE \approx \dfrac{1}{n} \sum_{i=1}^{n} Err_{i}.
\end{equation}
Následne nie je ťažké dopočítať rozptyl a smerodajnú odchýlku tohoto odhadu. Pomocou týchto údajov je potom možné zistiť interval, v ktorom sa s určitou pravdepodobnosťou nachádza skutočná generalizačná chyba. Tento interval je pri použití $n$-násobnej krížovej validácie kvalitnejší ako v prípade odhadu pomocou jednej testovacej chyby.

Viac informácií o ďalších odhadoch je v knihe \cite[s. 56]{kap1-DataMiningForTrees} a vo voľne dostupných zdrojoch \cite{wiki-CrossValidation,online-validation}.

\subsection{Alternatívne metriky založené na matici chybovosti}\label{kap1:2.6:2.6.2:Alternatives}
Kritérium presnosti, ktoré sme spomínali v predchádzajúcej časti je síce najdôležitejším kritériom na vyhodnotenie modelu, no v niektorých prípadoch je jeho použitie úplne nevhodné. Príkladom, kedy je použitie kritéria presnosti nevhodné je v prípade nerovnomerného rozdelenia hodnôt výstupného atribútu. Kritérium presnosti by v prípade tejto nerovnomernosti preferoval modely, ktoré by vždy predikovali majoritnú hodnotu výstupného atribútu. V tomto prípade je nutné použiť nejaké z alternatívnych metrík.  
Nadchádzajúce definície kritérií dokážu pracovať aj s $n$-árnym výstupným atribútom, no ich výpočet bude vždy uvedený pre jednu konkrétnu hodnotu výstupného atribútu. Postup na spočítanie hodnoty globálneho kritéria (pre celý model) je rovnaký pre všetky ostatné kritériá a bude spomenutý na konci tejto časti. Lepšie vysvetlenie s dodatočnými kritériami sa nachádza v scikit príručke \cite{online-NAryConfusion}).

\begin{def-sk}[Matica chybovosti]
Nech hodnoty výstupného atribútu sú z konečnej množiny $C = \{c_{1},\ldots,c_{n}\}$. Ďalej nech $M$ je prediktívny model a $MC$ je matica nezáporných čísel, kde riadky predstavujú predikované triedy modelom $M$, stĺpce skutočné hodnoty výstupného atribútu a jednotlivé položky $c_{ij}$ predstavujú počet takých inštancií, ktorých predikovaná hodnota výstupného atribútu je $c_{i}$ a skutočná hodnota je $c_{j}$. Maticu $MC$ nazveme maticou chybovosti a slúži na posudzovanie kvality modelu $M$ (angl. confusion matrix, Obrázok \ref{fig:Confusion}).
\end{def-sk}

\begin{figure}[h]
\centering
\centerline{\mbox{\includegraphics{../img/kap1/DT-confusion.pdf}}}
\caption{Na obrázku sú príklady matíc chybovosti pre binárny a $k$-árny výstupný atribút. Inštancie sú zaraďované do jednotlivých buniek matice podľa toho, aký je skutočný a predikovaný výstup tejto inštancie. V binárnom prípade bývajú štyri bunky tejto matice aj pomenované (z angl. TP = true positive, FP = false positive, TN = true negative, FN = false negative)}\label{fig:Confusion}
\end{figure}

\begin{def-sk}[Kritérium citlivosti]\label{kap1:2.6:2.6.2:Sensitivity}
Citlivosť (angl. sensitivity ale aj recall) na hodnotu $c_{i}$ je alternatívna metrika, ktorá kvantifikuje kvalitu modelu pri rozpoznávaní inštancií, ktorých skutočná hodnota výstupného atribút je $c_{i}$.
\begin{equation}
\mbox{Citlivosť}_{c_{i}} = \dfrac{c_{ii}}{\sum_{j=1}^{k} c_{ji}}
\end{equation}
\end{def-sk}

\begin{def-sk}[Kritérium precíznosti]\label{kap1:2.6:2.6.2:Sensitivity}
Precíznosť (angl. precision) pre hodnotu $c_{i}$ je alternatívna metrika, ktorá popisuje kvalitu modelu podľa toho, koľko klasifikovaných inštancií na hodnotu $c_{i}$ je aj naozaj správne klasifikovaných.
\begin{equation}
\mbox{Precíznosť}_{c_{i}} = \dfrac{c_{ii}}{\sum_{j=1}^{k} c_{ij}}
\end{equation}
\end{def-sk}

\begin{def-sk}[Kritérium špecificity]\label{kap1:2.6:2.6.2:Specificity}
Špecificita (angl. specificity) pre hodnotu $c_{i}$ je metrikou, ktorá kvantifikuje kvalitu modelu pri rozpoznávaní inštancií, ktorých skutočná hodnota výstupného atribút nie je $c_{i}$.
\begin{equation}
\mbox{Špecificita}_{c_{i}} = \dfrac{\sum_{l \ne i}^{k}\sum_{j \ne i}^{k} c_{lj}}{\sum_{l = 1}^{k}\sum_{j \ne i}^{k} c_{lj}}
\end{equation}
\end{def-sk}

\begin{def-sk}[F-metrika]\label{kap1:2.6:2.6.2:Specificity}
F-metrika (angl. f-measure) pre hodnotu $c_{i}$ je metrikou, ktorá kvantifikuje kvalitu modelu podľa precíznosti a citlivosti. Nech $P_{c_{i}}$ je precíznosť a $S_{c_{i}}$ je citlivosť pre hodnotu $c_{i}$, potom:
\begin{equation}
\mbox{F-metrika}_{\beta, c_{i}} = (1 + \beta^2) \dfrac{P_{c_{i}} \times R_{c_{i}}}{\beta^2 \times P_{c_{i}} + R_{c_{i}}}
\end{equation}
\end{def-sk}

Doteraz boli kritéria počítané pre konkrétnu hodnotu výstupného atribútu. Na spočítanie hodnoty globálneho kritéria pre celý model je nutné použiť ďalšie výpočtové techniky. Nech $C = \{c_{1},\ldots,c_{n}\}$ je konečná množina hodnôt pre výstupný atribút, $D$ je dátová množina a $krit_{c_{i}}$ je hodnota ľubovoľného kritéria pre hodnotu výstupného atribútu $c_{i}$. Potom na spočítanie globálnej hodnoty ľubovoľného kritéria $krit$ (citlivosti, precíznosti, $\ldots$) môžeme napríklad použiť priemer podľa veľkosti množiny $C$:
\begin{equation}
krit = \dfrac{1}{\lvert C \lvert} \sum_{c_{i} \in C} krit_{c_{i}},
\end{equation}
alebo vážený priemer
\begin{equation}
krit = \dfrac{1}{\lvert C \lvert} \sum_{c_{i} \in C} \lvert D_{c_{i}} \lvert \times krit_{c_{i}}.
\end{equation}
Ďalšie techniky na spočítanie globálnej hodnoty kritéria sú na scikit stránke \cite{online-NAryConfusion}. 

Jedno z kritérií, ktoré môžeme dopočítať z matice chybovosti modelu $M$ je dokonca aj odhad presnosti $ACC$ (tak ako sme o ňom písali v predchádzajúcej časti):
\begin{equation}
ACC(M) \approx \dfrac{\sum_{i=1}^{k} c_{ii}}{\sum_{i=1}^{k}\sum_{j=1}^{k} c_{ij}} \nonumber
\end{equation}
Z matice chybovosti je pravdaže možné získať viac kritérií, než sme popísali v tejto časti. Viac informácii a zoznam ďalších kritérií poskytuje Wikipédia \cite{wiki-Confusion}.

\subsection{Alternatívne metriky založené na zložitosti stromu}\label{kap1:2.6:2.6.3:TreeLook}
Veľa techník z dobývania znalostí, ako sú SVM alebo neurónové siete vytvárajú modely, ktoré sa síce snažia zlepšiť kritérium presnosti, no väčšinou fungujú ako čierne skrinky, o vnútornej štruktúre ktorých nevieme skoro nič. Rozhodovacie stromy, ako uź vieme, získali popularitu v rôznych odvetviach hlavne kvôli dvom výhodám: jednoduchosti a zrozumiteľnosti modelu. V prípade príliš veľkých stromov sú ale tieto výhody do istej miery redukované. 

V oblasti medicíny je dôležitou vlastnosťou (popri presnosti modelu) schopnosť interpretovať rozhodnutia, čo priamo súvisí s jednoduchosťou a zrozumiteľnosťou rozhodovacích stromov. Tieto dve vlastnosti stromu uľahčujú lekárovi diagnózu pacientov a zároveň mu dovoľujú nahliadnuť na vnútornú štruktúru stromu. Z nej je vidieť, ako sú tvorené a ako kvalitné sú rozhodnutia tohoto stromu (počet správne klasifikovaných inštancií v listoch určuje kvalitu rozhodnutia).

Najznámejšími a najobľúbenejšími metrikami založenými na zložitosti stromov sú:
\begin{itemize}
\item hĺbka stromu, ktorá je definovaná ako maximálna dĺžka cesty od koreňa k nejakému listu,
\item počet vrcholov stromu,
\item počet listov stromu,
\end{itemize}
pričom medzi ďalšie metriky môžu patriť ľubovoľné vlastnosti stromov/grafov.

\section{Techniky tvorby stromov}\label{kap1:2.7:DTTechniques}
Cieľom tejto časti je predviesť najznámejšie techniky tvorby stromov (regresných aj klasifikačných). Tieto techniky spadajú pod oblasť učenia s učiteľom, ktorá bola bližšie popísaná v oddieli \ref{kap1:2.2:2.2.4:Supervised}. Jedná sa o algoritmy, ktoré vytvárajú strom z trénovacej množiny postupne od koreňa. Dáta sú v algoritme rozdelené podľa deliaceho kritéria na podmnožiny. Ďalej pokračujeme rekurzívnym volaním toho istého algoritmu na každú z týchto podmnožín. V obore dobývania znalostí sa celý tento proces nazýva indukcia rozhodovacích stromov zhora nadol, anglicky top down induction of decision trees (TDIDT). Indukčné algoritmy sú príkladom hladných techník. V praxi sú jedným z najpoužívanejších stratégii pri trénovaní rozhodovacích stromov. V tejto časti sme čerpali z kníh \cite{kap1-DataMiningForTrees,kap1-DataMiningAndAnalysis} pričom na rozšírenie ďalších poznatkov sme použili online zdroje \cite{online-SplitCriterias,online-SplitCriteriasMatter,online-DTLectures}.

Najskôr v časti \ref{kap1:2.7:2.7.2:Generic} ukážeme generický algoritmus na indukciu stromov. Ďalej v časti \ref{kap1:2.7:2.7.1:Pruning} predstavíme techniku orezávania stromov, ktorá sa používa pri zlepšení generalizačných schopností stromov.  V ďalších častiach rozoberieme tri najznámejšie indukčné algoritmy na tvorbu stromov. Konkrétne v časti \ref{kap1:2.7:2.7.3:ID3} ukážeme algoritmus ID3. Následne v časti \ref{kap1:2.7:2.7.4:C4.5} predstavíme vylepšený algoritmus C4.5. Pre regresné typy stromov predvedieme v časti \ref{kap1:2.7:2.7.5:CART} obľúbený algoritmus CART.

\subsection{Generický indukčný algoritmus}\label{kap1:2.7:2.7.2:Generic}
Indukčné algoritmy majú za úlohu vytvoriť rozhodovací strom zo zadanej trénovacej množiny $D$. Hlavnou úlohou pri tvorbe stromov je minimalizovať generalizačnú chybu (\ref{kap1:2.6:2.6.1:Generalize}). Nie vždy sa jedná len o minimalizovanie. Algoritmus sa môže snažiť o vytvorenie stromu s čo najnižšou výškou alebo tiež s obmedzeným počtom vrcholov.

Tvorba optimálneho rozhodovacieho stromu je ťažkou úlohou. Viac o tom v knihe \cite[s.51]{kap1-DataMiningForTrees}. Indukčné algoritmy sú dvoch typov. Tie, ktoré vytvárajú stromy zhora nadol a zdola nahor. Medzi algoritmy typu zhora nadol patria ID3, C4.5, CART a mnoho ďalších. Tvorba stromu je pri niektorých algoritmoch (C4.5 a CART) spojená aj s jeho orezaním. Väčšinou na začiatku existuje jeden vrchol stromu (koreň), ktorému odpovedá celá trénovacia množina $D$. Následníci tohoto koreňa ďalej zodpovedajú podmnožinám trénovacej množiny. Týmto postupom stále delíme trénovaciu množinu, a tým vytvárame nové vrcholy stromu. Delenie prebieha až dokým nie je pravdivé niektoré zo stanovených kritérií zastavenia. Kostra takejto tvorby stromu je popísaná v Algoritme  \ref{fig:genericAlgoritm}.

\begin{algorithm} 
\floatname{algorithm}{Algoritmus}
\caption{Generický algoritmus na tvorbu stromov, z ktorého vychádzajú známe algoritmy ID3,C4.5,a pod.}\label{fig:genericAlgoritm}
$D$ - Trénovacia množina \\
$X$ - Množina atribútov \\
$Y$ - Výstupný atribút \\
$Del\_Krit$ - Kritérium pre delenie \\
$Stop\_Krit$ - Kritérium na zastavenie tvorby stromu \\
$T$ - Vytvorený strom 
\begin{algorithmic}
\Function{IndukciaStromu}{$D$,$X$,$Y$,$Del\_Krit$, $Stop\_Krit$}
\State $\var{T} \gets $ Strom s jedným vrcholom (koreň)
\If {$Stop\_Krit(D)$}:	  
\State $\var{T} \gets $ list s najčastejšou hodnotou atribútu $Y$ v $D$.
\Else
\State $\var{A} \gets min_{X_{i}}Del\_Krit(X_{i},D)$
\State $attr(\var{T}) \gets A$
\EndIf
\ForAll{$\var{v_{i}} \in dom(A)$}:
\State $PodStrom_{i}(T) \gets $ \parbox[t]{280pt}{\textsc{IndukciaStromu}($D_{X_{i} = v_{i}}$,$X$,$Y$,$Del\_Krit$, $Stop\_Krit$)} 
\State $Hrana_{i}(T) \gets v_{i}$  
\EndFor \\
\Return{T}
\EndFunction
\end{algorithmic}
\end{algorithm}

%\begin{figure}[h]
%\lstset{inputencoding=utf8,extendedchars=true,basicstyle=\ttfamily\small,literate={á}{{\'a}}1 {é}{{\'e}}1 {ž}{{\v{z}}}1 {ň}{{\v{n}}}1 {š}{{\v{s}}}1 {č}{{\v{c}}}1 {ď}{{\v{d}}}1 {í}{{\'i}}1 {ý}{{\'y}}1 {ú}{{\'u}}1}
%\begin{lstlisting}[mathescape]
%IndukciaStromu($D$,$A$,$y$,$DeliaceKrit\acute{e}rium$,$Krit\acute{e}riumZastavenia$)
%$D$ - Trénovacia množina
%$A$ - Vstupné príznaky
%$y$ - Výstupný atribút
%$DeliaceKrit\acute{e}rium$ - Kritérium pre delenie
%$Krit\acute{e}riumZastavenia$ - Kritérium na zastavenie tvorby stromu
%
%Vytvor strom $T$ s jedným vrcholom (koreň)
%IF Krit\acute{e}riumZastavenia(D) THEN
%	Sprav z $T$ list s najčastejšou
%	hodnotou atribútu $y$ v $D$.
%ELSE
%	Vyber atribút $a \in A$ s najlepšou
%	hodnotou $DeliaceKrit\acute{e}rium(a,D)$.
%	Aktuálnemu vrcholu priraď atribút $a$.
%	FOR $v_{i} \in A$:
%		Nastav $PodStrom_{i}$ = IndukciaStromu($D_{v_{i}}$,$A$,$y$,_,_).
%		Spoj vrchol $t_{T}$ a $PodStrom_{i}$ s hranou označenou
%		$v_{i}$.
%	END FOR
%END IF
%RETURN OrezanieStromu(D,T,y).
%\end{lstlisting}
%\caption{Generický algoritmus z ktorého vychádzajú známe algoritmy ID3,C4.5,$\ldots$. Prevzaté a prepísané z \cite[s.52]{kap1-DataMiningForTrees}}\label{fig:genericAlgoritm}
%\end{figure}

Existujú rôzne kritéria zastavenia, ktoré prerušia vytváranie stromu. Tými najznámejšími sú

\begin{itemize}
\item Všetky inštancie v trénovacej množine majú jednotný výstupný atribút $y$.
\item Strom dosiahol maximálnu výšku.
\item Hodnota najlepšieho deliaceho kritéria je menšia než určitý prah.
\item Počet inštancií v danom uzle je menší než dopredu zadaná konštanta.
\end{itemize}

\subsection{Orezávanie stromov}\label{kap1:2.7:2.7.1:Pruning}
Pri vytváraní stromov je nutné dať pozor nato, aké zastavovacie kritérium použijeme. Pri málo obmedzujúcom kritériu zastavenia dostávame stromy malej veľkosti, no trénovacia chyba takéhoto stromu je príliš vysoká. Na opačnej strane, príliš obmedzujúce kritérium zastavenia donúti algoritmus vytvoriť veľký preučený strom (malá trénovacia chyba, vysoká generalizačná chyba).

Pri preučených stromoch je používanou taktikou orezanie stromu. Strom je typicky zbavovaný prebytočných podstromov, ktoré zhoršujú generalizáciu (keď používame kritérium presnosti). Orezanie stromu je skvelým nástrojom pri dodržaní pravidla Occamovej britvy\footnote{Filozofický prístup založený na preferovaní jednoduchších modelov pri zachovaní približne rovnakej kvality.}.

Existuje mnoho techník na orezanie stromov. Typickým postupom je 
\begin{enumerate}
\item Vytvorenie stromu nejakým indukčným algoritmom.
\item Zmenšenie stromu aplikáciou nejakej techniky orezávania.
\end{enumerate}
Orezávanie stromu sa dá generalizovať pre ľubovoľné kritérium $\epsilon$.
Tieto techniky vykonávajú prechod stromu zhora nadol (od koreňa k listom) alebo naopak (od listov ku koreňu). Keď strom po orezaní vylepšil predom dané kritérium kvality stromu $\epsilon$, tak je orezaný strom prijatý.
Hodnotu vyhodnocovacieho kritéria $\epsilon$ väčšinou počítame na úplne novej, orezávacej množine nezávislej na trénovacej a testovacej množine. Pri malej veľkosti vstupných dát je odporúčané použiť krížovú validáciu alebo iné metódy.
\subsubsection{Orezanie stromu zmenšovaním chyby}
Rozšírená a veľmi jednoduchá technika orezania stromu. Patrí medzi techniky zhora nadol, ktorá pri každom orezaní zaručuje zmenšenie chyby. Pre každý vrchol stromu zisťujeme, či po jeho nahradení za list (s maximálnou hodnotou výstupného atribútu pre dáta v tomto vrchole), znížime hodnotu kritéria kvality stromu. Hodnotu kritéria počítame na orezávacej množine. Celý proces pokračuje dokým sa takýto vrchol nenájde.
\subsubsection{Orezanie podľa zložitosti stromu}
Táto technika pracuje v dvoch krokoch. V prvom kroku vybudujeme pomocou trénovacích dát stromy $T_{0},\ldots,T_{n}$, kde $T_{0}$ je strom bez nejakého orezania a $T_{n}$ je koreň. V druhom kroku vyberieme taký strom $T_{i}$, ktorý maximalizuje kritérium kvality stromu $\epsilon$. 

Množina stromov je budovaná postupne. Strom $T_{i+1}$ je vytvorený zo stromu $T_{i}$ tak, že niektoré podstromy nahradíme listami. Podstromy na orezanie sú vybrané podľa toho, ako moc zlepšujú (pre generalizačnú chybu minimalizujeme, pre iné kritéria maximalizujeme) hodnotu:
\begin{equation}
\delta = \dfrac{\epsilon(orezanie(T_{i},t),D) - \epsilon(T_{i},D)}{\lvert listy(T_{i}) \lvert - \lvert listy(orezanie(T_{i},t)) \lvert}, \nonumber
\end{equation}
pričom $\epsilon(T_{i},D)$ je hodnota daného kritéria pre strom $T_{i}$ a dáta $D$. Ďalej $listy(T_{i})$ je množina listov stromu $T_{i}$ a $orezanie(T_{i},t)$ je funkcia, ktorá zmení vrchol $t$ stromu $T_{i}$ na list.

Viac informácií a ďalšie techniky orezávania stromov sú v knihe \cite[s. 72]{kap1-DecisionTree}.

%\begin{algorithm} 
%\floatname{algorithm}{Algoritmus}
%\caption{Pseudo-algoritmus popisujúci proces orezania stromov. Funkcia orezanie(T,t) }\label{fig:pruneAlgorithm}
%\begin{algorithmic}
%\Function{OrezanieStromu}{$D$,$T$,$Y$}
%\Repeat 
%\State $t \gets$ \parbox[t]{350pt}{vyber vrchol stromu T, ktorý po orezaní maximalizuje predom zvolené vyhodnocovacie kritérium}
%\If{$t \neq \emptyset$}:
%\State $T \gets$ orezanie($T$,$t$)
%\EndIf
%\Until{$t = \emptyset$} \\
%\Return{$T$}
%\EndFunction
%\end{algorithmic}
%\end{algorithm}

\subsection{ID3}\label{kap1:2.7:2.7.3:ID3}
ID3 algoritmus je najznámejší a najjednoduchší z indukčných algoritmov. Deliacim kritériom je Informačný zisk (\ref{kap1:2.5:2.5.2:InfoGainDef}). Kritériom zastavenia je nulový informačný zisk alebo dosiahnutie homogénnosti výstupného atribútu. Algoritmus nevyužíva žiadne techniky orezávania, nedokáže riešiť problémy s chýbajúcimi hodnotami v atribútoch a nevie pracovať s numerickými atribútmi.

Najväčšou výhodou ID3 je jeho jednoduchosť. Zvyčajne je kvôli tomu využívaný pre vzdelávacie účely. 

Nevýhody tohoto algoritmu presahujú spomenuté výhody. Stromy vytvorené týmto algoritmom sa nachádzajú v nejakom lokálnom optime (hladný algoritmus).
Vytvorené stromy bývajú väčšinou malé, ale keď algoritmus vytvorí väčší strom tak dochádza veľmi často k preučeniu. Preto pri tomto druhu algoritmu sú preferovanejšie menšie stromy. Algoritmus nepodporuje numerické hodnoty a pre prácu s nimi je nutné predspracovať dáta do nejakých diskrétnych skupín (hodnoty menšie ako $x$ budú v skupine $a$, ostatné v skupine $b$).
\subsection{C4.5}\label{kap1:2.7:2.7.4:C4.5}
C4.5 je vylepšením algoritmu ID3. Obidva majú rovnakého autora. Algoritmus používa ako deliace kritérium koeficient zisku (\ref{kap1:2.5:2.5.4:GainRatioDef}). Vytváranie stromu končí, keď počet inštancií pri delení je menší než zvolená konštanta. Zároveň využíva techniky orezávania a dokáže pracovať aj s numerickými atribútmi. C4.5 vie popri tomu  pracovať aj s chýbajúcimi hodnotami atribútov.

Indukcia pomocou C4.5 poskytuje oproti ID3 veľa zlepšení. Pri orezávaní odstraňuje vetvy, ktoré neprispievajú k presnosti a nahradzuje ich listami. Inštanciám môžu chýbať niektoré hodnoty atribútov. Numerické atribúty sú riešené rozdelením domény daného atribútu na dve podmnožiny podľa deliaceho bodu. Tento deliaci bod je zvolený tak, aby maximalizoval koeficient zisku.

Jedným z ďalších vylepšení C4.5 je ešte o niečo novší, komerčný algoritmus C5.0. Popri výhodám z C4.5 obsahuje ešte ďalšie zlepšenia, ako je rýchlosť výpočtu a zlepšenie práce s pamäťou. Ďalším zlepšením je vstavané využitie techniky boosting, ktorá dokáže zvýšiť prediktívne schopnosti modelu.

Implementovanú verziu algoritmu C4.5, nazývanú J48 môžeme nájsť už v spomínanom nástroji Weka.

\subsection{CART}\label{kap1:2.7:2.7.5:CART}
CART je hladný algoritmus, ktorý slúži na tvorbu klasifikačných aj regresných stromov. Autorom algoritmu je Breiman. Vytvorený strom je v tomto prípade vždy binárny (každý vnútorný vrchol ma dvoch potomkov). Algoritmus využíva Twoing kritérium (\ref{kap1:2.5:2.5.5:TwoingDef}) ako deliace kritérium. CART používa na orezávanie stromu metriku, ktorá berie do úvahy počet listov spolu s chybou modelu (\cite{wiki-costcompprune}, \cite[s.382]{kap1-DecisionTree}).

Dôležitou vlastnosťou algoritmu je vytváranie regresných stromov. Vytvorené stromy obsahujú v listoch reálne hodnoty. V tomto prípade sú deliace body vyberané tak, aby minimalizovali kvadratickú odchýlku predikovaných hodnôt od správnych hodnôt na dátovej množine. V listových vrcholoch je hodnota rovná váženému priemeru výstupnej hodnoty dát v tomto vrchole.

\section{Využitie rozhodovacích stromov pri predikcii}\label{kap1:2.8:DTUsage}
V tomto oddieli zhrnieme výhody a nevýhody rozhodovacích stromov, ktoré priamo ovplyvňujú využiteľnosť tohoto modelu v praxi. Zároveň predstavíme projekty, ktoré prakticky využívajú techniku rozhodovacích stromov. Knihy s informáciami, ktoré boli použité v tejto časti sú \cite{kap1-DataMiningAndAnalysis,kap1-DataMiningForTrees}. Informácie s praktickým použitím rozhodovacích stromov sme čerpali zo stránok príslušných projektov \cite{online-astronomy} a \cite{online-psychoterapy}.

V časti \ref{kap1:2.8:2.8.1:AdvAndDis} popíšeme výhody a nevýhody rozhodovacích stromov pri ich využití v praxi. V ďalších častiach predstavíme dva projekty, ktoré využívajú rozhodovacie stromy pri predikcii. Filtrovanie šumu v obrázkoch získaných z Hubblovho vesmírneho ďalekohľadu bude popísané v časti \ref{kap1:2.8:2.8.2:Hubble}. Využitie rozhodovacích stromov pre klinickú prax predvedieme v poslednej časti \ref{kap1:2.8:2.8.3:Clinical}.

\subsection{Výhody a nevýhody stromov}\label{kap1:2.8:2.8.1:AdvAndDis}
V predchádzajúcich oddieloch sme spomínali rôzne výhody a nevýhody rozhodovacích stromov. Medzi výhody môžeme zaradiť:
\begin{itemize}
\item Rozhodovacie stromy majú jednoduchú reprezentáciu, z ktorej je vidieť chovanie stromu. Pri malom počte listov je reprezentácia ľahko spracovateľná aj neprofesionálnym užívateľom.
\item Model je dostatočne bohatý na to, aby sa naňho dal previesť ľubovoľný iný diskrétny klasifikátor.
\item Cesty od koreňa k listom predstavujú jednotlivé pravidlá. Existuje teda jednoduchý prevod medzi stromom a množinou pravidiel.
\item Model rozhodovacích stromov je zaraďovaný medzi neparametrické techniky a teda neuvažuje žiadne predpoklady o štruktúre klasifikátora či rozdelení dát.
\item Rozhodovacie stromy dokážu pracovať s dátami, ktoré obsahujú chyby alebo ktorým chýbajú hodnoty niektorých atribútov.
\item Rozhodovacie stromy vedia pracovať s kategoriálnymi atribútmi, tak ako aj s numerickými.
\end{itemize}

K nevýhodám rozhodovacích stromov patrí:
\begin{itemize}
\item Hladné techniky tvorby stromov môžu uviaznuť v lokálnom optime. Stromy sú taktiež náchylné na preučenie.
\item Tvorba stromu hladnými technikami sa môže zamerať na nerelevantné atribúty a šum v dátach.
\item Krátkozrakosť algoritmu spôsobená tým, že delenia v jednotlivých vrcholoch pracujú iba s priamymi potomkami vrcholu.
\item Väčšina algoritmov požaduje, aby bol výstupný atribút kategoriálny (C4.5, ID3, \ldots).
\item Slabinou môžu byť metódy riešenia problému s chýbajúcimi hodnotami niektorých atribútov. Trénovacie dáta s chýbajúcimi hodnotami môžu byť napríklad odstránené z trénovacej množiny. Keď hodnota atribútu nechýba náhodou a nevyplnenie popisuje nejakú vlastnosť, tak je odstraňovanie inštancií úplne nevhodné (napr. neuvedený plat u dátovej množiny osôb môže znamenať, že osoba má príliš vysoký/nízky plat a nechcela ho uviesť).
\item Problém fragmentácie pri rozdeľovaní dát vo vrcholoch. K tomuto problému dochádza v prípade, keď dátové množíny vo vrcholoch rozdelíme na rovnako veľké podmnožiny. Strom, ktorý by takto rozdeľoval dátovú množinu veľkosti $n$, obsahuje na ceste z koreňa do listu iba $\log n$ vrcholov, čo je ekvivalentné s otestovaním maximálne $\log n$ atribútov. Toto býva veľký problém pri existencii väčšieho počtu významných atribútov.
\end{itemize}

\subsection{Filtrovanie šumu z Hubblovho ďalekohľadu}\label{kap1:2.8:2.8.2:Hubble}
Ako prvý príklad aplikácie rozhodovacích stromov stručne predstavíme jeden projekt NASA. Cieľom autorov tohoto projektu \cite{online-astronomy} bolo vytvoriť klasifikačný model, ktorý by dokázal na obrázkoch získaných z Hubblovho vesmírneho ďalekohľadu rozlišovať medzi hviezdami a šumom. Šum, ktorý je pri takto získaných fotkách veľmi častý, je spôsobený kozmickým žiarením. 

Trénovacie dáta pozostávajú z príznakov $x_{1},\ldots,x_{20}$. Hodnoty jednotlivých príznakov sa počítajú z okolia pixelu veľkosti $3 \times 3$ a sú založené na meraní rôznych svetelných úkazov s využitím poznatkov o rôznych druhoch žiarenia. Výslednú trénovaciu množinu použili na konštrukciu rozhodovacieho stromu, ktorým bol predikovaný typ žiarenia -- dokázali rozlíšiť žiarenie hviezd od šumu. Strom, ktorý vytvorili nazvali OC1. Model bol atypický v tom, že uvažoval aj také rozdeľovacie nadroviny, ktoré nie sú rovnobežné s hlavnými osami. Viac o algoritme je priamo v článku \cite[str. 281]{online-astronomy}.

\begin{figure}[h]
\centering
\centerline{\mbox{\includegraphics{../img/kap1/DT-hubble.pdf}}}
\caption{Získaný model klasifikujúci vstupný vektor s 20 príznakmi (Čísla v elipsách znamenajú delenia vo vrcholoch. Percentá pod vrcholmi predstavujú relatívny pomer počtu hviezd ku kozmickému žiareniu). Obrázok modelu je prebraný a prepísaný z \cite[str. 284]{online-astronomy}}\label{fig:HubbleTree}
\end{figure}

Na testovacích dátach dosahovali modely, vytvorené metódou OC1, presnosť až 95\%. Vytvorené stromy boli navyše veľmi kompaktné. Príklad takto vytvoreného stromu je zobrazený na Obrázku \ref{fig:HubbleTree}. Experimenty, popisované v článku uvádzajú, že zlepšením metód na eliminovanie šumu sa môže ďalej vylepšiť presnosť vytvorených klasifikátorov.

\subsection{Rozhodovacie stromy pre klinickú prax}\label{kap1:2.8:2.8.3:Clinical}
Ako druhý príklad praktického nasadenia rozhodovacích stromov uvádzame ich použitie v medicíne. Cieľom projekte bolo uľahčiť prácu doktorom a zdravotnému personálu vytvorením modelu, ktorý by varoval pred neúspechom liečby a pomocou správnych rozhodnutí odporúčal, ako čo najlepšie predchádzať takémuto neúspechu. 

Dáta, pre účely vytvorenia modelu, boli získané z iného projektu (Project TR-EAT), ktorý skúmal vzťahy medzi dĺžkou/intenzitou liečby poruchy prijímania potravy a výsledkom liečby. Z dát bola vybratá vzorka ľudí diagnostikovaná bulímiou nervosa ($n$ = 630). Pri výbere dát využívali hlavne údaje o prijatí, liečebnej stratégii a reakcii na liečbu. Na výslednú trénovaciu množinu bol použitý algoritmus CART.

\begin{figure}[h]
\centering
\centerline{\mbox{\includegraphics[width=385pt]{../img/kap1/DT-clinical.pdf}}}
\caption{Získaný model klasifikujúci výsledok liečby. Čísla v elipsách predstavujú veľkosti skupín získané po delení. Percentá pod vrcholmi predstavujú relatívny počet ľudí s lepším verzus horším výsledkom. SCLGSI je skratka pre výsledok testu Symptom Checklist-90-R Global Severity Index. Obrázok modelu je prebraný a prepísaný z \cite[s. 454]{online-psychoterapy}}\label{fig:ClinicalTree}
\end{figure}

Vytvorený strom metódou CART obsahoval desať listových vrcholov a jeho chybovosť bola 18\%. Zo 448 pacientov, ktorým bol predpovedaný zlý výsledok malo 86\% skutočne zlý výsledok. Zo 182 pacientov s predikovaným dobrým výsledkom malo 74\% dobrý výsledok. Vytvorený strom je znázornený na Obrázku \ref{fig:ClinicalTree}. Obrázok môže slúžiť ako učebnicová ukážka stromu vytvoreného algoritmom CART.