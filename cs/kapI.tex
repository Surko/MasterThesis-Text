\chapter{Popis priloženého média}\label{kapI}
{\LARGE\verb|/|}

\begin{itemize}
\item \verb|praca.pdf| Text práce v elektronickej podobe
\end{itemize}

{\LARGE\verb|/GenDTLib/|}

Adresár s celou aplikáciou a skriptom na jej rýchle spustenie. Taktiež obsahuje adresáre pre logy, zásuvné moduly, lokalizačné súbory a skripty na generovanie súborov \verb|GenericPropertiesCreator.props| a \verb|GenericObjectEd|- \verb|itor.props|.
\begin{itemize}
\item \verb|GenDTLib.bat| Spustiteľný skript pre operačný systém Windows
\item \verb|GenDTLib.sh| Spustiteľný shell skript pre operačné systémy UNIXového typu
\end{itemize}

{\LARGE\verb|/GenDTLib/bin/|}

Distribúcia aplikácie \verb|GenDTLib| a nástroja \verb|Weka|. V adresári sa nachádza minimálna množina súborov, ktoré sú potrebné na spustenie nástroja Weka s genetickým klasifikátorom.
\begin{itemize}
\item \verb|GenDTLib.jar| Nová aplikácia vo formáte jar, ktorá slúži ako zásuvný modul do nástroja Weka
\item \verb|weka.jar| Distribúcia nástroja Weka
\item \verb|config.properties| Konfiguračný súbor pre nastavenie genetického algoritmu na tvorbu rozhodovacích stromov
\item \verb|GenericPropertiesCreator.props| Konfiguračný súbor vybraný z distribúcie programu Weka, ktorá sa používa na definovanie nových klasifikátorov a ďalších jej nástrojov
\item \verb|GenericObjectEditor.props| Vygenerovaný súbor z konfiguračného súboru \verb|GenericPropertiesCreator.propr|. Táto konfigurácia nahradzuje tú, ktorá sa nachádza vo vnútri súboru \verb|weka.jar|
\end{itemize}

{\LARGE\verb|/GenDTLib/plugins/|}

{
Adresár, v ktorom sa musia nachádzať zásuvné moduly nových komponent genetického algoritmu. Každá komponenta je vo svojom špecifickom adresári.
}

{\LARGE\verb|/GenDTLib/_generate/|}

{
\begin{itemize}
\item \verb|RefreshGPC.bat| Skript pre Windows, ktorý môžeme použiť na obnovenie konfiguračného súboru \verb|GenericPropertiesEditor.props| z distribúcie programu Weka
\item \verb|GenerateGPC.bat| Skript pre Windows, ktorý slúži na pregenerovanie \verb|Gene|- \verb|ricObjectEditor.props| z pozmeneného konfiguračného súboru \verb|GenericP|- \verb|ropertiesEditor.props|
\end{itemize}
}

{\LARGE\verb|/GenDTLib/_logs/|}

{
Adresár, v ktorom sa musia nachádzať zásuvné moduly nových komponent genetického algoritmu. Každá komponenta je vo svojom špecifickom adresári.
}

{\LARGE\verb|/src/|}

{
Zdrojový kód vytvorenej aplikácie. Adresár obsahuje taktiež kód JUnit testov.
}

{\LARGE\verb|/data/|} 

{
Adresár s voľne dostupnými dátovými súbormi (prebrané zo \cite{online-uci}), na ktorých sme vykonávali experimenty.
}

{\LARGE\verb|/tests/|}

{
Adresár so spúšťanými experimentami a ich výsledkami.
\begin{itemize}
\item \verb|*.exp| Experimenty, ktoré sme vykonali a spracovali v práci
\item \verb|results/| Výsledky spúšťaných experimentov
\end{itemize}
}

{\LARGE\verb|/javadoc/|}

Adresár, ktorý obsahuje programátorskú dokumentáciu aplikácie a skript na jej vygenerovanie. 
{
\begin{itemize}
\item \verb|GenJavaDoc.bat| Skript pre Windows, ktorým generujeme programátorskú dokumentáciu priamo z kódu.
\item \verb|options| Nastavenie pre program \verb|javadoc| (z adresáru Java), ktorý definuje, ako sa generuje programátorská dokumentácia.
\item \verb|doc/| Adresár s vygenerovanou dokumentáciou vo forme html stránky. Otvorením súboru \verb|index.html| ju zobrazíme.
\end{itemize}
}


