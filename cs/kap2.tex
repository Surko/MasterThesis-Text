\chapter{Genetické algoritmy}
Hlavným cieľom práce je optimalizovanie tvorby rozhodovacích stromov pomocou genetických algoritmov (GA). Preto 
v tejto kapitole predstavíme genetické algoritmy a podrobnejšie popíšeme jeho jednotlivé časti, ktoré budeme ďalej potrebovať v ďalšej kapitole.

Táto kapitola je delená do viacerých oddielov. V oddieli \ref{kap2:2.1:Info} popíšeme základné informácie o genetických algoritmoch. Genetické operátory používané v genetických algoritmoch zavedieme v oddieli \ref{kap2:2.2:Operators}. Rôzne druhy funkcií, používaných na vyhodnotenie jedincov, zadefinujeme v oddieli \ref{kap2:2.3:Fitnesses}. Ďalej v časti \ref{kap2:2.4:Selection} predvedieme najzákladnejšie metódy na výber jedincov. Nakoniec v časti \ref{kap2:2.5:Parameters} popíšeme problém výberu správnych parametrov v genetických algoritmoch.

\section{Základný popis genetických algoritmov}\label{kap2:2.1:Info}
Genetické algoritmy sú optimalizačné techniky a prehľadávacie heuristiky, ktoré napodobujú princíp prirodzeného výberu vychádzajúceho z darwinovskej teórie. V tomto oddieli ponúkneme úvod ku genetickým algoritmom s popisom, ako by mal taký genetický algoritmus vyzerať. Informácie o genetických algoritmoch sme čerpali z knihy \cite{kap2-evolution} a úvod do oblasti genetiky zase z voľne dostupnej učebnice biológie \cite{online-biology}. Ďalšie doplňujúce poznatky boli získané z online zdrojov \cite{wiki-evolution,wiki-genetics}.

V časti

\subsection{Genetika z pohľadu Biológie}\label{kap2:2.1:2.1.1:Genetics}

\section{Genetické operátory}\label{kap2:2.2:Operators}

\section{Vyhodnotenie jedincov}\label{kap2:2.3:Fitnesses}

\section{Výber jedincov}\label{kap2:2.4:Selection}

\section{Nastavenie správnych parametrov}\label{kap2:2.5:Parameters}

