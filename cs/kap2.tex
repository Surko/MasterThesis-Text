\chapter{Genetické algoritmy}
Hlavným cieľom práce je optimalizovanie tvorby rozhodovacích stromov pomocou genetických algoritmov (GA). Preto 
v tejto kapitole predstavíme genetické algoritmy a podrobnejšie popíšeme jeho jednotlivé časti, ktoré budeme ďalej potrebovať v ďalšej kapitole.

Táto kapitola je delená do viacerých oddielov. V oddieli \ref{kap2:2.1:Info} popíšeme základné informácie o genetických algoritmoch. Genetické operátory používané v genetických algoritmoch zavedieme v oddieli \ref{kap2:2.2:Operators}. Rôzne druhy funkcií, používaných na vyhodnotenie jedincov, zadefinujeme v oddieli \ref{kap2:2.3:Fitnesses}. Ďalej v časti \ref{kap2:2.4:Selection} predvedieme najzákladnejšie metódy na výber jedincov. Nakoniec v časti \ref{kap2:2.5:Parameters} popíšeme problém výberu správnych parametrov v genetických algoritmoch.

\section{Základný popis genetických algoritmov}\label{kap2:2.1:Info}
Genetické algoritmy sú optimalizačné techniky a prehľadávacie heuristiky, ktoré napodobujú princíp prirodzeného výberu vychádzajúceho z darwinovskej teórie. V tomto oddieli ponúkneme úvod ku genetickým algoritmom s popisom, ako by mal taký genetický algoritmus vyzerať. Informácie o genetických algoritmoch sme čerpali z knihy \cite{kap2-evolution} a úvod do oblasti genetiky zase z voľne dostupnej učebnice biológie \cite{online-biology}. Ďalšie doplňujúce poznatky boli získané z online zdrojov \cite{wiki-evolution,wiki-genetics}.

V časti \ref{kap2:2.1:2.1.1:Genetics} najprv z pohľadu biológie vysvetlíme základné pojmy z genetiky. Ďalej v časti \ref{kap2:2.2:2.1.2:}.

\subsection{Genetika z pohľadu Biológie}\label{kap2:2.1:2.1.1:Genetics}
Na popísanie a zadefinovanie ďalších súčastí genetických algoritmov je dobré pripomenúť niektoré základné pojmy z oblasti genetiky. 

Bunka je základnou stavebnou jednotkou každého organizmu. Každá bunka má uložený popis toho ako sa má správať pri typických bunečných procesoch. Tento popis sa nazýva genetická informácia a v organizmoch je uskladnená v DNA, čo je obrovská molekula, polymér zložený z veľkého množstva pospájaných nukleotidov (kyselina fosforečná s deoxyribózou a naviazanou dusíkatou bázou). Tieto nukleotidy a konkrétne ich báze sú zodpovedné zato, ako vyzerá genetický kód tejto genetickej informácie. Malý kúsok vybraný zo sekvencie DNA, ktorý kóduje nejakú funkčnosť bunky alebo proces v bunke, sa nazýva gén. Presnejšie vyjadrenie by bolo, že gén je zodpovedný za tvorbu proteínov, ktoré sú ovplyvňujú funkčnosť bunky. Gén si je teda možné predstaviť ako významnú črtu organizmu (napríklad farba očí). Rôzne nastavenia génu nazývame alely (modré, hnedé, zelené oči).

DNA je kvôli jej veľkosti ďalej komprimovaná do štruktúry nazývaná chromozóm. Chromozóm sa nachádza v jadre bunky. Organizmy môžu mať v jadre viacero rôznych chromozómov. Podľa typu organizmu môžu byť tieto chromozómy spárované (diploidné organizmy) alebo nespárované (haploidné organizmy). Genóm organizmu potom nazývame genetický materiál zo všetkých chromozómov dokopy. Určitá množina génov z genómu organizmu sa nazýva genotyp. Vo vývojovom štádiu jedinca ovplyvňuje genotyp výzor a ďalšie charakteristiky organizmu, nazývané aj fenotyp.

\begin{figure}[h]
\centering
\centerline{\mbox{\includegraphics[width=250pt]{../img/kap2/DNA.pdf}}}
\caption{Na obrázku je príklad bunky s jadrom, v ktorom sú viditeľné chromozómy. Jeden z chromozómov je v spodnej časti roztiahnutý do DNA, ktorej jednotlivé časti sú popísané. Značky A,G,T predstavujú nukleotidové báze adenín, guanín a cytozín.}\label{fig:DNA}
\end{figure}

Úlohou každého živého organizmu je rozšíriť svoju genetickú informáciu a tým pádom aj jeho gény. Sexuálna reprodukcia dvoch rodičovských organizmov prebieha v dvoch procesoch:
\begin{itemize}
\item Krížením (jednoduchých buniek alebo komplexných živočíchov), ktoré prebieha na úrovni chromozómov. Každá dvojica chromozómov si medzi sebou vymieňa gény a vytvára nové chromozómy.
\item Mutáciou vytvorených chromozómov, pri ktorej sú niektoré nukleotidy pozmenené kvôli chýbam v krížení.
\end{itemize}
Po reprodukcii dostávame nové chromozómy, ktoré tvoria genóm novo vytvoreného organizmu.

Z biologického hľadiska je nutné popísať kvalitu nových organizmov. Tú je možné vyhodnotiť pomocou pravdepodobnosti, že sa jedinec dožije ďalšieho párenia. Ďalším kritériom môže byť plodnosť potomka alebo aj schopnosť prežitia.

\section{Genetické operátory}\label{kap2:2.2:Operators}

\section{Vyhodnotenie jedincov}\label{kap2:2.3:Fitnesses}

\section{Výber jedincov}\label{kap2:2.4:Selection}

\section{Nastavenie správnych parametrov}\label{kap2:2.5:Parameters}

