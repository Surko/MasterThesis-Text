\chapter{Genetická tvorba rozhodovacích stromov}
V tejto kapitole popíšeme genetický algoritmus, ktorý sa dá použiť na tvorbu rozhodovacích stromov a ktorým je inšpirovaná implementácia v nástroji Weka. Popíšeme, ako musíme pozmeniť kódovanie jedincov tak, aby títo jedinci čo najlepšie popisovali rozhodovacie stromy. Zavedené operátory v predchádzajúcej kapitole rozšírime o definície nových operátorov. Ostatné súčasti genetického algoritmu iba pripomenieme a spojíme ich s pojmami z rozhodovacích stromov.

Najskôr v oddieli \ref{kap3:3.1:Intro} popíšeme úlohu tvorby stromov s genetickými algoritmami. Ďalej v oddieli \ref{kap3:3.2:Encoding} pripomenieme kódovanie jedincov pomocou stromov a ako sa dá toto kódovanie pozmeniť na reprezentovanie rozhodovacích stromov. V oddieli \ref{kap3:3.3:Fitness} uvedieme ako budú vyzerať fitness funkcie v takomto genetickom algoritme. Na konci v oddieli \ref{kap3:3.4:Operators} pripomenieme operátor kríženia stromov a zadefinujeme dve nové mutácie stromov.
\section{Základný popis}\label{kap3:3.1:Intro}
V tomto oddieli popíšeme genetický algoritmus, ktorý sa dá použiť na vytvorenie rozhodovacích stromov. Dve najznámejšie techniky tvorby rozhodovacích stromov pomocou genetických algoritmov sú
\begin{itemize}
\item vytváranie stromov úplne od počiatku,
\item optimalizovanie už vytvorených stromov a zlepšenie ich prediktívnych schopností.
\end{itemize}
V tejto práci sa skorej zameriame na optimalizovanie vytvorených stromov. Vytvorené stromy nebudú náhodné, ale do určitej miery kvalitné. Kvalitu stromu vynútime použitím nejakého vhodného indukčného algoritmu (spomínaných v časti \ref{kap1:2.6:DTEvaluation}).

Celým procesom optimalizácie stromov sa snažíme čo najviac vylepšiť generalizačnú chybu. Z časti \ref{kap1:2.6:2.6.1:Generalize} vieme že generalizačnú chybu môžeme aproximovať rozdelením dátovej množiny $D$ na trénovaciu $T$ a validačnú $V$ podmnožinu. Na najlepšiu aproximáciu generalizačnej chyby musíme použiť $n$-násobnú krížovú validáciu. Obidva prístupy pomáhajú pri ohodnotení kvality vytvorených stromov. Algoritmus by mal kvôli tomu poskytovať nejakú formu aproximácie generalizačnej chyby.

Počiatočné rozhodovacie stromy, ktoré budeme optimalizovať teda budú vybudované z trénovacej množiny $T$. Operátory slúžia na prehľadávanie priestoru riešení a vyhľadanie nových rozhodovacích stromov. Typy operátorov sú zvolené tak, aby čo najviac diverzifikovali populáciu. Na vyberanie jedincov je vhodné použiť elitizmus (kvôli udržaniu najlepšieho riešenia) a turnajovú selekciu hlavne kvôli jej spomínaným výhodám v časti \ref{kap2:2.4:2.4.2:Tournament}. Genetickým algoritmom vytvárame nové stromy, ktorých kvalitu určíme podľa zvolených kritérií. Výpočet fitness z jedného alebo viacerých kritérií je spomínaná v časti \ref{kap2:2.3:Fitnesses}. Validačná množina $V$ bude slúžiť na otestovanie modelu a na zistenie generalizačnej chyby.

Pri tomto genetickom algoritme zavedieme nový krok, ktorý sa nazýva selekcia prostredia. Tá má za úlohu vyberať jedincov do ďalšej generácie z doposiaľ vytvorených jedincov. Selekcia prostredia je vykonaná po aplikovaní operátorov a slúži na vyfiltrovanie úplne nevhodných jedincov. Typy selekcie prostredia sú rovnaké ako v prípade obyčajnej selekcie, no najvhodnejšie je znova zvoliť turnajovú selekciu.

V časti \ref{kap2:2.1:2.1.2:AboutGeneticAlgo}, konkrétne na obrázku \ref{fig:GAdiagram}, sme zaviedli diagram genetického algoritmu. Podľa tohoto diagramu sa riadi aj chovanie genetického algoritmu na optimalizovanie stromov.
V algoritme \ref{fig:DTGeneticAlgo} uvádzame pseudo kód tohoto genetického optimalizovania stromov a popisujeme jeho jednotlivé súčasti.
Pseudokód je veľmi podobný tomu spomenutému v časti \ref{kap2:2.1:2.1.3:SimpleGeneticAlgo} v algoritme \ref{fig:geneticAlgoritm}.

\begin{algorithm}
\floatname{algorithm}{Algoritmus}
\caption{Kroky genetického algoritmu, ktorý optimalizuje rozhodovacie stromy vytvorené indukčným algoritmom.}\label{fig:DTGeneticAlgo}
$A$ - aktuálna populácia rozhodovacích stromov \\
$O$ - potomkovia \\
$e$ - hodnota elitizmu, $e \in [0,1]$ \\
$s$ - zvolená metóda selekcie (turnaj, ruleta) \\
$sp$ - zvolená metóda selekcie prostredia (turnaj, ruleta).
$f_1,\ldots,f_p$ - $p$ kritérií na vyhodnotenie rozhodovacích stromov pre problém $P$\\
$k_{1},\ldots,k_{q}$ - pravdepodobnosti $q$ operátorov kríženia\\
$m_{1},\ldots,m_{r}$ - pravdepodobnosti $r$ operátorov mutácie\\
$max\_gen$ - maximálny počet generácií \\
$max\_pop$ - maximálny počet jedincov v populácií \\
\bigskip
\begin{algorithmic}[1]
\State \parbox[t]{375pt}{$A \gets $ populácia $max\_pop$ stromov, vytvorených nejakým indukčným algoritmom.}
\For{$i = 1$ \textbf{to} $max\_gen$} 
	\State \textsc{Krok()}
\EndFor
\\
\Procedure{Krok}{\mbox{}}
	\State $O \gets \{\}$
	\State \textsc{VyhodnoťPopuláciu()}
	\While{$\lvert O \rvert < max\_pop$}
		\State \parbox[t]{350pt}{$J_1,J_2 \gets $ \textsc{VyberJedincov()}}
		\State \parbox[t]{350pt}{$K_1,K_2 \gets $ \textsc{Kríženie($J_1$,$J_2$)}}
		\State \parbox[t]{350pt}{$O_1 \gets $ \textsc{Mutácia($K_1$)}}
		\State \parbox[t]{350pt}{$O_2 \gets $ \textsc{Mutácia($K_2$)}}		
		\State $O \gets \{O_1,O_2\} \cup O$.
	\EndWhile
	\State $E \gets$ \textsc{Elitizmus($A$)}.
	\State $A \gets  E$ $\cup$ \textsc{VyberJedincovProstredia($O$)} 
\EndProcedure
\\
\Procedure{VyhodnoťPopuláciu}{\mbox{}}
\State \parbox[t]{350pt}{každý rozhodovací strom $x$ z aktuálnej populácie $A$ ohodnoť kritériami $f_i, 1 \leq i \leq p$.}
\EndProcedure
\\
\Function{VyberJedincov}{\mbox{}}
\State \Return \parbox[t]{300pt}{$O_1$,$O_2$ vybraných metódou selekcie $s$.}
\EndFunction
\\
\Function{Kríženie}{$J_1$,$J_2$}
\State $K_1, K_2 \gets J_1, J_2$
\For{$i = 1$ \textbf{to} $q$} 
\State $r \gets [0,1]$
\If{$r < k_i$}
	\State \parbox[t]{300pt}{kríženie $i$ aplikuj na stromy $K_1, K_2$.}
\EndIf
\State \Return $K_1$,$K_2$
\EndFor 
\EndFunction
\algstore{myalg}
\end{algorithmic}
\end{algorithm}

\begin{algorithm}
\ContinuedFloat
\caption{pokračovanie...}
\begin{algorithmic}[1]
\algrestore{myalg}
\Function{Mutácia}{$K$}
\State $M \gets K$
\For{$i = 1$ \textbf{to} $r$} 
\State $r \gets [0,1]$
\If{$r < m_i$}
	\State \parbox[t]{300pt}{mutáciu $i$ aplikuj na strom $M$.}
\EndIf
\State \Return $M$
\EndFor 
\EndFunction
\\
\Function{Elitizmus}{$A$}
\State \Return \parbox[t]{300pt}{použitie elitizmu o hodnote $e$, ktorý je zadefinovaný v časti \ref{kap2:2.4:2.4.3:Elitism}.}
\EndFunction
\\
\Function{VyberJedincovProstredia}{$O$}
\State \Return \parbox[t]{300pt}{rozhodovacie stromy vybrané z $O$ selekciou prostredia $sp$. Počet vybraných stromov bude rovný $(1-e)max\_pop$.}
\EndFunction
\end{algorithmic}
\end{algorithm}

\section{Kódovanie jedincov}\label{kap3:3.2:Encoding}
Jedinci v tomto genetickom algoritme sú rozhodovacie stromy. Z časti \ref{kap1:2.3:2.3.2:DT} vieme, že rozhodovací strom ma štruktúru stromu. Najrozumnejším kódovaním je preto kódovanie pomocou stromov, ktoré bolo spomínané v predchádzajúcej kapitole, v časti \ref{kap2:2.2:2.2.3:Tree}. Toto kódovanie je ale nutné upraviť tak, aby bralo do úvahy zbytok definície rozhodovacieho stromu spolu s kritériami delenia zadefinovanými v oddieli \ref{kap1:2.5:DTSplitCriterias}.

Vnútorný uzol $u$ rozhodovacieho stromu je kódovaný trojicou $(a,z,h)$, kde $a$ predstavuje atribút delenia, $z$ znamienko použité pri delení a $v$ hodnotu, v ktorej delíme dátovú množinu. Vieme, že typ atribútu môže byť numerický alebo kategoriálny. Podľa typu atribútu $a$ v trojici $(a,z,h)$ sú $z$ a $h$ definované takto:
\begin{itemize}
\item v prípade numerického atribútu $a$ je $z$  porovnávacie znamienko a $h$ reálna hodnota.
\item v prípade kategoriálneho atribútu $a$ je $z$ a $h$ nedefinované. Každá hrana vychádzajúca z vrcholu predstavuje jedno rozhodnutie.
\end{itemize}
Počet hrán vychádzajúcich z vrcholu $u$ sa riadi typom atribútu $a$.
\begin{itemize}
\item V prípade numerického atribútu sú hrany dve.
\item V prípade kategoriálneho atribútu je počet hrán rovný veľkosti definičného oboru atribútu $a$ (každá hrana predstavuje rozhodnutie).
\end{itemize}
Listový uzol $l$ rozhodovacieho stromu obsahuje iba hodnotu $c$, ktorá predstavuje klasifikáciu.
Takto kódovaný rozhodovací strom je na obrázku \ref{fig:DTEncode}.

\begin{figure}[h]
\centering
\centerline{\mbox{\includegraphics{../img/kap3/dtgen.pdf}}}
\caption{Príklad kódovania rozhodovacieho stromu v genetickom algoritme na tvorbu rozhodovacích stromov. Nech hodnoty výstupného atribútu sú z množiny $C=\{c_0,c_1\}$. Na obrázku sú tri vnútorné uzly s atribútami $x,y,z$. Atribúty $x$ a $y$ sú numerické a $z$ je kategoriálny. Hrany predstavujú rozhodnutia a počet hrán závisí na type atribútu uzlu. Listové vrcholy obsahujú hodnoty z množiny $C$ (klasifikáciu).}\label{fig:DTEncode}
\end{figure}

\section{Fitness funkcie}\label{kap3:3.3:Fitness}
Fitness funkcie v genetickom algoritme určujú, ako často budú jedinci vyberaní k reprodukcii. Tieto funkcie volíme podľa toho, čo chceme v rozhodovacom strome optimalizovať. Voľba vhodnej fitness funkcie v genetickom algoritme ovplyvňuje
\begin{itemize}
\item prediktívne vlastnosti stromu, kam zaraďujeme trénovaciu, testovaciu a generalizačnú chybu. 
\item zložitosť stromu, do ktorého zaraďujeme napríklad výšku stromu, počet listov, počet vrcholov.
\end{itemize} 

Na optimalizovanie prediktívnych vlastností stromu musíme voliť také fitness funkcie, ktoré s predikciou súvisia. Tieto fitness funkcie väčšinou počítajú určitú metriku z matice chybovosti. Na zlepšenie prediktívnych vlastností nie je každá metrika vhodná. Kvôli tomu je lepšie sa zamerať iba na niektoré z nich.
V časti \ref{kap1:2.6:2.6.2:Alternatives} sme definovali alternatívne metriky založené na matici chybovosti, ktoré môžeme použiť pri ohodnocovaní rozhodovacích stromov. Spomínané kríteriá citlivosti, precíznosti, špecificity a f-metriky sú vhodnými kandidátmi fitness funkcií. Na zlepšenie generalizačnej chyby by mal byť výpočet metrík prevedený na validačnej množine.

Z časti \ref{kap2:2.2:2.2.3:Tree} vieme, že stromy v genetických algoritmoch majú tendenciu narastať a že je ich nutné limitovať. To sa dá dosiahnuť voľbou fitness funkcií, ktoré ovplyvňujú zložitosť stromu. Tieto fitness funkcie využívajú metriky zadefinované v časti \ref{kap1:2.6:2.6.3:TreeLook}.

Voľba správnych fitness funkcií je obtiažnou úlohou. Optimálne by mali zvolené funkcie zmenšovať generalizačnú chybu a zbytočne nekomplikovať stromy. Tým sa úloha stáva multi-kriteriálny problém a výpočet kvality jedinca je vykonaná jednou z metód spomínaných v časti \ref{kap2:2.3:Fitnesses}.
\section{Operátory na tvorbu rozhodovacích stromov}\label{kap3:3.4:Operators}
V oddieli \ref{kap2:2.5:Operators} sme zadefinovali operátory a uviedli ich dôležitosť v genetických algoritmoch. V prípade genetickej tvorby rozhodovacích stromov sú operátory zodpovedné za vytváranie nových rozhodovacích stromov. Tieto operátory by mali pracovať v súlade s dátovými množinami. To znamená nevykonávať náhodné zmeny, ale iba také, ktoré nejako súvisia s dátami.

Z predchádzajúcich častí vieme, že v genetických algoritmoch najprv vykonávame kríženie a potom mutáciu. Taktiež vieme, že každý operátor aplikujeme iba za určitej pravdepodobnosti a že toto nastavenie je neľahkou úlohou. Jednou z možností je inšpirovať sa oddielom \ref{kap2:2.6:Parameters}.

V časti \ref{kap3:3.4:3.4.1:Crossover} pripomenieme kríženie podstromov. Potom v časti \ref{kap3:3.4:3.4.2:Mutation} predvedieme dva nové druhy mutácií.

\subsection{Kríženia}\label{kap3:3.4:3.4.1:Crossover}
V časti \ref{kap2:2.5:2.5.1:Crossover} sme definovali operátor kríženia podstromov. V prípade stromov je toto jediné kríženie, ktoré dáva zmysel a preto ho budeme používať aj pri genetickej tvorbe rozhodovacích stromov. 

\subsection{Mutácie}\label{kap3:3.4:3.4.2:Mutation}
V časti \ref{kap2:2.5:2.5.2:Mutation} sme uviedli rôzne druhy mutácií jedincov kódovaných stromami. Po menších úpravách sa dajú tieto mutácie použiť v genetickom algoritme na tvorbu rozhodovacích stromov. Mutácie je väčšinou nutné upraviť tak, aby brali do úvahy kódovanie uvedené v časti \ref{kap3:3.2:Encoding}. Jednotlivé mutácie by tiež mali vykonávať zmeny, ktoré sú v súlade s danou dátovou množinou $D$ (zabránenie náhodných zmien).

Jednou zo spomínaných mutácií, ktorú použijeme v genetickom algoritme, je zmena náhodného vnútorného uzlu na list. Nech $(a,z,h)$ je zakódovaný vnútorný uzol rozhodovacieho stromu $S$. Ďalej nech $D'$ je podmnožina množiny $D$ získaná delením $D$ podľa vrcholov (konkrétne atribútov a hodnôt vo vrchole) na ceste z koreňa do vrcholu $(a,z,h)$. Táto mutácia odstráni všetky odchádzajúce hrany z tohoto vrcholu a zmení $(a,z,h)$ na $c$. Hodnota $c$ je maximálna trieda výstupného atribútu v množine $D'$. Chovanie mutácie je na obrázku \ref{fig:mutnodetoleaf}.

\begin{figure}[h]
\centering
\centerline{\mbox{\includegraphics[width=400pt]{../img/kap3/mutnodetoleaf.pdf}}}
\caption{}\label{fig:mutnodetoleaf}
\end{figure}

Druhou spomínanou mutácia je priradenie nového podstromu do nejakého uzlu stromu. Podstromy budú generované z množiny $D$ a uložené v predom definovanej množine podstromov. Nové podstromy na priradenie by nemali byť náhodné a ani zbytočne veľké. Mutácia preto používa rozhodovacie korene (definícia \ref{kap1:2.3:2.3.2:stumpDT}). 

Nech $K$ je množina podstromov (rozhodovacích koreňov) a $K_i$ je $i$-tý podstrom. Ďalej nech $S$ je strom, na ktorom vykonávame mutáciu, $S_j$ je $j$-tý listový uzol a $\lvert S \rvert_{l}$ je počet listov stromu $S$. Potom jednotlivé kroky mutácie sú 
\begin{enumerate}
\item $r \gets [1,\ldots,\lvert K \rvert]$
\item $u \gets [1,\ldots,\lvert S \rvert_{v}]$
\item $S_u \gets K_r$
\end{enumerate}
Chovanie mutácie je na obrázku \ref{fig:mutdecisionstump}.

\begin{figure}
\centering
\centerline{\mbox{\includegraphics[width=400pt]{../img/kap3/mutdecisionstump.pdf}}}
\caption{}\label{fig:mutdecisionstump}
\end{figure}