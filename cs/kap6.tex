\chapter{Podobné práce}\label{kap6:SimilarWorks}
V tejto kapitole popíšeme dve práce, ktoré sa zaoberajú vytváraním rozhodovacích stromov pomocou genetických algoritmov. Ani jedna z týchto prác neposkytuje program, ktorý by sa dal otestovať a porovnať s mojim riešením.

\section{LEGAL-Tree: A lexicographic multi-objective genetic algorithm for decision tree induction}\label{kap6: 6.1:Legal}
Cieľom práce a jej autorov \cite{kap6-legal} bolo vytvoriť taký genetický algoritmus na tvorbu rozhodovacích stromov, ktorý by dokázal nahradiť existujúce indukčné algoritmy -- vytvorené modely by nekonvergovali k lokálnemu optimu, ako to je pri indukčných algoritmoch. 

Genetický algoritmus, ktorý v práci navrhli, vytvára jedincov od počiatku pomocou rozhodovacích koreňov (viď. \ref{kap1:2.3:2.3.2:stumpDT}). Mutácie a kríženia sú v algoritme rovnaké ako v našej implementácii, pričom na porovnávanie jedincov používajú tzv. lexikografické multi-kriteriálne funkcie, ktoré fungujú podobne ako prioritná fitness z časti \ref{kap2:2.3:2.3.2:Priority}.

Vytvorené modely testovali na známych dátových množinách a porovnávali s indukčným algoritmom C4.5. Pri testovaní dostali stromy, ktoré pri porovnaní s C4.5 neboli presnosťou signifikantne lepšie, ale tieto stromy boli v niektorých prípadoch jednoduchšie. Výsledky, ktoré v práci dostali sú v Tabuľke \ref{fig:legaltable}.

\begin{table}[h!]
% Because czech 
\catcode`\-=12
\centering 
\newcommand\T{\rule{0pt}{2.6ex}}       % Top strut
\newcommand\B{\rule[-1.2ex]{0pt}{0pt}} % Bottom strut
\begin{tabular}{|l||c|c||}
\hline
Dátová & \multirow{2}{*}{Presnosť} & Veľkosť \\ 
množina & & stromu \\
\hline
\multirow{2}{*}{colic} & 84.72\% & 8.62 \T\\[-1.5ex]
& \tiny (2.69\%) & \tiny (5.01\B\\
\hline
\multirow{2}{*}{credit-a} & 85.45\% & 4.74 \T\\[-1.5ex]
& \tiny (2.69\%) & \tiny (2.55)\B\\
\hline
\multirow{2}{*}{credit-g} & 71.86\% & 16.23 \T\\[-1.5ex]
& \tiny (2.57\%) & \tiny (10.73)\B\\
\hline
\multirow{2}{*}{hepatitis} & 81.13\% & 20.39 \T\\[-1.5ex]
& \tiny (4.29\%) & \tiny (6.58)\B\\
\hline
\end{tabular}
\caption{Presnosť a zložitosť modelu vytvoreného genetickým algoritmom navrhnutým autormi z \cite{kap6-legal}.}\label{fig:legaltable}
\end{table}

\section{A new genetic programming algorithm for building decision tree}\label{kap6:6.2:Hybrid}
Cieľom autorov \cite{kap6-group} tejto práce bolo vytvárať rozhodovacie stromy genetickými algoritmami, pričom sa snažili zabrániť tomu, aby boli pri reprodukcii použité také stromy, ktorých rozdiel veľkostí je príliš veľký. 

Algoritmus uvažuje populácie stromov, ktoré rozdeľuje do skupín, pričom v každej skupine sú iba stromy rovnakých veľkostí. Kríženia a mutácia sú vykonané iba na jedincoch v rámci jednej skupiny. V práci uvádzajú, že takýto prístup dokáže vhodnejšie prehľadávať priestor riešení a vývoj kvality nájdených riešení nie je až tak skokový, ako v prípade obyčajných genetických algoritmov.

Vytvorené modely znova testovali na známych dátových množinách, porovnávali s indukčným algoritmom C4.5 a obyčajným genetickým algoritmom na vytváranie stromov. Autori uvádzajú presnosti modelov, ktoré sú vo veľa prípadoch signifikantne lepšie. Rýchlosť implementácie je v porovnaní s normálnym genetickým algoritmom menšia a stabilnejšia (pomalšie stúpa so zvyšujúcou sa veľkosťou dát).

