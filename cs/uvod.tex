\chapter{Úvod}
\setlength{\parskip}{0.5em}
\setlength{\parindent}{0cm}
\addcontentsline{toc}{chapter}{Úvod}
Spracovanie dát je v poslednej dobe nutnou súčasťou množstva procesov (priemyselných, vedeckých, lekárskych, ...). So zvyšovaním množstva, tak ako aj zložitosti dát, je nutné zaručiť určitú mieru automatického spracovania. Hlavnou úlohou teda zostáva získať čo najrelevantnejšie informácie, pre tú danú sféru, pričom ľudský faktor by mal zohrávať minimum. Obor, ktorý rieši takéto úlohy sa nazýva Dobývanie znalostí\footnote{Dobývanie znalosti je len jednou zo súčastí o moc bohatšieho, väčšieho procesu nazývaného KDD}. LinkedIn ako aj Facebook sú jedným z príkladov kedy je táto technika využívaná na automatické odporúčania nových priateľov, ktorých ešte nemáme v kontaktoch.

Celý proces spracovania dát sa zlepšovaním výpočtovej techniky, ktorú si je ľahké a lacné zadovážiť s prípadnou paralelizáciou, radikálne zlepšil. Došlo k zrýchleniu tvorby modelov, čo v konečnom dôsledku zlacnilo celý tento postup. Zlepšenie modelu je ale taktiež dôležitou súčasťou. Vhodná optimalizácia pomocou genetických algoritmov, či iných heuristík môže významne zlepšiť počiatočný stav. Reťazenie, kombinovanie a využívanie známych techník ako bagging, boosting a ďalšie je taktiež dôležitou taktikou, ktorá zostáva prínosom pre túto sféru. Dostupné metódy sa ale neustále vylepšujú a to hlavne kvôli ustavičnému komplikovaniu dát.

Rozhodovacie stromy, ako aj ďalšie iné techniky (asociačné pravidlá, neurónové siete, ...) sú súčasťou dobývania znalosti. Rozhodovací strom patrí pod skupinu prediktívnych modelov, ktoré môžeme použiť na klasifikáciu\footnote{Klasifikačný strom}, ako aj na regresiu\footnote{Regresný strom}. V praxi je obľúbený hlavne kvôli jeho jednoduchosti a zrozumiteľnosti, ktorá vychádza z jeho priamočiarej stromovej štruktúry. Extrakcia pravidiel je v tomto prípade len otázka prechodu touto štruktúrou. Tieto pravidlá sú ľahko zrozumiteľné a je ich následne možné použiť na klasifikáciu. Väčšina techník z dobývania znalostí je založená na induktívnom učení. V takomto prípade je model skonštruovaný na trénovacích dátach, tak aby bol čo najviac generalizovaný aj na reálnych dátach. Takýto prístup ale vždy neposkytuje dostačujúce prediktívne schopnosti, ktoré by zaručili využiteľnosť tohoto modelu. Veľkým omedzením týchto techník je neexistujúca schopnosť využiť komplikovanejšie kritéria alebo nejakú ich kombináciu.

Keď neberieme do úvahy problém s komplikovanými kritériami, tak je na zlepšenie modelu možné využiť viacero prístupov. Jednoduchým zlepšením schopností modelu spočíva v poskytnutí zväčšenej trénovacej množiny. Najväčším problémom býva to, že sa pri úlohách stretávame skorej s menším množstvom dát. Bez použitia nejakých ďalších techník, ktoré dokážu vytvárať využiteľné modely aj na týchto omedzených množinách, je tento prístup nepoužiteľný. Tiež treba brať do úvahy aj to, že indukčný algoritmus nemusí vedieť efektívne pracovať s takouto zväčšenou množinou a môže dojsť k preučeniu. Potenciálnym riešením pri stromoch býva v takomto prípade orezanie.

Ďalšou možnosťou je využiť nejaký iný indukčný algoritmus, ktorý by vedel využiť všetky informácie v dátach čo najlepšie. Vytvorenie rýchleho a správneho algoritmu ale nie je tou najľahšou úlohou. Existujúce algoritmy sú v tejto dobe na vysokej úrovni a získavajú maximum z týchto dát bez toho, aby sme znížili generalizačné schopnosti.

Veľmi efektívnym riešením by bolo vyberanie len takých dát z trénovacej množiny, ktoré sú pre naše prediktívne schopnosti modelu dôležité. Jediným problémom je ale to, že predom nevieme určiť, ktoré dáta budú dôležité.

Na zlepšenie schopností modelu budeme v tejto práci využívať už spomínané genetické algoritmy. Najvýznamnejšou výhodou tohoto prístupu je popri robustnosti, pridanie schopnosti vytvárať také modely, ktoré dokážu využívať aj rôzne komp\-likované kritéria. Nevýhodou je zvyšovanie výpočtovej náročnosti (odzrkadlenie na rýchlosti), ktorá narastá priamo úmerne so zložitosťou kritérií.

Existuje mnoho článkov, ktoré sa zaoberajú vytváraním rozhodovacích stromov pomocou genetických algoritmov a ich výsledky sú pomerne zaujímavé. Voľne dostupných, implementovaných riešení je ale málo. Týmto sa teda tento prístup stáva obtiažnejším na využitie.
\section{Cieľ práce}
Táto práca má za úlohu vytvoriť algoritmus na budovanie rozhodovacích stromov využívajúci genetické algoritmy. Ťažiskom práce ale nebude iba v generovaní stromov kompletne len genetickými algoritmami, ale hlavne využiť už stávajúcich indukčných techník, ktoré nám dajú dobrý základ počiatočnej populácie. Následne budeme túto metaheuristiku využívať na dooptimalizovanie stromov a to aj na základe ľubovoľných kritérií alebo ich kombinácii. Takto sa snažíme čo najviac využiť informácie v dátach a zvýšiť prediktívne schopnosti. Zároveň sa snažíme tento algoritmus sprístupniť už z nejakého existujúceho voľne dostupného prostredia, ktoré je používané širokou verejnosťou a nezávislé na platforme. Vytvorený program by mal taktiež zaručovať určitú mieru rozšíriteľnosti. Týmto rozhodnutím bude riešenie dostupné aj pre bežných užívateľov, ktorí budú môcť vytvárať výzkumy a štúdia aj na štandardnom osobnom počítači. 

\section{Štruktúra práce}
TODO : Popísaná štruktúra

TODO : Dodatky 

Výsledok práce je implementácia algoritmu na konštrukciu stromov a vytvorená knižnica do nástroja Weka nazvaná GenLib.