\chapter{Úvod}
\setlength{\parskip}{0.5em}
\setlength{\parindent}{0cm}
Spracovanie dát je v poslednej dobe nutnou súčasťou množstva procesov (priemyselných, vedeckých, lekárskych a pod). So zvyšovaním množstva, tak ako aj zložitosti dát, je nutné zaručiť určitú mieru automatického spracovania. Hlavnou úlohou je získať čo najrelevantnejšie informácie pre danú oblasť, pričom ľudský faktor by mal zohrávať minimálnu rolu. Obor, ktorý rieši takéto úlohy sa nazýva Dobývanie znalostí\footnote{Dobývanie znalosti je len jednou zo súčastí oveľa bohatšieho, väčšieho procesu nazývaného proces dobývania znalostí z databází, anglicky knowledge discovery in databases (KDD)}. LinkedIn ako aj Facebook sú príkladmi, kde je táto technika využívaná na automatické odporúčania nových priateľov, ktorých ešte nemáme v kontaktoch.

Celý proces spracovania dát sa zlepšovaním výpočtovej techniky, ktorú si je ľahké a lacné zadovážiť i s prípadnou paralelizáciou, radikálne zlepšil. Došlo k zrýchleniu tvorby modelov, čo v konečnom dôsledku zlacnilo celý tento postup. Zlepšenie modelu je ale taktiež dôležitou súčasťou. Vhodná optimalizácia pomocou genetických algoritmov alebo iných heuristík môže významne zlepšiť kvalitu modelov. Ďalšie zlepšenie môžeme dostať reťazením modelov alebo ich kombinovaním (napríklad bagging a boosting), ktoré sú veľkým prínosom pre sféru dobývania znalostí a v súčastnosti sú tieto techniky používané širokou vrstvou ľudí. Dostupné metódy sa ale neustále vylepšujú, a to hlavne kvôli ustavične rastúcej zložitosti dát.

Rozhodovacie stromy, ako aj ďalšie iné techniky (asociačné pravidlá, neurónové siete a pod.) sú súčasťou dobývania znalosti. Rozhodovací strom patrí pod skupinu prediktívnych modelov, ktoré môžeme použiť na klasifikáciu\footnote{Klasifikačný strom}, ako aj na regresiu\footnote{Regresný strom}. V praxi je obľúbený hlavne kvôli jeho jednoduchosti a zrozumiteľnosti, ktorá vychádza z jeho priamočiarej stromovej štruktúry. Extrakcia pravidiel je v tomto prípade len otázka prechodu touto štruktúrou. Tieto pravidlá sú ľahko zrozumiteľné a je ich následne možné použiť na klasifikáciu. Väčšina techník z dobývania znalostí je založená na induktívnom učení. V takomto prípade je model skonštruovaný na trénovacích dátach tak, aby bol čo najviac generalizovaný aj pre reálne dáta. Takýto prístup ale neposkytuje vždy dostačujúce prediktívne schopnosti, ktoré by zaručili využiteľnosť tohoto modelu. Veľkým omedzením týchto techník je neexistujúca schopnosť využiť komplikovanejšie kritéria alebo nejakú ich kombináciu.

Keď neberieme do úvahy problém s komplikovanými kritériami, tak je na zlepšenie modelu možné použiť viacero prístupov. Jednoduché zlepšenie schopností modelu sa dá dosiahnuť poskytnutím zväčšenej trénovacej množiny. Najväčším problémom býva to, že sa pri úlohách stretávame skorej s menším množstvom dát. Bez použitia nejakých ďalších techník, ktoré dokážu vytvárať použiteľné modely aj na týchto omedzených množinách, je tento prístup nepoužiteľný. Do úvahy treba brať aj to, že indukčný algoritmus nemusí vedieť efektívne pracovať s takouto zväčšenou množinou a môže dôjsť k preučeniu. Potenciálnym riešením pri stromoch býva v takomto prípade jeho orezanie.

Ďalšou možnosťou je použiť nejaký iný indukčný algoritmus, ktorý by vedel spracovať všetky informácie v dátach čo najlepšie. Vytvorenie nového rýchleho a správneho algoritmu ale nie je tou najľahšou úlohou. Existujúce algoritmy sú v tejto dobe na vysokej úrovni a získavajú maximum z týchto dát bez toho, aby sme znížili generalizačné schopnosti.

Veľmi efektívnym riešením by bolo vyberanie len takých dát z trénovacej množiny, ktoré sú pre naše prediktívne schopnosti modelu dôležité. Hlavným problémom je však to, že dopredu nevieme určiť, ktoré dáta budú dôležité.

Na zlepšenie schopností modelu budeme v tejto práci využívať už spomínané genetické algoritmy. Najvýznamnejšou výhodou tohoto prístupu je popri robustnosti, pridanie schopnosti vytvárať také modely, ktoré dokážu využívať aj rôzne komp\-likované kritéria. Nevýhodou je zvýšená výpočtová zložitosť (najmä doba výpočtu) genetických algoritmov, ktorá narastá priamo úmerne so zložitosťou kritérií.

Existuje mnoho článkov, ktoré sa zaoberajú vytváraním rozhodovacích stromov pomocou genetických algoritmov. Ich výsledky sú aj pomerne zaujímavé. Voľne dostupných, implementovaných riešení je ale málo. Experimentovanie s algoritmami a ich otestovanie je teda nemožné. V takomto prípade nezostáva nič iné, len si takéto vytváranie rozhodovacích stromov implementovať sami.
\section{Cieľ práce}
Táto práca má za úlohu vytvoriť algoritmus na budovanie rozhodovacích stromov využívajúci genetické algoritmy. Ťažiskom práce nebude generovanie stromov kompletne len genetickými algoritmami, ale hlavne využitie dostupných indukčných techník, ktoré nám dajú dobrý základ pre počiatočnú populáciu. Následne budeme genetické algoritmy využívať na dooptimalizovanie stromov v počiatočnej populácii a to aj na základe ľubovoľných kritérií alebo ich kombinácii. Takto sa snažíme čo najviac využiť informácie v dátach a zvýšiť prediktívne schopnosti generovaných rozhodovacích stromov. Zároveň sa snažíme tento algoritmus sprístupniť už z nejakého existujúceho voľne dostupného prostredia, ktoré je používané širokou verejnosťou a je naviac nezávislé na platforme. Vytvorený program by mal taktiež zaručovať určitú mieru rozšíriteľnosti. Týmto rozhodnutím bude riešenie dostupné aj pre bežných užívateľov, ktorí budú môcť využívat vyvinuté postupy aj na bežnom osobnom počítači. 

\section{Štruktúra práce}
V kapitole \ref{kap1:DT} najprv uvedieme dobývanie znalosti a predstavíme rozhodovacie stromy. V kapitole \ref{kap2:GA} zavedieme metaheuristiku nazývanú genetické algoritmy. V kapitole \ref{kap3:DTGA} zase pospájame doposiaľ definované pojmy a zostavíme nový genetický algoritmus na vytváranie rozhodovacích stromov. Následne v kapitole \ref{kap4:Implementation} aplikáciu \verb|GenDTLib| i s podstatnými detailami implementácie nášho genetického algoritmu. V kapitole \ref{kap5:Tests} popíšeme, ako sme algoritmus testovali a aké výsledky táto technika produkovala. V tejto kapitole zároveň porovnáme náš genetický algoritmus na tvorbu rozhodovacích stromov so známym indukčným algoritmom C4.5. Ďalej v kapitole \ref{kap6:SimilarWorks} uvedieme popis existujúcich prác, ktoré vytvárajú rozhodovacie stromy genetickými algoritmami. Nakoniec v závere práce \ref{kap:fin} zhodnotíme dosiahnuté výsledky a popíšeme možné budúce rozšírenia práce.

Práca obsahuje viacero dodatkov. V dodatku \ref{kapI} je obsah priloženého média. Dodatok \ref{kapII} slúži ako užívateľská príručka aplikácie \verb|GenDTLib| s popisom parametrov genetického algoritmu, ktoré môžeme zmeniť v konfiguračnom súbore. Nakoniec v dodatku \ref{kapIII} popíšeme návod pre užívateľov, ktorí by chceli rozširovať aplikáciu o nové komponenty implementovaním daných rozhraní.