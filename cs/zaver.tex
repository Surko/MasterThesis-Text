\chapter{Záver}\label{kap:fin}
\renewcommand{\figurename}{Obrázok}
Všetky ciele práce sa podarilo bezozvyšku splniť. Genetický algoritmus na vytváranie rozhodovacích stromov, ktorý sme navrhli a naprogramovali v tejto práci dokáže skonštruovať rozhodovacie stromy. Kvalitou sú tieto stromy porovnateľné so stromami vytvorenými algoritmom C4.5 a zároveň sú od nich výrazne menšie.



Aplikácia \verb|GenDTLib| je spustiteľná a použiteľná z nástroja Weka. Implementácia je spoľahlivá a rýchlosť tvorby modelu je postačujúca, keď berieme do úvahy, že používame genetické algoritmy. Na implementovanie sme použili jazyk Java, ktorá poskytuje nezávislosť aplikácie od operačného systému. S aplikáciou sme navrhli aj aplikačné rozhranie, pri vytváraní ktorého sme sa snažili o to, aby bolo zrozumiteľné a v prípade ďalšieho rozšírenia sa sním pracovalo čo najľahšie. Zároveň sme museli zohľadniť to, že aplikácia má byť kompatibilná s nástrojom Weka. Parametre genetického algoritmu je možné nastaviť z konfiguračného súboru a existujúce komponenty je taktiež možné rozšíriť o nové pomocou zásuvných modulov.

Túto aplikáciu je možné spustiť na bežne dostupných počítačoch, na ktorých je spustiteľný aj nástroj Weka. 
Voľba vytvoriť aplikáciu ako zásuvný modul do nástroja Weka uľahčila prácu pri testovaní a umožňuje použiť klasifikátor z grafického rozhrania. Skonštruované stromy môžeme navyše vďaka tomu zobraziť zo vstavaného vizualizačného nástroja (obr. \ref{fig:wekatree}).

\begin{figure}[h]
\centering
\centerline{\mbox{\includegraphics[width=300pt]{../img/zaver/visualize.pdf}}}
\caption{Obrazovka z nástroja Weka, na ktorej je vizualizačný nástroj na zobrazovanie rozhodovacích stromov. Štruktúra stromov je podľa Definície \ref{kap1:2.3:2.3.2:DT}}\label{fig:wekatree}
\end{figure}

\subsection{Možné rozšírenia práce}
Aplikácia \verb|GenDTLib| je v súčasnej dobe plnohodnotný klasifikátor v nástroji Weka, ktorý vytvára rozhodovacie stromy pomocou genetických algoritmov. Možnosť nastaviť parametre genetického algoritmu v konfiguračnom súbore umožňuje užívateľovi meniť a experimentovať s existujúcimi komponentami a ich nastaveniami. Aplikáciu je ale možné rozšíriť aj o nové komponenty, ktoré sme v práci neuvažovali. Pri ďalšom vývoji alebo v budúcich verziách aplikácie by bolo vhodné pridať niektoré funkcie:

\begin{itemize}
\item \textbf{Nový kontajner pre populáciu}. V pokročilých genetických algoritmoch sa problém správnej voľby veľkosti populácie rieši dynamicky sa meniacou veľkosťou populácie (na začiatku väčšia a postupne sa zmenšuje). Aplikáciu je možné rozšíriť o nové druhy populácií a preto by bolo v budúcnosti vhodné takýto druh dynamickej populácie implementovať.
\item \textbf{Lepšia paralelizácia}. Aplikácia v aktuálnej verzii umožňuje nastaviť počet vlákien, ktoré počítajú fitness funkcie a generujú počiatočnú populáciu, no do budúcna by bolo možno vhodné paralelizovať aj niektoré komplikovanejšie operátory. Ďalšou možnosťou je implementovať niektorú z techník na paralelizovanie genetických algoritmov (napr. ostrovný model, hybridný paralelný genetický algoritmus).
\item \textbf{Vylepšiť zastavovacie kritéria}. Najpoužívanejším kritériom zastavenia je pri genetických algoritmoch počet generácií. V aktuálnej verzii je algoritmus zastavený po pevne danom počte generácií. Kritériá zastavenia môžu byť ale vhodnejšie. Pri ďalšom vývoji aplikácie by bolo vhodné vyskúšať a otestovať kvalitu riešení a rýchlosť genetických algoritmov s upravenými kritériami zastavenia. Takáto zmena by takisto mohla urýchliť celkový beh genetických algoritmov.
\item \textbf{Pareto fitness}. Pri niektorých úlohách riešených genetickými algoritmami sa používa pareto fitness. V aktuálnej verzii aplikácie je tento druh fitness nepodporovaný, a preto by bolo vhodné pareto fitness v budúcich verziách pridať a otestovať.
\end{itemize}