\chapter*{Záver}\label{kap:fin}
\addcontentsline{toc}{chapter}{Závěr}
Genetický algoritmus na vytváranie rozhodovacích stromov, ktorý sme navrhli a vytvorili v tejto práci dokáže skonštruovať rozhodovacie stromy. Kvalitou sú tieto stromy porovnateľné so stromami vytvorenými algoritmom C4.5 a zároveň sú od nich výrazne menšie.

Najväčším problémom pri testovaní bolo nastaviť parametre genetického algoritmu tak, aby sme dosiahli čo najlepšie prediktívne schopnosti. Taktiež sme sa snažili o vytvorenie stromov, ktoré by boli kvalitnejšie než tie vytvorené algoritmom C4.5. To sa nám ale nepodarilo. Napriek tomu sme nevytvorili rozhodovacie stromy, ktoré by boli signifikantne horšie. Zmenou nastavení je možné ďalej vylepšiť špecifickú vlastnosť stromu (napr. niektorá vlastnosť z matice chybovosti, hĺbka stromu, počet listov, počet vrcholov). Nastavenia parametrov je možné meniť pri každom spustení a teda ponechávame možnosť optimalizovať tieto parametre pre každú dátovú množinu zvlášť. Voľbou správnych nastavení sme sa v práci hlbšie nezaoberali, pretože veľkosť priestoru parametrov je príliš veľký a vyžadovala by ďalšie experimenty, ktoré by skúmali dopad rôznych nastavení na výsledné stromy.

Našu vytvorenú implementáciu sme porovnávali s C4.5  indukčným algoritmom na tvorbu stromov, ktorý je v nástroji Weka implementovaný ako klasifikátor J48. Genetický algoritmus s našimi zvolenými parametrami vytvoril pre každú dátovú množinu stromy, ktoré aj keď v niektorých prípadoch mali horšie generalizačné schopnosti, tak boli tieto rozdiely porovnateľné (neboli signifikantne horšie). Vytvorené stromy boli pri určitých nastaveniach (GA6,GA4), na všetkých dátových množinách okrem jednej, výrazne menšie od tých vytvorených algoritmom C4.5. Jediným problémom algoritmu je čas potrebný na vytvorenie modelu, ktorý je vo všetkých prípadoch výrazne horší. Vylepšenie modelu je teda na úkor času, ktorí sme ale v praxi väčšinou ochotní obetovať.

Aplikácia \verb|GenDTLib| je spustiteľná a použiteľná z nástroja Weka. Implementácia je spoľahlivá a rýchlosť tvorby modelu je zvládnuteľná, keď berieme do úvahy, že používame genetické algoritmy. Na implementovanie sme použili jazyk Java, ktorá poskytuje nezávislosť aplikácie od operačného systému. S aplikáciou sme navrhli aj aplikačné rozhranie, pri vytváraní ktorého sme sa snažili o to, aby bolo zrozumiteľné a v prípade ďalšieho rozšírenia sa sním pracovalo čo najľahšie. Zároveň sme museli brať do úvahy, že aplikácia má byť kompatibilná s nástrojom Weka. Parametre genetického algoritmu je možné nastaviť z konfiguračného súboru a existujúce komponenty je taktiež možné rozšíriť o nové pomocou zásuvných modulov.

Túto aplikáciu je možné spustiť na bežne dostupných počítačoch, na ktorých je spustiteľný aj nástroj Weka. 
Voľba vytvoriť aplikáciu ako zásuvný modul do nástroja Weka uľahčilo prácu pri testovaní a umožňuje použiť klasifikátor z grafického rozhrania. Skonštruované stromy môžeme navyše vďaka nej zobraziť zo vstavaného vizualizačného nástroja (obr. \ref{fig:wekatree}).

\subsection{Budúce práce}
Aplikácia \verb|GenDTLib| je v súčasnej dobe plnohodnotný klasifikátor v nástroji Weka, ktorý vytvára rozhodovacie stromy pomocou genetických algoritmov. Možnosť nastaviť parametre genetického algoritmu v konfiguračnom súbore umožňuje užívateľovi meniť a experimentovať s existujúcimi komponentami a ich nastaveniami. Aplikáciu je ale možné rozšíriť aj o nové komponenty, ktoré sme v práci neuvažovali. Pri ďalšom vývoji alebo v budúcich verziách aplikácie by bolo vhodné pridať niektoré funkcie:

\begin{itemize}
\item \textbf{Nový kontajner pre populáciu}. V genetických algoritmoch sa problém správnej voľby veľkosti populácie rieši dynamicky sa meniacou veľkosťou populácie (na začiatku väčšia a postupne sa zmenšuje). Aplikáciu je možné rozšíriť o nové druhy populácií a preto by bolo v budúcnosti vhodné takýto druh dynamickej populácie implementovať.
\item \textbf{Lepšia paralelizácia}. Aplikácia v aktuálnej verzii umožňuje nastaviť počet vlákien, ktoré počítajú fitness funkcie a generujú počiatočnú populáciu, no do budúcna by bolo možno vhodné paralelizovať aj niektoré komplikovanejšie operátory. Ďalšou možnosťou je implementovať niektorú z techník na paralelizovanie genetických algoritmov (napr. ostrovný model, hybridný PGA).
\item \textbf{Vylepšiť zastavovacie kritéria}. Najpoužívanejším kritériom zastavenia je pri genetických algoritmoch počet generácií. V aktuálnej verzii je algoritmus zastavený po pevne danom počte generácií. Kritériá zastavenia môžu byť ale vhodnejšie. Napríklad, tento počet môžeme meniť v každom kroku algoritmu. V budúcom vývoji by bolo vhodné vyskúšať a otestovať kvalitu riešení a rýchlosť genetických algoritmov s upravenými kritériami zastavenia. Takáto zmena by takisto mohla urýchliť celkový beh genetických algoritmov.
\item \textbf{Pareto fitness}. Pri niektorých úlohách riešených genetickými algoritmami sa používa pareto fitness. V aktuálnej verzii aplikácie je tento druh fitness nepodporovaný a preto by ho bolo vhodné v budúcich verziách pridať a otestovať.
\end{itemize}