%%% Hlavní soubor. Zde se definují základní parametry a odkazuje se na ostatní části. %%%

%% Verze pro jednostranný tisk:
% Okraje: levý 40mm, pravý 25mm, horní a dolní 25mm
% (ale pozor, LaTeX si sám přidává 1in)
\documentclass[12pt,a4paper]{report}
\setlength\textwidth{145mm}
\setlength\textheight{247mm}
\setlength\oddsidemargin{15mm}
\setlength\evensidemargin{15mm}
\setlength\topmargin{0mm}
\setlength\headsep{0mm}
\setlength\headheight{0mm}
% \openright zařídí, aby následující text začínal na pravé straně knihy
\let\openright=\clearpage

%% Pokud tiskneme oboustranně:
% \documentclass[12pt,a4paper,twoside,openright]{report}
% \setlength\textwidth{145mm}
% \setlength\textheight{247mm}
% \setlength\oddsidemargin{15mm}
% \setlength\evensidemargin{0mm}
% \setlength\topmargin{0mm}
% \setlength\headsep{0mm}
% \setlength\headheight{0mm}
% \let\openright=\cleardoublepage

%% Pokud používáte csLaTeX (doporučeno):
%\usepackage{czech}
%% Pokud nikoliv:
\usepackage[english,czech]{babel}
\usepackage[T1]{fontenc}

%% Použité kódování znaků: obvykle latin2, cp1250 nebo utf8:
\usepackage[utf8]{inputenc}

%% Ostatní balíčky
\usepackage{graphicx}
\usepackage[bottom]{footmisc}
\usepackage{amsthm,amsmath,amssymb}

%% Balíček hyperref, kterým jdou vyrábět klikací odkazy v PDF,
%% ale hlavně ho používáme k uložení metadat do PDF (včetně obsahu).
%% POZOR, nezapomeňte vyplnit jméno práce a autora.
\usepackage[hypertexnames=false,unicode]{hyperref}   % Musí být za všemi ostatními balíčky
\hypersetup{pdftitle=Využitie genetických algoritmov pri tvorbe rozhodovacích stromov}
\hypersetup{pdfauthor=Lukáš Šurín}
\usepackage[all]{hypcap}

%%% Drobné úpravy stylu

% Tato makra přesvědčují mírně ošklivým trikem LaTeX, aby hlavičky kapitol
% sázel příčetněji a nevynechával nad nimi spoustu místa. Směle ignorujte.
\makeatletter
\def\@makechapterhead#1{
  {\parindent \z@ \raggedright \normalfont
   \Huge\bfseries \thechapter. #1
   \par\nobreak
   \vskip 20\p@
}}
\def\@makeschapterhead#1{
  {\parindent \z@ \raggedright \normalfont
   \Huge\bfseries #1
   \par\nobreak
   \vskip 20\p@
}}
\makeatother

% Toto makro definuje kapitolu, která není očíslovaná, ale je uvedena v obsahu.
\def\chapwithtoc#1{
\chapter*{#1}
\addcontentsline{toc}{chapter}{#1}
}

\newtheorem{def-sk}{Definícia}
\newtheorem*{remark-sk}{Pozorovanie}
\theoremstyle{definition}
\newtheorem{observation}{\textbf{Poznámka}}

\usepackage{caption}
\usepackage{listings} 
\usepackage{subcaption}
\usepackage{algorithm}
\usepackage{multirow}
\usepackage[noend]{algpseudocode}
\usepackage{tabulary}
\renewcommand{\algorithmicforall}{\textbf{for each}}
\newcommand{\var}[1]{\mathit{#1}}
\newcommand{\func}[1]{\mathrm{#1}}
\usepackage[
	backend=biber,
	style=iso-numeric,
	babel=other,
	sortlocale=cs_CZ,
	bibencoding=UTF8
	]{
	biblatex
	}
\bibliography{bib/wiki,bib/online,bib/kap1,bib/kap2,bib/kap6}
\begin{document}

% Trochu volnější nastavení dělení slov, než je default.
\lefthyphenmin=2
\righthyphenmin=2

%%% Titulní strana práce

\pagestyle{empty}
\begin{center}

\large

Univerzita Karlova v Praze

\medskip

Matematicko-fyzikální fakulta

\vfill

{\bf\Large DIPLOMOVÁ PRÁCE}

\vfill

\centerline{\mbox{\includegraphics[width=60mm]{../img/logo.pdf}}}

\vfill
\vspace{5mm}

{\LARGE Lukáš Šurín}

\vspace{15mm}

% Název práce přesně podle zadání
{\LARGE\bfseries Využitie genetických algoritmov pri tvorbe rozhodovacích stromov}

\vfill

% Název katedry nebo ústavu, kde byla práce oficiálně zadána
% (dle Organizační struktury MFF UK)
Katedra softwaru a výuky informatiky

\vfill

\begin{tabular}{rl}

Vedoucí diplomové práce: & RNDr. František Mráz, CSc. \\
\noalign{\vspace{2mm}}
Studijní program: & Informatika \\
\noalign{\vspace{2mm}}
Studijní obor: & ITI \\
\end{tabular}

\vfill

% Zde doplňte rok
Praha 2015

\end{center}

\newpage

%%% Následuje vevázaný list -- kopie podepsaného "Zadání diplomové práce".
%%% Toto zadání NENÍ součástí elektronické verze práce, nescanovat.

%%% Na tomto místě mohou být napsána případná poděkování (vedoucímu práce,
%%% konzultantovi, tomu, kdo zapůjčil software, literaturu apod.)

\openright

\noindent
Ďakujem RNDr. Františkovi Mrázovi, CSc. za odborné vedenie diplomovej práce, za čas, ktorý mi behom vypracovania tejto práce venoval, a za jeho cenné pripomienky a trpezlivosť. Taktiež ďakujem aj ľuďom, ktorí mi pomohli s korektúrou textu.

\newpage

%%% Strana s čestným prohlášením k diplomové práci

\vglue 0pt plus 1fill

\noindent
Prohlašuji, že jsem tuto diplomovou práci vypracoval(a) samostatně a výhradně
s~použitím citovaných pramenů, literatury a dalších odborných zdrojů.

\medskip\noindent
Beru na~vědomí, že se na moji práci vztahují práva a povinnosti vyplývající
ze zákona č. 121/2000 Sb., autorského zákona v~platném znění, zejména skutečnost,
že Univerzita Karlova v Praze má právo na~uzavření licenční smlouvy o~užití této
práce jako školního díla podle §60 odst. 1 autorského zákona.

\vspace{10mm}

\hbox{\hbox to 0.5\hsize{%
V ........ dne ............
\hss}\hbox to 0.5\hsize{%
Podpis autora
\hss}}

\vspace{20mm}
\newpage

%%% Povinná informační strana diplomové práce

\vbox to 0.5\vsize{
\setlength\parindent{0mm}
\setlength\parskip{5mm}

Názov práce:
Využitie genetických algoritmov pri tvorbe rozhodovacích stromov
% přesně dle zadání

Autor:
Lukáš Šurín

Katedra:
Katedra softwaru a výuky informatiky

Vedoucí diplomové práce:
RNDr. František Mráz, CSc.

Abstrakt:
Rozhodovacie stromy sú úznávanou a široko používanou technikou pri spracovaní a analyzovaní dát. Tieto stromy sú konštruované typickými a všeobecne známymi indukčnými technikami (napr. ID3, C4.5, C5.0, CART, CHAID, MARS). Kvalita takto získaných stromov nie je vždy dokonalá a často má rezervy. Indukcia kvalitných stromov, na ktoré kladieme zložité požiadavky, je ťažkou až nereálnou úlohou. V tejto práci sa budeme zaoberať práve takýmito rozhodovacími stromami, konkrétne ich vytváraním. Spomínanú rezervu sa snažíme využiť na zlepšenie stromov pomocou metaheuristiky, genetických algoritmov, ktorá sa využíva pri rôznych optimalizačných úlohách. Práca taktiež obsahuje implementáciu navrhnutého algoritmu vo forme zásuvného modulu do prostredia Weka. Neoddeliteľnou súčasťou práce je porovnanie novej metódy so
známymi algoritmami tvorby rozhodovacích stromov.

Klíčová slova:
rozhodovacie stromy, genetické algoritmy, indukcia stromov, Weka, strojové učenie 

\vss}\nobreak\vbox to 0.49\vsize{
\setlength\parindent{0mm}
\setlength\parskip{5mm}

Title:
Applying genetic algorithms for decision trees induction

Author:
Lukáš Šurín

Department:
Department of Software and Computer Science Education

Supervisor:
RNDr. František Mráz, CSc.

Abstract:
Decision trees are recognized and widely used technique for processing and analyzing data. These trees are designed with typical and generally known inductive techniques (such as ID3, C4.5, C5.0, CART, CHAID, MARS). Predictive power of created trees is not always perfect and they often provide a room for improvement. Induction of trees with difficult criterias is hard and sometime impossible. In this paper we will deal with decision trees, namely their creation. We use the mentioned room for improvement by metaheuristic, genetic algorithms, which is used in all types of optimalization. The work also includes an implementation of a new proposed algorithm in the form of plug-in into Weka environment. A comparison of the proposed method for induction of decision trees with known algorithms is an integral part of this thesis.

Keywords:
decision trees, genetic algorithms, tree induction, Weka, machine learning

\vss}

\newpage

%%% Strana s automaticky generovaným obsahem diplomové práce. U matematických
%%% prací je přípustné, aby seznam tabulek a zkratek, existují-li, byl umístěn
%%% na začátku práce, místo na jejím konci.

\openright
\pagestyle{plain}
\setcounter{page}{1}
\tableofcontents

%%% Jednotlivé kapitoly práce jsou pro přehlednost uloženy v samostatných souborech

\renewcommand{\figurename}{Obrázok}
\chapter{Úvod}
\setlength{\parskip}{0.5em}
\setlength{\parindent}{0cm}
Spracovanie dát je v poslednej dobe nutnou súčasťou množstva procesov (priemyselných, vedeckých, lekárskych a pod). So zvyšovaním množstva, tak ako aj zložitosti dát, je nutné zaručiť určitú mieru automatického spracovania. Hlavnou úlohou je získať čo najrelevantnejšie informácie pre danú oblasť, pričom ľudský faktor by mal zohrávať minimálnu rolu. Obor, ktorý rieši takéto úlohy sa nazýva Dobývanie znalostí\footnote{Dobývanie znalosti je len jednou zo súčastí oveľa bohatšieho, väčšieho procesu nazývaného proces dobývania znalostí z databází, anglicky knowledge discovery in databases (KDD)}. LinkedIn ako aj Facebook sú príkladmi, kde je táto technika využívaná na automatické odporúčania nových priateľov, ktorých ešte nemáme v kontaktoch.

Celý proces spracovania dát sa zlepšovaním výpočtovej techniky, ktorú si je ľahké a lacné zadovážiť i s prípadnou paralelizáciou, radikálne zlepšil. Došlo k zrýchleniu tvorby modelov, čo v konečnom dôsledku zlacnilo celý tento postup. Zlepšenie modelu je ale taktiež dôležitou súčasťou. Vhodná optimalizácia pomocou genetických algoritmov alebo iných heuristík môže významne zlepšiť kvalitu modelov. Ďalšie zlepšenie môžeme dostať reťazením modelov alebo ich kombinovaním (napríklad bagging a boosting), ktoré sú veľkým prínosom pre sféru dobývania znalostí a v súčastnosti sú tieto techniky používané širokou vrstvou ľudí. Dostupné metódy sa ale neustále vylepšujú, a to hlavne kvôli ustavične rastúcej zložitosti dát.

Rozhodovacie stromy, ako aj ďalšie iné techniky (asociačné pravidlá, neurónové siete a pod.) sú súčasťou dobývania znalosti. Rozhodovací strom patrí pod skupinu prediktívnych modelov, ktoré môžeme použiť na klasifikáciu\footnote{Klasifikačný strom}, ako aj na regresiu\footnote{Regresný strom}. V praxi je obľúbený hlavne kvôli jeho jednoduchosti a zrozumiteľnosti, ktorá vychádza z jeho priamočiarej stromovej štruktúry. Extrakcia pravidiel je v tomto prípade len otázka prechodu touto štruktúrou. Tieto pravidlá sú ľahko zrozumiteľné a je ich následne možné použiť na klasifikáciu. Väčšina techník z dobývania znalostí je založená na induktívnom učení. V takomto prípade je model skonštruovaný na trénovacích dátach tak, aby bol čo najviac generalizovaný aj pre reálne dáta. Takýto prístup ale neposkytuje vždy dostačujúce prediktívne schopnosti, ktoré by zaručili využiteľnosť tohoto modelu. Veľkým omedzením týchto techník je neexistujúca schopnosť využiť komplikovanejšie kritéria alebo nejakú ich kombináciu.

Keď neberieme do úvahy problém s komplikovanými kritériami, tak je na zlepšenie modelu možné použiť viacero prístupov. Jednoduché zlepšenie schopností modelu sa dá dosiahnuť poskytnutím zväčšenej trénovacej množiny. Najväčším problémom býva to, že sa pri úlohách stretávame skorej s menším množstvom dát. Bez použitia nejakých ďalších techník, ktoré dokážu vytvárať použiteľné modely aj na týchto omedzených množinách, je tento prístup nepoužiteľný. Do úvahy treba brať aj to, že indukčný algoritmus nemusí vedieť efektívne pracovať s takouto zväčšenou množinou a môže dôjsť k preučeniu. Potenciálnym riešením pri stromoch býva v takomto prípade jeho orezanie.

Ďalšou možnosťou je použitie iného indukčného algoritmu, ktorý by vedel spracovať všetky informácie v dátach čo najlepšie. Vytvorenie nového rýchleho a správneho algoritmu ale nie je tou najľahšou úlohou. Existujúce algoritmy sú v tejto dobe na vysokej úrovni a získavajú maximum z týchto dát bez toho, aby sme znížili generalizačné schopnosti.

Veľmi efektívnym riešením by bolo vyberanie len takých dát z trénovacej množiny, ktoré sú pre naše prediktívne schopnosti modelu dôležité. Hlavným problémom je však to, že dopredu nevieme určiť, ktoré dáta budú dôležité.

Na zlepšenie schopností modelu budeme v tejto práci využívať už spomínané genetické algoritmy. Najvýznamnejšou výhodou tohoto prístupu je popri robustnosti, pridanie schopnosti vytvárať také modely, ktoré dokážu využívať aj rôzne komp\-likované kritéria. Nevýhodou je zvýšená výpočtová zložitosť (najmä doba výpočtu) genetických algoritmov, ktorá narastá priamo úmerne so zložitosťou kritérií.

Existuje mnoho článkov, ktoré sa zaoberajú vytváraním rozhodovacích stromov pomocou genetických algoritmov. Ich výsledky sú aj pomerne zaujímavé. Voľne dostupných, implementovaných riešení je ale málo. Experimentovanie s algoritmami a ich otestovanie je teda nemožné. V takomto prípade nezostáva nič iné, len si takéto vytváranie rozhodovacích stromov implementovať sami.
\section{Cieľ práce}
Táto práca má za úlohu vytvoriť algoritmus na budovanie rozhodovacích stromov využívajúci genetické algoritmy. Ťažiskom práce nebude generovanie stromov kompletne len genetickými algoritmami, ale hlavne využitie dostupných indukčných techník, ktoré nám dajú dobrý základ pre počiatočnú populáciu. Následne budeme genetické algoritmy využívať na dooptimalizovanie stromov v počiatočnej populácii a to aj na základe ľubovoľných kritérií alebo ich kombinácii. Takto sa snažíme čo najviac využiť informácie v dátach a zvýšiť prediktívne schopnosti generovaných rozhodovacích stromov. Zároveň sa snažíme tento algoritmus sprístupniť už z nejakého existujúceho voľne dostupného prostredia, ktoré je používané širokou verejnosťou a je naviac nezávislé na platforme. Vytvorený program by mal taktiež zaručovať určitú mieru rozšíriteľnosti. Týmto rozhodnutím bude riešenie dostupné aj pre bežných užívateľov, ktorí budú môcť využívat vyvinuté postupy aj na bežnom osobnom počítači. 

\section{Štruktúra práce}
V kapitole \ref{kap1:DT} najprv uvedieme dobývanie znalosti a predstavíme rozhodovacie stromy. V kapitole \ref{kap2:GA} zavedieme metaheuristiku nazývanú genetické algoritmy. V kapitole \ref{kap3:DTGA} zase pospájame doposiaľ definované pojmy a zostavíme nový genetický algoritmus na vytváranie rozhodovacích stromov. Následne v kapitole \ref{kap4:Implementation} aplikáciu \verb|GenDTLib| i s podstatnými detailami implementácie nášho genetického algoritmu. V kapitole \ref{kap5:Tests} popíšeme, ako sme algoritmus testovali a aké výsledky táto technika produkovala. V tejto kapitole zároveň porovnáme náš genetický algoritmus na tvorbu rozhodovacích stromov so známym indukčným algoritmom C4.5. Ďalej v kapitole \ref{kap6:SimilarWorks} uvedieme popis existujúcich prác, ktoré vytvárajú rozhodovacie stromy genetickými algoritmami. Nakoniec v závere práce \ref{kap:fin} zhodnotíme dosiahnuté výsledky a popíšeme možné budúce rozšírenia práce.

Práca obsahuje viacero dodatkov. V dodatku \ref{kapI} je obsah priloženého média. Dodatok \ref{kapII} slúži ako užívateľská príručka aplikácie \verb|GenDTLib| s popisom parametrov genetického algoritmu, ktoré môžeme zmeniť v konfiguračnom súbore. Nakoniec v dodatku \ref{kapIII} popíšeme návod pre užívateľov, ktorí by chceli rozširovať aplikáciu o nové komponenty implementovaním daných rozhraní.
\chapter{Úvod k rozhodovacím stromom}
V tejto kapitole zadefinujeme základné pojmy a koncepty, ktoré budú obsiahnuté v ďalších častiach tejto práce. V nadchádzajúcich podsekciách vysvetlíme postupne všetky dôležité pojmy. V oddieli \ref{kap1:2.1:Data} pomenovanom Dáta zadefinujemem čo sú to dáta, zavedieme pojem atribút a  vysvetlíme základné pojmy, s ktorými budeme pracovať. Dobývanie znalostí, ktoré predstavuje oddiel \ref{kap1:2.2:DataMining}, popíše tento obor so zaradením rozhodovacích stromov v ňom. V oddieli \ref{kap1:2.3:DT} zavedieme základný model rozhodovacích stromov, pre ktorý bude nutné uviesť pár pojmov z teórie grafov. Oddiel \ref{kap1:2.4:DTTypes} zoznamuje s rôznymi typmi rozhodovacích stromov. V oddieli \ref{kap1:2.5:DTTechniques} zhrnieme techniky nazývané indukčné algoritmy na tvorbu stromov.
Nakoniec v oddieli \ref{kap1:2.6:DTUsage} ukážeme využitie rozhodovacích stromov pri predikcii dát.

\section{Dáta}\label{kap1:2.1:Data}
V oddieli popíšeme čo sú to dáta. Dozvieme sa tie najzákladnejšie pojmy s ktorými sa stretneme pri práci s nimi. Informácie sme čerpali z voľne dostupnej knihy  \cite{kap1-DataMiningAndAnalysis} a online zdrojov \cite{wiki-Data,online-Data}.

V prvej časti \ref{kap1:2.1:2.1.1:DataRepresentation} popíšeme, čo sú to dáta a ako ich môžme reprezentovať. Popíšeme základné pojmy, ako je atribút a príznakový vektor. V časti \ref{kap1:2.1:2.1.2:DataAttributes} rozdelíme atribúty na dva druhy - kategoriálne a numerické. Aj napriek jednoduchosti týchto pojmov je ich nutné zaviesť, pretože ich budeme využívať v neskorších kapitolách.

\subsection{Dáta a ich reprezentácia}\label{kap1:2.1:2.1.1:DataRepresentation}
Dáta sú množiny hodnôt, ktoré predstavujú kusy informácií.
Môžu byť zozbierané, odmerané, analyzované a následne vizualizované.
Zozberané dáta môžme nájsť v relačných, tabuľkových databázach. Tento typ dát nazývame štrukturované. V skutočnosti sa však vo väčšine prípadov stretávame presne s ich opakom, a to s neštrukturovanými dátami.
Podľa \cite{kap1-DataMiningAndAnalysis} si dáta možme predstaviť ako maticu $n \times d$, kde $n$ predstavuje počet riadkov a $d$ počet stĺpcov tejto matice. Riadky sú v tomto prípade popisujú objekty a stĺpce sú tvorené hodnotami danej vlastnosti. Matica je daná takto

\begin{center}
Data = 
$\begin{array}{c | c c c c}
& X_{1} & X_{2} & \ldots & X_{d} \\ \hline
\mathbf{x}_{1} & x_{11} & x_{12} & \ldots & x_{1d} \\
\mathbf{x}_{2} & x_{21} & x_{22} & \ldots & x_{2d} \\
\vdots & \vdots & \vdots & \ddots & \vdots \\
\mathbf{x}_{n} & x_{n1} & x_{n2} & \ldots & x_{nd} \\
\end{array}$
\end{center}

$x_{i}$ predstavuje $i$-tý riadok, ktorý obsahuje $d$ hodnôt.
Jeden riadok matice taktiež nazývame príznakový vektor, entita, objekt, transakcia alebo inštancia.
\begin{center}
$\mathbf{x}_{i} = (x_{i1},x_{i2},\ldots,x_{id})$
\end{center}

$X_{j}$ predstavuje $j$-tý stĺpec, ktorý obsahuje $n$ hodnôt.
Na druhú stranu, stĺpec matice označujeme aj ako príznak, vlastnosť, atribút.
\begin{center}
$X_{j} = (x_{1j},x_{2j},\ldots,x_{nj})$
\end{center}

Pri metódach učenia s učiteľom uvažujeme ešte jeden stĺpec. Jeho hodnoty odpovedajú klasifikácii alebo regresii jednotlivých inštancií. Tieto hodnoty sú z domény množiny označovanej $C$. Kvôli konzistencii budeme tento posledný stĺpec ďalej nazývať ako výstupný atribút.
\begin{align}
C &= (c_{1},c_{2},\ldots,c_{m}) \nonumber
\end{align}

Premennú \textit{n} nazývame veľkosť dát a premennú \textit{d} zase dimenzionalita alebo rozmer dát.

Ako sme už spomínali, štrukturované dáta sú tie, ktoré máme uchované v riadkovo-stĺpcových databázach a teda ich ľahko prevedieme do maticového zápisu. Pre komplexnejšie dátové množiny (neštrukturované), ktoré sa vyskytujú napr. v bioinformatike (DNA, proteínové sekvencie,...), je buď nutné použiť iný zápis alebo využiť techniku extrakcie príznakov. Medzi ďaľšie neštrukturované dáta zaraďujeme obrázky, text, audio ale aj video nahrávky.

\subsection{Atribúty}\label{kap1:2.1:2.1.2:DataAttributes}
Doména atribútu je množina hodnôt, ktoré môže atribút nadobudnúť. Atribúty môžme rozdeliť na 2 typy podľa ich domény:
\begin{itemize}
\item \textbf{Kategoriálne} -- doména týchto atribútov sa skladá z konečnej množiny symbolov. Takýmto atribútom môže byť napríklad pohlavie (muž, žena), rodinný stav (slobodný, ženatý), ale aj komplikovanejšie ako je študijný program (informatika, fyzika, matematika, ...). Tieto atribúty je možné rozdeliť ešte do dvoch skupín:
\begin{itemize}
\item \textbf{Nominálne}, keď hodnoty v doméne nie sú usporiadané. Zmysluplné je iba porovnanie na zhodu. Príkladom môže byť pohlavie.
\item \textbf{Usporiadané}, keď sa hodnoty v doméne dajú porovnávať. Takéto hodnoty môžeme porovnávať nie len na zhodu, ale aj porovnávať (hodnota je väčšia, menšia). Ako príklad môžeme uviesť vzdelanie (základné, stredné, vysokoškolské).
\end{itemize}
\item \textbf{Numerické} sú také, ktorých doména je založená na celých alebo reálnych číslach. Jedným z takýchto atribútov môže byť príjem, zostatok na účte, \ldots. 
\end{itemize}

\section{Dobývanie znalostí}\label{kap1:2.2:DataMining}
V tomto oddieli zhrnieme vedomosti z dobývania znalostí, ktorého základ je dôležitý pri zaradení rozhodovacích stromov do správneho oboru.
Znalosti sme čerpali z kníh, ktoré sú vhodné ako úvod do tohoto oboru \cite{kap1-DataMiningAndAnalysis,kap1-DataMiningForMasses,kap1-DataMiningForTrees,kap1-StatisticLearn}. Pomimo týchto kníh sú informácie získané aj z ďaľších zdrojov ako je Wikipédia \cite{wiki-DataMining} ale aj menej známa HTML stránka od Dr. Saed Sayada \cite{online-DataMining}, ktorá stojí za povšimnutie. Všetky materiály sú voľne dostupné online.
 
V časti \ref{kap1:2.2:2.2.1:KDD} popíšeme proces dobývania znalostí z databází, anglicky knowledge discovery in databases (KDD). Hlavné informácie o samotnom dobývaní znalostí zhrnieme v časti \ref{kap1:2.2:2.2.2:DataMineProcess}. Tu ho rozdelíme do štyroch vrstiev. Každú vrstvu vysvetlíme a pri tom do tejto schémy zaradíme rozhodovacie stromy. V časti \ref{kap1:2.2:2.2.3:Taxonomy} predstavíme taxonómiu metód s nadväznosťou na zavedený 4-vrstvový model. Ďalší oddiel \ref{kap1:2.2:2.2.4:Supervised} vysvetlí, čo sú to metódy učenia s učiteľom, rozlíšime jeho dve obmeny a spomenieme tu aj učenie bez učiteľa.
Nakoniec v oddieli \ref{kap1:2.2:2.2.5:Tools} zhrnieme a uvedieme pár základných nástrojov, ktoré môžeme použiť pri práci v oboru dobývania znalostí. 

\subsection{Dobývanie znalostí z databází}\label{kap1:2.2:2.2.1:KDD}
Tiež známe pod skratkou KDD je komplikovaný proces pozostávajúci z viacerých fáz, ktorý ma za úlohu identifikovať nové, využiteľné vzory v dátach. KDD zahŕňa aj samotné dobývanie znalostí, ktoré je pravdepodobne jeho najvýznamnejšou súčasťou. Kvôli prominentnému postaveniu dobývania znalostí v rámci KDD, ich mnoho odborníkov stotožňuje.

\begin{figure}[h]
\centering
\centerline{\mbox{\includegraphics[width=400pt]{../img/kap1/DM-KDD.pdf}}}
\caption{Štrukturovaný KDD proces (prekreslené a preložené z \cite{kap1-DataMiningForTrees}).}\label{fig:dataMineKDD}
\end{figure}

Jednou z úloh v minulosti bolo formalizovať a štandardizovať prístup k dobývaniu znalostí. Na obrázku \ref{fig:dataMineKDD} sú jednotlivé časti sedem-stupňového modelu. Toto rozdelenie ale nie je pevné. Mnoho iných prác a spoločností navrhlo svoje obmeny. Veľké korporácie ako Daimler-Benz, poskytovateľ poistenia OHRA, vývojári softwaru a hardwaru NCR Corp. sa v roku 1999 spojili a vytvorili svoj vlastný model. Týmto vznikol známy CRISP-DM, ktorého štruktúru je možné vidieť na obrázku~\ref{fig:dataMineCRISP} \cite{kap1-DataMiningForMasses}. Ďalším známym modelom je model 5A alebo SEMMA.

\begin{figure}[h]
\centering
\centerline{\mbox{\includegraphics{../img/kap1/DM-CRISP.pdf}}}
\caption{CRISP-DM model (prekreslené a preložené z \cite{kap1-DataMiningForMasses})}\label{fig:dataMineCRISP}
\end{figure}

\subsection{Popis dobývania znalostí}\label{kap1:2.2:2.2.2:DataMineProcess}
Dobývanie znalostí je technika z KDD, ktorá je priamo zodpovedná za vytváranie poznatkov o existujúcich dátach, tak isto ako za predikciu určitých vlastnosti dát. Je to obor, ktorý kombinuje viaceré poznatky zo štatistiky, umelej inteligencie, strojového učenia a čiastočnej znalostí databázových strojov (obrázok \ref{fig:dataMineComb}).

% IMAGE 1
\begin{figure}[h]
\centering
\centerline{\mbox{\includegraphics[width=225pt]{../img/kap1/DM-comb.pdf}}}
\caption{Dobývanie znalostí ako kombinácia rôznych oborov (prekreslené a preložené z \cite{online-DataMining})}\label{fig:dataMineComb}
\end{figure}

Celý proces dobývania znalostí je možné popísať v štyroch vrstvách. Tento model je možné vidieť na obrázku \ref{fig:layerModel}. Každá z nižších vrstiev je dôležitá pre vrstvu vyššie.

% IMAGE 2
\begin{figure}[h]
\centering
\centerline{\mbox{\includegraphics[width=400pt]{../img/kap1/DM-layer.pdf}}}
\caption{Proces dobývania znalostí v 4 vrstvách (prekreslené a preložené z \cite[s. 26]{kap1-DataMiningForTrees})}\label{fig:layerModel}
\end{figure}

Prvá vrstva sa zaoberá už konečnou, výstupnou aplikáciou, pri tvorbe ktorej sme využili všetky predchádzajúce vrstvy. Tieto majú široký záber využitia v podnikateľských sférach. Medzi najvyužívanejšie patrí ohodnocovanie zákazníkov podľa ich príjmu, tak ako aj detekcia fraudov\footnote{podvod alebo nepoctivý trik, ktorým je poškodený zainteresovaný uživateľ}. Na vytvorenie aplikácie použijeme jednu alebo viacero techník, ktoré na našom obrázku predstavujú druhú vrstvu. Táto vrstva sa zaoberá úlohami strojového učenia ako je regresia, zhlukovanie a mnoho iných. Tieto úlohy využívajú rôzne, už konkrétne modely (tretia vrstva), kde patria už spomínané rozhodovacie stromy, neurónové siete, kohonenove mapy, \ldots. Posledná vrstva sa zaoberá tvorbou týchto modelov, v ktorej sa nachádzajú aj indukčné algoritmy tvorby rozhodovacích stromov.


\subsection{Taxonómia metód}\label{kap1:2.2:2.2.3:Taxonomy}
Z predchádzajúceho oddielu vieme, že časti dobývania znalostí sú umiestnené v nejakých vrstvách. V tomto oddieli nás budú zaujímať hlavne vzťahy v rámci druhej a tretej vrstvy. Vertikálne rozlíšenie je uvedené v predchádzajúcom oddieli. Na zistenie nejakých ďalších vzťahov je nutné zaradiť metódy do širších jednotiek ako na obrázku \ref{fig:dataMineParad}. Jeden taký pohľad na rozdelenie metód môžeme nájsť aj v \cite{online-DataMining}. V tomto druhom prípade je taxonómia v konečnom dôsledku funkcionálne rovnaká (niektoré názvy sú iné), ale líšia sa štruktúrou. 

V úlohách je v prvom rade nutné rozlišovať medzi dvoma typmi dobývacích techník, verifikačné a objaviteľské. Každý z nich má svoju metodológiu. Verifikačný sa zaoberá testovaním hypotéz, analýzou rozptylu a ďalšími praktikami. Objaviteľské zase automaticky hľadajú nové pravidlá a vzory.

Verifikačné metódy spoliehajú na určitú predom známu hypotézu, ktorá býva vyhodnotená expertom. Tento prístup nie je až tak spojený s dobývaním znalostí, ako to je pri objaviteľských metódach. Najvýznamnejším dôvodom je ten, že techniky dobývania znalostí by mali vyberať hypotézu z množiny vhodných hypotéz (identifikácia modelu), než aby sa spoliehali na nejakú dopredu známu (odhadnutie modelu).

Medzi takéto objaviteľské techniky patria ako predikčné, tak aj popisné metódy (zhlukovanie, vizualizácia). Popisnými sa hlavne snažíme pochopiť ako a prečo sú dáta umiestnené tak, ako sú nám prezentované. Predikčné zase dokážu predpovedať hodnoty jednotlivých atribútov daného objektu. Indukčné algoritmy sú typickou technikou, ktoré sa používajú pri vytváraní týchto modelov.

\begin{figure}[h]
\centering
\centerline{\mbox{\includegraphics[width=400pt]{../img/kap1/DM-paradigm.pdf}}}
\caption{Taxonómia modelov, prekreslené a preložené z \cite{kap1-DataMiningForTrees})}\label{fig:dataMineParad}
\end{figure}

\subsection{Učenie s učiteľom a bez učiteľa}\label{kap1:2.2:2.2.4:Supervised}
Učenie s učiteľom je oblasťou dobývacích techník, ktoré na vytvorenie modelu potrebuje trénovaciu množinu. Z taxonómie zavedenej v predchádzajúcom oddieli sem patria predikčné metódy. Pri tomto type sa teda snažíme nájsť vzťahy v trénovacích dátach medzi vstupnými atribútmi (nazývanými aj nezávislé premenné) a výstupným, klasifikačným atribútom (tiež nazývaný závislá premenná). Nájdený vzťah v dátach je reprezentovaný určitou štruktúrou a je to v konečnom dôsledku náš model. Takýto model je vytváraný už nejakou konkrétnou metódou. Medzi najzákladnejšie patria rozhodovacie stromy, neurónové siete, SVM, \ldots. Poskytnutím nových inštancií môžeme model využiť na klasifikáciu výstupného atribútu z tých vstupných.
Využitie je veľmi bohaté v rôznych sférach (financie, obchod, služby, \ldots). 

Hlavným cieľom učenia je zlepšenie modelu na nejakej úlohe pomocou predom daných dát. K tomuto cieľu je nutné poznať tri jeho komponenty:
\begin{itemize}
\item Úloha $U$, ktorú chceme učením vylepšiť
\item Dáta $D$, ktoré použijeme pri učení
\item Meradlo výkonu $M$, ktoré je použité pri meraní miery zlepšenia.
\end{itemize} 
Pre lepšie pochopenie môžeme použiť známy problém identifikácie emailov do spamu\footnote{Spamová správa - nevyžiadaná správa, ktorú užívateľovi prišla do emailovej schránky.}. Pre túto vyzerajú komponenty takto:
\begin{itemize}
\item Úlohou $U$ je identifikovať nevyžiadané emaily.
\item Dáta $D$ sú v tomto prípade množiny emailov, v ktorých sa nachádzajú správne aj nesprávne emaily.
\item Meradlom výkonu $M$ je percento nevyžiadaných emailov, ktoré boli klasifikované správne a percento správnych emailov, ktoré boli klasifikované nesprávne.
\end{itemize} 

Trénovacie dáta sú v tomto prípade v trochu odlišnejšom dátovom zápise, ako sme definovali v časti \ref{kap1:2.1:Data}. Nová matica by vyzerala ako $TrainData = (Data | Y)$, kde znak $|$ znamená pripojenie stĺpca $Y$ za maticu $Data$. Iným zápisom môže byť aj $TrainData = \{(\mathbf{x}_{k},y_{k})\}_{k=1}^{n}$, kde $y_{k}$ je hodnota klasifikácie/regresie pre vektor $\mathbf{x}_{k}$. Hodnota $y_{k}$ je z domény množiny $C$ 

Pri modeloch učenia s učiteľom je nutné rozlišovať medzi dvoma základnými typmi podľa typu výstupného atribútu:
\begin{itemize}
\item \textbf{Klasifikačné} mapujú vstupný vektor na dopredu známe triedy. Príkladom môže byť predikovanie rizika rakoviny (žiadne, nízke, vysoké) pacienta podľa jeho lekárskej anamnézy.
\item \textbf{Regresné} mapujú vstupný vektor na reálne čísla. Takýto model dokáže napríklad určiť výšku pôžičky, ktorú môže banka poskytnúť z informácii o žiadateľovi.
\end{itemize} 

Učenie bez učiteľa je presným opakom od učenia s učiteľom. Tento typ vytvára model bez trénovacej množiny. Podľa taxonómie sem môžme zase zaradiť popisné metódy (nie vizualizáciu) z objaviteľských techník. 

\subsection{Nástroje pre DM}\label{kap1:2.2:2.2.5:Tools}
Existuje veľké množstvo nástrojov, ktoré boli vytvorené aby zvládali úlohy z dobývania znalostí. Medzi nimi existuje plno profesionálnych nástrojov určených priamo pre takéto úlohy (Weka, RapidMiner). Iné druhy sú vo forme knižníc do stávajúcich jazykov (Python, Matlab, ...). Stavané sú tak, aby dokázali využiť techniky umelej inteligencie, strojového učenia, štatistiky a iné. Veľa z nich je ale platených, niektoré zase komplikované na inštaláciu, konfiguráciu alebo použitie. Pre učenie, ľahšiu využiteľnosť a hlavne pre účel našej práce sú preto nevhodné. 

\begin{figure}[h]
\centering
\centerline{\mbox{\includegraphics[width=300pt]{../img/kap1/DM-Weka.png}}}
\caption{Náhľad na užívateľské prostredie nástroja Weka}\label{fig:dataMineWeka}
\end{figure}

Najvhodnejšími kandidátmi sú:
\begin{itemize}
\item \textbf{RapidMiner} je napísaný v jazyku Java a poskytuje pokročilé analytické techniky pomocou framework-u založeného na šablónach. Výhodou tohto nástroja je aj to, že používateľ nie je nútený vytvárať žiaden kód. Ponú\-kaný je skorej ako služba, než kus lokálneho softwaru. Popri riešení úloh dobývania znalostí dokáže aj predspracovanie, vizualizáciu dát, štatistické modelovanie a vyhodnocovanie. RapidMiner je jedným z najpoužívanejších a najlepších nástrojov na úlohy z dobývania znalostí. Sila nástroja narastá vďaka využiteľnosti schém, modelov a algoritmov z Weky a R-skriptov. Distribuovaný je pod AGPL licenciou a je voľne stiahnuteľný zo stránky SourceForge.
\item \textbf{Weka} je kolekciou algoritmov pre strojové učenie a dobývanie znalostí. Algoritmy môžu byť volané priamo z užívateľského prostredia programu (obrázok \ref{fig:dataMineWeka}) alebo z vlastne vytvorenej aplikácie bežiacej pod Java prostredím. Obsahuje nástroje na predspracovanie, klasifikáciu, regresiu, asociačné pravidlá a vizualizáciu. Taktiež je použiteľná na vytváranie vlastných schém (prebrané z \cite{online-DataMiningWeka}). Aplikáciu si je možné stiahnuť zadarmo z hlavnej stránky pod GNU General Public licenciou. Tento typ licencie je jedna z najdôležitejších výhod Weky oproti RapidMiner-u. Vďaka nej je možná ľubovoľná úprava nástroja podľa našich požiadavkov, tak ako aj pridanie úplne nových algoritmov.
\end{itemize}

Ďalší kandidáti nie sú samostatnými aplikáciami, ale moduly pre dobývanie znalostí do nejakého dynamického jazyka. Tieto možnosti môžeme považovať za pokročilejšie, keďže na ich využitie je nutné spísať vlastný kód vo forme skriptov.
\begin{itemize}
\item \textbf{Matlab} je jazyk vysokej úrovne a interaktívne prostredie používané širokou skupinou vedcov a technikov \cite{online-DataMiningMatlab}. Príjemné prostredie tohoto jazyka a možnosť interaktívneho ladenia svojich programov zjednodušuje prácu. Rozširujúce moduly nazývané toolboxy\footnote{panel nástrojov, pomenovanie modulov do Matlab-u}, majú široké spektrum zamerania. Medzi najvýznamnejšie moduly na tvorbu modelov a dátovú analýzu pat\-ria štatistický modul (Statistics and Machine Learning Toolbox), modul pre neu\-rónové siete (Neural Network Toolbox) a mnoho ďalších. Tieto moduly sa dajú často ovládať z takzvaných pomocných nástrojov\footnote{pomocný nástroj/wizard je skupina po sebe nasledujúcich obrazoviek, ktoré prevedú užívateľa celým procesom tvorby požadovaného programu, bez ďalšej znalosti vnútornej funkcionality}.
Nevýhodou tohoto prístupu je buď nutnosť vlastniť Matlab, ktorý je spoplatnený, alebo mať alternatívny Octave, ktorým zase strácame príjemné užívateľské prostredie s ďalšími výhodami modulov, ako sú spomenuté pomocné nástroje.
\item \textbf{Python} je vysoko úrovňový dynamický jazyk podobne ako to je u Matlab-u \cite{wiki-Python}. Je to obľúbeným jazykom vedcov a jeho hlavnou filozofiou je zaručiť čitateľnosť kódu. Python podporuje mnoho programátorských paradigmát (objektovo orientovaný, imperatívny, funkcionálny, ...). Taktiež k nemu existuje mnoho knižníc na zjednodušenie práce v širokej škále oborov. Najoblúbenejší nástroj pre štatistické a strojové učenie je obsiahnutý v balíčku scikit-learn \cite{kap1-Scikit}. Tento balík v sebe viaže efektívne a jednoduché nástroje pre dátovú analýzu aj dobývanie znalostí. Stavia na obľúbených knižniciach NumPy, SciPy a matplotlib. Jazyk Python je k tomu voľne dostupný a scikit-learn je so svojou BSD licenciou skvelým kandidátom aj pre obchodné účely. 
\end{itemize}


\section{Základné pojmy rozhodovacích stromov}\label{kap1:2.3:DT}
V tomto oddieli vysvetlíme základné pojmy, ktoré sa používajú v spojitosti s rozhodovacími stromami. Najprv zadefinujeme čo je strom a čo znamenajú jednotlivé jeho časti v rozhodovacom strome. Informácií o tejto metóde existuje veľa, no základné pojmy sme hlavne čerpali z kníh \cite{kap1-DataMiningForTrees,kap1-DecisionTree} a \cite[s. 481-498]{kap1-DataMiningAndAnalysis}. Ďalšími použitými zdrojmi bola Wikipédia \cite{wiki-DecisionTree} a scikit-learn stránka \cite{online-DecisionTreeScikit}, na ktorej sa nachádzajú zaujímavé príklady.

V časti \ref{kap1:2.3:2.3.1} popíšeme čo je strom. Rozhodovacie stromy potom na základe grafovej štruktúry stromu zavedieme v časti \ref{kap1:2.3:2.3.2}.

\subsection{Strom}\label{kap1:2.3:2.3.1}
Na bližšie zadefinovanie rozhodovacieho stromu je potrebný aspoň minimálny základ z teórie grafov. Základné pojmy sme čerpali z knihy \cite{kap1-KapitolyDiskretka}.
\begin{def-sk}[Graf]\label{kap1:2.3:2.3.1:graf}
Graf je usporiadaná dvojica (V,E), kde V je nejaká neprázdna množina a E je množina dvojbodových podmnožín množiny V. Prvky množiny V sa nazývajú vrcholy grafu G a prvky množiny E sú hrany grafu G.
\end{def-sk}

\begin{figure}[h]
\centering
\centerline{\mbox{\includegraphics[width=300pt]{../img/kap1/DT-graphs.pdf}}} 
\caption{Príklady neorientovaných grafov. Čierne body predstavujú vrcholy a čiary medzi vrcholmi reprezentujú hrany z $E$.}\label{fig:decisionTreeGraphs}
\end{figure}

\begin{def-sk}[Orientovaný graf]\label{kap1:2.3:2.3.1:orient-graf}
Orientovaný graf G je dvojica (V,E), kde V je nejaká neprázdna množina a E je podmnožina kartézskeho súčinu $V \times V$. Prvky množiny V sa nazývajú vrcholy grafu G a prvky množiny E nazývame orientované hrany. Orientovaná hrana má tvar (x,y). Hovoríme, že orientovaná hrana vychádza z x a končí v y.
\end{def-sk}

Príklady obyčajných(neorientovaných) a orientovaných grafov je možné vidieť na obrázkoch \ref{fig:decisionTreeGraphs} a \ref{fig:decisionTreeOrientGraphs}.

\begin{figure}[h]
\centering
\centerline{\mbox{\includegraphics[width=300pt]{../img/kap1/DT-orientgraphs.pdf}}}
\caption{Príklady orientovaných grafov. Čierne body predstavujú vrcholy a čiary medzi vrcholmi reprezentujú orientované hrany z $E$. Šípka určuje orientáciu hrany.}\label{fig:decisionTreeOrientGraphs}
\end{figure}

\begin{def-sk}[Symetrizácia]\label{kap1:2.3:2.3.1:symetrizacia}
Každému orientovanému grafu $G = (V,E)$ môžeme priradiť neorientovaný graf $sym(G) = (V,E')$, kde $E' = \{\{x,y\}; (x,y) \in E$ alebo $(y,x) \in E\}$. Graf $sym(G)$ nazývame symetrizáciou grafu G.
\end{def-sk}

\begin{def-sk}[Súvislosť]\label{kap1:2.3:2.3.1:suvislost}
Hovoríme, že graf G je súvislý, keď pre každé dva jeho vrcholy x a y v ňom existuje cesta z x do y. Orientovaný graf je súvislý, keď je súvislá jeho symetrizácia (takáto súvislosť sa nazýva slabá).
\end{def-sk}

Pre orientované hrany existuje ešte druhý typ súvislosti a to silná. Avšak táto súvislosť je pre naše účely a hlavne definíciu rozhodovacích stromov nepodstatná.

\begin{def-sk}[Kružnica]\label{kap1:2.3:2.3.1:kruznica}
Graf $C_{n}$, kde n > 3, nazývame kružnicou, keď $ V = \{1,2,...,n\}$ a $E = \{\{i,i+1\};i=1,...,n-1\} \cup \{\{1,n\}\}$.
\end{def-sk}

Na obrázku \ref{fig:decisionTreeCycles} sú príklady štyroch druhov kružníc (tých najmenších).

\begin{figure}[h]
\centering
\centerline{\mbox{\includegraphics[width=300pt]{../img/kap1/DT-cycles.pdf}}}
\caption{Pár príkladov kružníc. Čierne body predstavujú vrcholy a čiary medzi vrcholmi reprezentujú hrany z $E$.}\label{fig:decisionTreeCycles}
\end{figure}

\begin{def-sk}[Strom]\label{kap1:2.3:2.3.1:strom}
Strom je súvislý graf neobsahujúci kružnicu. Orientovaný strom je súvislý orientovaný graf, ktorý po symetrizácii neobsahuje žiadnu kružnicu.
\end{def-sk}

Podrobnejší výklad o stromoch môže čitateľ nájsť v knihe \cite{kap1-KapitolyDiskretka}. 

\begin{def-sk}[Zakorenený strom]\label{kap1:2.3:2.3.1:korenovystrom}
Zakorenený strom je orientovaným stromom, v ktorom je jeden význačný vrchol nazývaný koreň stromu. Hrany takéhoto stromu sú jednoznačne orientované a vedú smerom od koreňa. Pre zakorenený strom ďalej definujeme tieto pojmy:
\begin{itemize}
\item \textbf{potomok} vrcholu $V$ je každý vrchol, do ktorého je vedená orientovaná hrana z vrcholu $V$,
\item \textbf{list} je vrchol, ktorý nemá ďalšieho potomka.
\end{itemize} 
\end{def-sk}

\begin{figure}[h]
\centering
\centerline{\mbox{\includegraphics[width=200pt]{../img/kap1/DT-parentchild.pdf}}}
\caption{Popis jednotlivých druhov vrcholov vzhľadom k danému vrcholu (červený). Modrá -- nasledovníci, žltá -- predchodcovia, zelená -- potomkovia, fialová -- rodič)}\label{fig:decisionTreeParentChild}
\end{figure}

\begin{def-sk}[Vzťahy v strome]\label{kap1:2.3:2.3.1:naslednik}
Nech $G = (V,E)$ je zakorenený strom (obrázok \ref{fig:decisionTreeParentChild}) a $k$ je koreň stromu, potom uvažujme:
\begin{itemize}
\item \textbf{predchodca} vrcholu $v$ je každý vrchol na orientovanej ceste z koreňa $k$ do vrcholu $v$,
\item \textbf{nasledovník} vrcholu $v$ je každý vrchol, kde jedným z predchodcov tohoto vrcholu je vrchol $v$,
\item \textbf{rodič} vrcholu $v$ je vrchol, z ktorého vedie orientovaná hrana do vrcholu $v$. Inak aj bezprostredný predchodca vrcholu $v$,
\item \textbf{potomok} vrcholu $v$ je vrchol, do ktorého vedie orientovaná hrana z vrcholu $v$. Inak aj bezprostredný nasledovník vrcholu $v$.
\end{itemize} 
\end{def-sk}

\subsection{Rozhodovací strom}\label{kap1:2.3:2.3.2}
Rozhodovací strom je prediktívny model, ktorý môžeme použiť pri riešení klasifikačných, ale aj regresných úloh. Z časti \ref{kap1:2.2:2.2.4:Supervised} vieme, že táto technika patrí pod metódy učenia s učiteľom. Z tejto časti taktiež vieme, že trénovacie dáta majú mierne pozmenený zápis od toho zadefinovaného v oddieli \ref{kap1:2.1:2.1.1:DataRepresentation}.

\begin{def-sk}[Trénovacie dáta]\label{kap1:2.3:2.3.2:traindata}
Trénovacie dáta sú n prvkovou množinou dvojíc $D = \{(\mathbf{x}_{k},y_{k})\}_{k=1}^{n}$, kde $\mathbf{x}_{k}$ je $d$-rozmerný príznakový vektor $d \geq 1$ a $y_{k}$ je jeho klasifikácia/regresia vybraná z množiny $C$. Pre klasifikáciu je doménou $C$ konečná množina symbolov, $C = \{c_{1},c_{2},\ldots,c_{m}\}$. Pri regresii je doména $C$ nejaká nekonečná množina, ako napríklad interval reálnych čísel $\mathbb{R}$, podmnožina celých čísel, $\ldots$.
\end{def-sk}

\begin{def-sk}[Podmnožina dát podľa hodnoty atribútu]\label{kap1:2.3:2.3.2:subsetValueData}
Nech $D = \{(\mathbf{x}_{i},y_{i})\}_{i=1}^{n}$ sú trénovacie dáta. Ďalej nech $D_{X_{i} = t}$ je podmnožina $D$ taká, že obsahuje práve všetky dvojice $(\mathbf{x},y) \in D$, pre ktoré platí $\mathbf{x} = (x_{1}, \ldots, x_{d})$ a $x_{i} = t$. Podobne $D_{c_{i}}$ je podmnožina dátovej množiny $D$ skladajúca sa z dvojíc $(\mathbf{x},y) \in D$, pre ktoré platí $y = c_{i}$.
\end{def-sk}

Na obrázku \ref{fig:decisionPlaneDataa} sú zobrazené štyri množiny dát, ktoré dokopy tvoria celú množinu $D$ ($=D_{1} \cup D_{2} \cup D_{3} \cup D_{4}$). Farby jednotlivých útvarov určuje ich klasifikáciu.
 
\begin{figure}[h]
\centering
\begin{subfigure}[b]{0.45\textwidth}
\includegraphics[width=\textwidth]{../img/kap1/DT-planedata.pdf}
\caption{}\label{fig:decisionPlaneDataa}
\end{subfigure}
\qquad
\begin{subfigure}[b]{0.45\textwidth}
\includegraphics[width=\textwidth]{../img/kap1/DT-planedatasep.pdf}
\caption{}\label{fig:decisionPlaneDatab}
\end{subfigure}
\caption{Umiestnenie dátových množín $D = D_{1} \cup D_{2} \cup D_{3} \cup D_{4}$ v rovine. Farby určujú klasifikáciu/regresiu pre množinu. V prípade b) sú vyznačené aj deliace nadroviny.}\label{fig:decisionPlaneData}
\end{figure}

\begin{def-sk}[Zobrazenie na atribúty]\label{kap1:2.3:2.3.2:mapovanievrcholov}
Nech P je príznakový vektor v daných dátach, $G = (V,E)$ je zakorenený strom a L je množina všetkých listov tohoto stromu. Potom zobrazením $\alpha:(V \setminus L) \times P \rightarrow V$ priraďujeme nelistovému vrcholu $u \in V \setminus L$ a špecifickému príznakovému vektoru $P$ vrchol $v \in V$ (listový alebo nelistový), taký že $(u,v)$ je orientovaná hrana z $E$.
\end{def-sk}

\begin{def-sk}[Zobrazenie na výstupný atribút]\label{kap1:2.3:2.3.2:mapovanielistov}
Nech C je výstupný atribút a dom(C) jeho obor hodnôt v daných dátach, $G$ je koreňový strom a L je množina všetkých listov tohoto stromu. Potom zobrazenie $\beta:L \rightarrow dom(C)$ priraďuje každému listu triedu alebo reálnu hodnotu.
\end{def-sk}

\begin{def-sk}[Rozhodovací strom]\label{kap1:2.3:2.3.2:DT}
Majme zakorenený strom $G = (V,E)$ a zobrazenia $\alpha$ a $\beta$, ako sú uvedené v Definíciach \ref{kap1:2.3:2.3.2:mapovanielistov} a \ref{kap1:2.3:2.3.2:mapovanievrcholov}. Rozhodovací strom pre trénovacie dáta $D$ definujeme ako trojicu $(G,\alpha,\beta)$.
\end{def-sk}

Pri takejto definícii klasifikácia/regresia funguje nasledovne. Daný príznakový vektor je testovaný na nelistovom vrchole pomocou funkcie $\alpha$, ktorou získame ďalší vrchol na otestovanie. Tento proces začína v koreni stromu a pokračuje až dokým nedostaneme nejaký listový vrchol. V tomto vrchole, za použitia funkcie $\beta$ dostávame požadovanú klasifikáciu/regresiu.
Testovanie väčšinou prebieha porovnávaním určitého atribútu alebo skupiny atribútov vektora s nejakou hodnotou.

\begin{def-sk}[Deliaca nadrovina]\label{kap1:2.3:2.3.2:axisHyperplanes}
Deliacu nadrovinu $h$, určenú vektorom $\mathbf{w}$ a číslom $b$ definujeme ako množinu všetkých bodov $\mathbf{x}$, pre ktoré platí $h:\mathbf{w^{T}x} + b = 0$, kde $\mathbf{w^{T}}$ nazývame normálny váhový vektor vzhľadom k nadrovine a $b$ je vzdialenosť od počiatku súradnicovej sústavy.
\end{def-sk}

V modeli rozhodovacích stromov sú uvažované oddeľujúce nadroviny, ktoré sú rovnobežné s jednou z hlavných osí. Z Definície \ref{kap1:2.3:2.3.2:axisHyperplanes} teda vyplýva, že váhový vektor $\mathbf{w}$ takejto nadroviny musí byť taktiež rovnobežný s niektorou z hlavných osí. K tomu je tento vektor obmedzený iba na jeden z množiny $\{\mathbf{e}_1,\mathbf{e}_2,\ldots,\mathbf{e}_d\}$, kde každý vektor $\mathbf{e}_i \in \mathbb{R}^{d}$ a obsahuje, okrem jedničky na $i$-tom mieste, samé nuly. Potom pre každý bod $\mathbf{x} = \{x_{1},x_{2},\ldots,x_{d}\}$ nadroviny $h$ platí:
\begin{align}
h:\mathbf{e}_i\mathbf{x} + b &= 0 \nonumber\\
h:x_{i} + b &= 0 \label{kap1:2.3:2.3.2:eq1}
\end{align}

\begin{def-sk}[Deliaci bod]\label{kap1:2.3:2.3.2:splitPoints}
Deliaci bod zodpovedajúci rozhodnutiu a definovaný nadrovinou rozdeľuje dátový priestor $\mathcal{R}$ na dva podpriestory $\mathcal{R}_{p}$ a $\mathcal{R}_{n}$. 
\end{def-sk}

\begin{remark-sk}\label{kap1:2.3:2.3.2:remarkSplitPoints}
Takže všetky body $\mathbf{x}$, pre ktoré platí $h(\mathbf{x}) < 0$, sú na jednej strane nadroviny, pričom $h(\mathbf{x}) > 0$ implikuje opačnú stranu. V prípade $h(\mathbf{x}) < 0$ podľa (\ref{kap1:2.3:2.3.2:eq1}) platí, že $x_{i} + b < 0$ a teda $x_{i} < -b$, pričom $x_{i}$ nadobúda nejakú hodnotu z domény atribútu $X_{i}$. Deliaci bod v rozhodovacom strome je preto potom uvádzaný vo formáte $X_{i} < v$, kde $v = -b$.
\end{remark-sk}

Rozhodovacie stromy teda obsahujú vrcholy, ktoré reprezentujú rozhodnutia zodpovedajúce deliacim bodom asociovanými s deliacimi nadrovinami.

\begin{def-sk}[Binárny rozhodovací strom]\label{kap1:2.3:2.3.2:binarnyDT}
Binárny rozhodovací strom je rozhodovací strom, pre ktorý platí, že počet potomkov každého vrcholu tohoto stromu je rovný 2 alebo 0 (len v prípade listov).
\end{def-sk}

Na obrázku \ref{fig:decisionPlaneDatab} sú nakreslené 4 množiny bodov zo štyroch tried spolu s deliacimi nadrovinami. Rozhodovací strom, v konečnom dôsledku, svojou vnútornou stavbou pokladá tieto deliace čiary.

\section{Typy rozhodovacích stromov}\label{kap1:2.4:DTTypes}
Z časti taxonómia metód \ref{kap1:2.2:2.2.3:Taxonomy} vieme, že rozhodovací strom je predikčný model.
Uvádzali sme, že predikčné modely sú dvoch typov, klasifikačné a regresné. Delenie rozhodovacích stromov bude v tomto ohľadu také isté. Informácie použité v tejto časti sme čerpali už zo spomínaných kníh \cite{kap1-DataMiningForTrees,kap1-DataMiningForMasses}, článku \cite{kap1-DecisionTreesTypes} a online zdroju \cite{online-DecisionTreeTypes}.

Čo je to klasifikačný strom spolu s jeho dôležitými vlastnosťami zhrnieme v časti \ref{kap1:2.4:2.4.1:DTClassification}. Poznatky o regresných stromoch budú uvedené v časti \ref{kap1:2.4:2.4.1:DTRegression}. V obidvoch častiach vypíšeme prevod rozhodovacieho stromu na súbor jednoduchých pravidiel pre ľahšiu predikciu.
\subsection{Klasifikačný strom}\label{kap1:2.4:2.4.1:DTClassification}
Rozhodovací strom, ako už vieme z predchádzajúcej kapitoly je hierarchická štruktúra, ktorý pomocou rozhodnutí dovedie zadaný vektor k určitej výstupnej hodnote. V obore dobývania znalostí je rozhodovací strom prediktívny model, ktorý môže reprezentovať ako klasifikačný, tak aj regresný model. Keď je strom použitý na klasifikačné úlohy, tak ho nazývame klasifikačný.

Klasifikačný strom používame vtedy, keď chceme klasifikovať objekt do triedy vybranej z konečnej, predom definovanej množiny tried podľa jeho príznakov. Klasifikačné stromy sú použiteľné najmä ako objaviteľská technika. Široko sú využívané v oblasti financií (pre ich zrozumiteľnosť), medicíny, vo vedeckých okruhoch, ale aj priemyselných procesoch. Aj napriek svojim výhodám nenahradzujú iné tradičné techniky predikcie ako sú Bayesovské alebo Neurónové siete. Pri klasifikačných stromoch si je taktiež nutné uvedomiť, že atribúty vo vrcholoch môžu byť ako numerické, tak aj kategorické.

Na Obrázku \ref{fig:TypesClassify} je nakreslený typický klasifikačný strom. Tento príklad vznikol priamym prepisom obrázku \ref{fig:decisionPlaneDatab} spolu s uvedenými deliacimi nadrovinami. Uvažujme, že každej definovanej množine $D_{1},\ldots,D_{4}$ z \ref{fig:decisionPlaneDatab} priradíme triedu $c_{1},\ldots,c_{4}$. Podľa pozorovania \ref{kap1:2.3:2.3.2:remarkSplitPoints} z predchádzajúcej časti je vidieť, že testy atribútov vo vnútorných uzloch stromu zodpovedajú deliacim bodom a konečné klasifikácie jednotlivým dátovým množinám na tomto obrázku. Vstupnú inštanciu, ktorú chceme klasifikovať označme $I$. Pravidlá tohoto klasifikačného stromu vyzerajú takto:
\begin{align}
Y(I) < 3.5 \wedge X(I) < 3.5 &\Rightarrow c_{2} \nonumber \\
Y(I) < 3.5 \wedge X(I) \geq 3.5 &\Rightarrow c_{3} \nonumber \\
Y(I) \geq 3.5 \wedge X(I) > 2 &\Rightarrow c_{4} \nonumber \\
Y(I) \geq 3.5 \wedge X(I) \leq 2 &\Rightarrow c_{1} \nonumber
\end{align}
\begin{figure}[h]
\centering
\centerline{\mbox{\includegraphics{../img/kap1/DT-binary.pdf}}}
\caption{Príklad klasifikačného stromu (binárneho), ktorý popisuje deliace nadroviny z obrázku \ref{fig:decisionPlaneDatab}}\label{fig:TypesClassify}
\end{figure}
Na tvorbu klasifikačných stromov sa využíva mnoho rôznych techník. 
Prvý publikovaný algoritmus sa nazýva THeta Automatic Interaction Detection (THAID). Neskôr vznikli vylepšené techniky ako Iterative Dichotomiser 3 (ID3), C4.5 (zlepšený ID3), CHi-squared Automatic Interaction Detection (CHAID) a Classification And Regression Tree (CART), ktorý sa dá použiť pri tvorbe klasifikačných aj regresných stromov. Každý z nich používa iný typ hľadania deliacich bodov. Medzi ďalšie, menej známe techniky môžeme uviesť CRUISE, GUIDE, QUEST, o ktorých si je možné prečítať v článkoch \cite{kap1-DecisionTreesUnused1,kap1-DecisionTreesUnused2,kap1-DecisionTreesUnused3}.
Konkrétnejšie budú niektoré z nich popísané v oddieli \ref{kap1:2.5:DTTechniques}. 

\subsection{Regresný strom}\label{kap1:2.4:2.4.1:DTRegression}
Regresný strom je variantou rozhodovacieho stromu navrhnutý na aproximáciu reálnych funkcií namiesto toho, aby bol použitý na klasifikáciu. Strom je veľmi podobný tomu klasifikačnému. Jediným rozdielom je výstupný atribút, ktorého doména je v tomto prípade nekonečná množina ($\mathbb{R},\mathbb{N},\ldots$).

\begin{figure}[h]
\centering
\centerline{\mbox{\includegraphics{../img/kap1/DT-regression.pdf}}}
\caption{Príklad regresneho stromu (binárneho), ktorý predikuje požičky pre žiadateľov}\label{fig:TypesRegression}
\end{figure}

Tento druh stromov teda používame vtedy, keď chceme predikovať vlastnosť objektu, ktorá je z nekonečnej množiny. Využitie je úplne rovnaké, ako pri klasifikačných stromoch. Regresné stromy majú ale bohatšie spôsoby pri predikcii. Týka sa to hlavne ďalšieho využitia iných druhov modelov v koncových listoch. Takto vytvorené stromy bývajú ale o niečo náročnejšie na konštrukciu kvôli ich komplikovanejšej štruktúre. Vďaka zložitejšej štruktúre stromy dávajú pri regresii zvyčajne lepšie výsledky než iné jednoduchšie metódy.

Na obrázku \ref{fig:TypesRegression} si je možné všimnúť typický regresný strom. V tomto prípade predikuje výšku priradenej pôžičky žiadateľom podľa ich veku a sociálneho zaradenia (príklad nevychádza z reálnych údajov). Označme žiadateľa/entitu o pôžičku ako $E$. Jednotlivé pravidlá pre tento strom by vyzerali takto:
\begin{align}
vek(E) < 25 \wedge soc.zar(E) = \check{s}tudent &\Rightarrow schv\acute{a}len\acute{a}\  po\check{z}i\check{c}ka = 30000 \nonumber \\
vek(E) < 25 \wedge soc.zar(E) \neq \check{s}tudent &\Rightarrow schv\acute{a}len\acute{a}\  po\check{z}i\check{c}ka = 15000 \nonumber \\
vek(E) \geq 25 \wedge soc.zar(E) = zamestnan\acute{y} &\Rightarrow schv\acute{a}len\acute{a}\  po\check{z}i\check{c}ka = 60000 \nonumber \\
vek(E) \geq 25 \wedge soc.zar(E) = nezamestnan\acute{y} &\Rightarrow schv\acute{a}len\acute{a}\ po\check{z}i\check{c}ka = 5000 \nonumber 
\end{align}
Regresné stromy sa konštruujú rôznymi technikami. Historicky prvým algoritmom na konštrukciu regresných stromov bol algoritmus Automatic Interaction Detection (AID), ktorý sa objavil pár rokov pred THAID. Ďalším, veľmi obľúbeným algoritmom je už spomínaný CART, ktorý dokáže vytvoriť popri regresných stromov aj tie klasifikačné. Spomedzi menej známych techník uvádzame M5' a GUIDE.

\section{Techniky tvorby stromov}\label{kap1:2.5:DTTechniques}
Cieľom tejto časti je predviesť najznámejšie techniky tvorby stromov (regresných aj klasifikačných). Tieto techniky spadajú pod oblasť učenia s učiteľom, ktorá bola bližšie popísaná v oddieli \ref{kap1:2.2:2.2.4:Supervised}. Jedná sa o algoritmy, ktoré vytvárajú strom z trénovacej množiny postupne od koreňa. Dáta sú v algoritme rozdelené podľa deliaceho kritéria na podmnožiny. Ďalej pokračujeme rekurzívnym volaním toho istého algoritmu na každú z týchto podmnožín. V obore dobývania znalostí sa celý tento proces nazýva indukcia rozhodovacích stromov zhora nadol, anglicky top down induction of decision trees (TDIDT). Indukčné algoritmy sú príkladom hladných techník. V praxi sú jedným z najpoužívanejších stratégii pri trénovaní rozhodovacích stromov. V tejto časti sme čerpali z kníh \cite{kap1-DataMiningForTrees,kap1-DataMiningAndAnalysis} pričom na rozšírenie ďalších poznatkov sme použili online zdroje \cite{online-SplitCriterias,online-SplitCriteriasMatter,online-DTLectures}.

Najskôr v časti \ref{kap1:2.5:2.5.1:SplitCriterias} zadefinujeme, čo sú to kritéria delenia a vypíšeme tie najznámejšie z nich. V časti \ref{kap1:2.5:2.5.2:Generic} ukážeme generický algoritmus na indukciu stromov. V ďalších častiach rozoberieme tri najznámejšie indukčné algoritmy na tvorbu stromov. Konkrétne v časti \ref{kap1:2.5:2.5.3:ID3} ukážeme algoritmus ID3. Následne v časti \ref{kap1:2.5:2.5.4:C4.5} predstavíme vylepšený algoritmus C4.5. Pre regresné typy stromov predvedieme v časti \ref{kap1:2.5:2.5.5:CART} obľúbený algoritmus CART.
\subsection{Kritéria delenia}\label{kap1:2.5:2.5.1:SplitCriterias}
Kritéria delenia sú funkcie, ktoré sú použité pri konštrukcii stromov. Tieto kritéria majú za úlohu zvoliť vhodný atribút do vnútorného vrcholu stromu.
V rozhodovacích stromoch majú tieto vnútorné vrcholy väčšinou jediný atribút, na ktorom porovnávame jednotlivé inštancie z dát. Kritéria, ktoré vytvárajú takýto typ atribútov nazývame jednorozmerné.

Jednorozmerné kritéria môžu byť ďalej charakterizované podľa
\begin{itemize}
\item pôvodu (teória informácii, závislosť, vzdialenosť),
\item štruktúry, kam patria binárne kritéria, kritéria založené na miere neusporiadanosti (impurity-based) a ich normalizované verzie.
\end{itemize}

Viac príkladov kritérií, doplnené informácie o spomínaných kritériách a podrobnejšie vysvetlenie deliacich kritérií je možné vyhľadať v knihách \cite{kap1-DataMiningForTrees,kap1-StatisticLearn}. 
\subsubsection{Neusporiadanosť}\label{kap1:2.5:2.5.1:Impurity}
Na bližšie zadefinovanie kritérií založených na neusporiadanosti potrebujeme vedieť pár pojmov z teórie informácii.  
\begin{def-sk}[Miera neusporiadanosti]\label{kap1:2.5:2.5.1:ImpurityFunction}
Nech $k \in \mathbb{N}$. Miera/funkcia neusporiadanosti $\gamma: [0,1]^{k}  \rightarrow \mathbb{R}$ je funkcia definovaná pre $n$-tice $P = (p_{1},\ldots,p_{k})$, pre ktoré platí
\begin{itemize}
\item $\sum_{i=1}^{k}p_{i} = 1, p_{i} \geq 0$,
\item $\gamma(P) \geq 0$,
\item $\gamma(P)$ dosahuje minimum v bodoch, pre ktoré platí $\exists i, p_{i} = 1$,
\item $\gamma(P)$ dosahuje maxima v bode $(1/k,\ldots,1/k)$,
\item $\gamma(P)$ je symetrická vzhľadom k hodnotám $p_{1},\ldots,p_{k}$,
\item $\gamma(P)$ je všade diferencovateľná.
\end{itemize}
\end{def-sk}

\begin{def-sk}[Neusporiadanosť množiny]\label{kap1:2.5:2.5.1:ImpuritySet}
Nech $D$ je dátová množina a $C = (c_{1},\ldots,c_{k})$ je množina klasifikačných tried. Ďalej nech $\mathbf{x}$ je príznakový vektor z $D$, $y$ jeho klasifikácia a $\gamma: [0,1]^{k}  \rightarrow \mathbb{R}$ je funkcia neusporiadanosti. Potom neusporiadanosť dátovej množiny $D$, označovaná $\gamma(D)$, je definovaná ako
\begin{equation}
\gamma(D) = \gamma
	\left(
	\dfrac
		{\lvert D_{c_{1}}\lvert}
		{\lvert D \lvert},		
	\ldots,
	\dfrac
		{\lvert D_{c_{k}}\lvert}
		{\lvert D \lvert}				
	\right).
\end{equation}
\end{def-sk}

\begin{def-sk}[Zníženie neusporiadanosti množiny]
Nech $X_{i}$ je daný atribút a $D_{1}, \ldots, D_{n}$ sú podmnožiny dátovej množiny $D$, ktoré vznikli rozdelením roviny podľa atribútu $X_{i}$. Potom výsledné zníženie neusporiadanosti označené ako $\Delta\gamma(X_{i},D) = \gamma(D) - \sum_{i=1}^{s}\dfrac{\lvert D_{i} \lvert}{\lvert D \lvert} \gamma(D_{i})$.
\end{def-sk}

\begin{figure}[h]
\centering
\centerline{\mbox{\includegraphics[width=400pt]{../img/kap1/DT-splitcrit.pdf}}}
\caption{Príklad, ako môžu kritéria delenia rozdeliť dátovú množinu so štyrmi triedami. Farby určujú pomery dát s danou triedou. Obrázok a) oproti b) je vhodnejší, pretože minimalizuje neusporiadanosť v listových vrcholoch týchto stromov.}\label{fig:SplitCriterias}
\end{figure}

Pri indukcii rozhodovacích stromov zvyčajne využívame nejakú hladnú stratégiu, ktorá sa snaží čo najviac znížiť neusporiadanosť celého rozhodovacieho stromu. Na obrázku \ref{fig:SplitCriterias} môžeme vidieť rozdelenie stromu v jednom z jeho vrcholov. V tomto obrázku preferujeme prípad a), pretože minimalizuje neusporiadanosť podmnožín $D$ v listových vrcholoch.
\subsubsection{Informačný zisk}\label{kap1:2.5:2.5.1:InfoGain} 
Informačný zisk je jedno z kritérií delenia založené na neusporiadanosti. Definícia kritéria vychádza z predchádzajúcej časti \ref{kap1:2.5:2.5.1:Impurity}, kde za funkciu neusporiadanosti $\gamma$ je dosadená entropia (z teórie informácii). Vzorec kritéria je potom definovaný ako ($D_{X_{i}}$ je v Definícii \ref{kap1:2.3:2.3.2:subsetValueData})
\begin{align}
Info\_zisk(& X_{i},D) = \nonumber \\
& Entropia(C,D) -
\sum_{t \in dom(X_{i})}^{}
\dfrac{\lvert D_{X_{i} = t}\lvert}{\lvert D \lvert} 
Entropia(C,D_{X_{i} = t}). \label{kap1:2.5:2.5.1:InfoGainDef} 
\end{align}
kde
\begin{equation}
Entropia(C,D) = \sum_{c_{i} \in C}^{} \left( -
\dfrac{\lvert D_{c_{i}}\lvert}{\lvert D \lvert} \log
\dfrac{\lvert D_{c_{i}}\lvert}{\lvert D \lvert}\right). 
\end{equation}

Informačný zisk blízko súvisí s metódou maximálnej vierohodnosti. Táto metóda je populárnou hlavne v oblasti štatistiky pri odhadovaní parametrov nejakého štatistického modelu. Odhad je prevádzaný pri aplikovaní na dátovú množinu a na daný štatistický model.

Ďalšie informácie o entropii si je možné nájsť v knihách spomenutých v týchto častiach. Pre naše potreby nám stačí použiť wikipédiu \cite{wiki-Entropy}.
\subsubsection{Gini Index}\label{kap1:2.5:2.5.1:GiniIndex}
Gini index je ďalšie z kritérii založené na neusporiadanosti. Používa sa ako miera diverzifikácie hodnôt vo výstupnom atribúte. Tak ako aj ostatné miery neusporiadanosti nadobúda hodnoty od 0 po 1, kde 0 znamená dokonalú rovnosť tried výstupu a 1 úplnú nerovnosť. Kritérium využívajúce Gini index ako funkciu neusporiadanosti $\gamma$ potom definujeme ako
\begin{align}
Gini\_zisk(X_{i},D) = Gini(C,D) -
\sum_{t \in dom(X_{i})}^{}
\dfrac{\lvert D_{X_{i} = t}\lvert}{\lvert D \lvert} 
Gini(C,D_{X_{i} = t}). \label{kap1:2.5:2.5.1:GiniIndexDef}
\end{align}
kde
\begin{equation}
Gini(C,D) = 1 - \sum_{c_{i} \in C}^{} 
\left(
\dfrac{\lvert D_{c{i}}\lvert}{\lvert D \lvert} 
\right) ^ 2. 
\end{equation}
Taktiež je možné nahliadnuť, že pre binárny prípad (počet tried je rovný dvom) je výpočet kritéria jednoduchší.
\begin{equation}
Gini(C,D) = 2p_{c_{1}}p_{c_{2}}, 
\end{equation}

kde $p_{c_{i}}$ je relatívny počet inštancií, ktorých výstupný atribút má hodnotu $c_{i}$.

\subsubsection{Koeficient zisku}\label{kap1:2.5:2.5.1:GainRatio} 
Koeficient zisku je ďalším známym kritériom, ktoré pre svoj výpočet využíva informačný zisk. Patrí medzi normalizované kritéria neusporiadanosti. Definované je ako
\begin{align}
Koef\_zisku(X_{i},D) =
\dfrac{Info\_zisk(X_{i},D)}{Entropia(X_{i},D)} \label{kap1:2.5:2.5.1:GainRatioDef} 
\end{align}

Z definície je vidieť, že kritérium nie je definované, keď je entropia rovná nule. Taktiež je kritérium náchylné k atribútom, pre ktoré je entropia veľmi malá. V minulosti bolo ukázané, že toto kritérium dokázalo prekonať normálny informačný zisk v presnosti aj v zložitosti.

\subsubsection{Twoing kritérium}\label{kap1:2.5:2.5.1:Twoing} 
Menej medzi známe kritéria patrí Twoing kritérium \cite[s.88]{kap1-DataMiningForTrees}. Je to binárne kritérium, ktoré je ako zbytok binárnych kritérii založené na rozdelení domény nejakého atribútu na dve podmnožiny. Zavedené bolo hlavne kvôli Gini indexu a jeho problému s príliš veľkou doménou výstupného atribútu. Kritérium je definované 
\begin{align}
Twoing(X_{i},& dom_{1}(X_{i}),dom_{2}(X_{i}),D) = \nonumber \\
& 0.25 
\dfrac{\lvert D_{X_{i} = t \in dom_{1}(X_{i})}\lvert}{\lvert D\lvert}
\dfrac{\lvert D_{X_{i} = t \in dom_{2}(X_{i})}\lvert}{\lvert D\lvert} \nonumber \\
& \left(
\sum_{c_{i} \in C}^{}
\left| 
\dfrac{
\lvert D_{X_{i} = t \in dom_{1}(X_{i})} \cap D_{c_{i}}\lvert }{\lvert D_{X_{i} = t \in dom_{1}(X_{i})}\lvert} -
\dfrac{
\lvert D_{X_{i} = t \in dom_{2}(X_{i})} \cap D_{c_{i}}\lvert }{\lvert D_{X_{i} = t \in dom_{2}(X_{i})}\lvert}
\right|
\right)^2 \label{kap1:2.5:2.5.1:TwoingDef} 
\end{align}

$dom_{1}(X_{i})$ a $dom_{2}(X_{i})$ sú podmnožiny domény atribútu $X_{i}$. Keď je výstupný atribút binárny, tak Gini Index a Twoing kritérium sú totožné. Toto kritérium v nebinárnom prípade preferuje atribúty s podobným rozdelením dátovej množiny. Algoritmus CART používa toto deliace kritérium pri indukcii stromov.

\subsection{Generický indukčný algoritmus}\label{kap1:2.5:2.5.2:Generic}
Indukčné algoritmy majú za úlohu vytvoriť rozhodovací strom z predom danej trénovacej množiny $D$. Hlavnou úlohou pri tvorbe je minimalizovať generalizačnú chybu. Nie vždy sa jedná len o minimalizovanie. Algoritmus sa môže snažiť o vytvorenie stromu s čo najnižšou výškou alebo tiež s obmedzeným počtom vrcholov.

Tvorba optimálneho rozhodovacieho stromu je ťažkou úlohou. Viac o tom v knihe \cite[s.51]{kap1-DataMiningForTrees}. Indukčné algoritmy sú dvoch typov. Tie, ktoré vytvárajú stromy zhora nadol a zdola nahor. Medzi algoritmy typu zhora nadol patria ID3, C4.5, CART a mnoho ďalších. Tvorba stromu je pri niektorých algoritmoch (C4.5 a CART) spojená spolu aj s jeho orezaním. Väčšinou na začiatku existuje jeden vrchol stromu (koreň), ktorému odpovedá celá trénovacia množina $D$. Následníci tohoto koreňa ďalej zodpovedajú podmnožinám trénovacej množiny. Týmto postupom stále delíme trénovaciu množinu, a tým vytvárame nové vrcholy stromu. Delenie prebieha až dokým nie je pravdivé niektoré zo stanovených kritérií zastavenia. Kostra takejto tvorby stromu je popísaná v Algoritme  \ref{fig:genericAlgoritm}.

\begin{algorithm} 
\floatname{algorithm}{Algoritmus}
\caption{Generický algoritmus na tvorbu stromov, z ktorého vychádzajú známe algoritmy ID3,C4.5,$\ldots$}\label{fig:genericAlgoritm}
$D$ - Trénovacia množina \\
$X$ - Množina atribútov \\
$Y$ - Výstupný atribút \\
$Del\_Krit$ - Kritérium pre delenie \\
$Stop\_Krit$ - Kritérium na zastavenie tvorby stromu \\
$T$ - Vytvorený strom 
\begin{algorithmic}
\Function{IndukciaStromu}{$D$,$X$,$Y$,$Del\_Krit$, $Stop\_Krit$}
\State $\var{T} \gets $ Strom s jedným vrcholom (koreň)
\If {$Stop\_Krit(D)$}:	  
\State $\var{T} \gets $ list s najčastejšou hodnotou atribútu $Y$ v $D$.
\Else
\State $\var{A} \gets min_{X_{i}}Del\_Krit(X_{i},D)$
\State $attr(\var{T}) \gets A$
\EndIf
\ForAll{$\var{v_{i}} \in A$}:
\State $PodStrom_{i}(T) \gets $ \textsc{IndukciaStromu}($D_{X_{i} = v_{i}}$,$X$,$Y$,$\_$,$\_$) 
\State $Hrana_{i}(T) \gets v_{i}$  
\EndFor \\
\Return{\textsc{OrezanieStromu}($D$,$T$,$Y$)}
\EndFunction
\\
\Function{OrezanieStromu}{$D$,$T$,$Y$}
\Repeat 
\State $t \gets$ vrchol stromu T, ktorý po orezaní maximalizuje predom zvolené vyhodnocovacie kritérium  
\If{$t \neq \emptyset$}:
\State $T \gets$ orezanie($T$,$t$)
\EndIf
\Until{$t = \emptyset$} \\
\Return{$T$}
\EndFunction
\end{algorithmic}
\end{algorithm}

%\begin{figure}[h]
%\lstset{inputencoding=utf8,extendedchars=true,basicstyle=\ttfamily\small,literate={á}{{\'a}}1 {é}{{\'e}}1 {ž}{{\v{z}}}1 {ň}{{\v{n}}}1 {š}{{\v{s}}}1 {č}{{\v{c}}}1 {ď}{{\v{d}}}1 {í}{{\'i}}1 {ý}{{\'y}}1 {ú}{{\'u}}1}
%\begin{lstlisting}[mathescape]
%IndukciaStromu($D$,$A$,$y$,$DeliaceKrit\acute{e}rium$,$Krit\acute{e}riumZastavenia$)
%$D$ - Trénovacia množina
%$A$ - Vstupné príznaky
%$y$ - Výstupný atribút
%$DeliaceKrit\acute{e}rium$ - Kritérium pre delenie
%$Krit\acute{e}riumZastavenia$ - Kritérium na zastavenie tvorby stromu
%
%Vytvor strom $T$ s jedným vrcholom (koreň)
%IF Krit\acute{e}riumZastavenia(D) THEN
%	Sprav z $T$ list s najčastejšou
%	hodnotou atribútu $y$ v $D$.
%ELSE
%	Vyber atribút $a \in A$ s najlepšou
%	hodnotou $DeliaceKrit\acute{e}rium(a,D)$.
%	Aktuálnemu vrcholu priraď atribút $a$.
%	FOR $v_{i} \in A$:
%		Nastav $PodStrom_{i}$ = IndukciaStromu($D_{v_{i}}$,$A$,$y$,_,_).
%		Spoj vrchol $t_{T}$ a $PodStrom_{i}$ s hranou označenou
%		$v_{i}$.
%	END FOR
%END IF
%RETURN OrezanieStromu(D,T,y).
%\end{lstlisting}
%\caption{Generický algoritmus z ktorého vychádzajú známe algoritmy ID3,C4.5,$\ldots$. Prevzaté a prepísané z \cite[s.52]{kap1-DataMiningForTrees}}\label{fig:genericAlgoritm}
%\end{figure}

Existujú rôzne kritéria zastavenia, ktoré prerušia vytváranie stromu. Tými najznámejšími sú

\begin{itemize}
\item Všetky inštancie v trénovacej množine majú jednotný výstupný atribút $y$.
\item Strom dosiahol maximálnu výšku.
\item Hodnota najlepšieho deliaceho kritéria je menšia než určitý prah.
\item Počet inštancií v danom uzle je menší než dopredu zadaná konštanta.
\end{itemize}

\subsection{ID3}\label{kap1:2.5:2.5.3:ID3}
ID3 algoritmus je najznámejší a najjednoduchší z indukčných algoritmov. Deliacim kritériom je Informačný zisk (\ref{kap1:2.5:2.5.1:InfoGainDef}). Kritériom zastavenia je nulový informačný zisk alebo dosiahnutie homogénnosti výstupného atribútu. Algoritmus nevyužíva žiadne techniky orezávania, nedokáže riešiť problémy s chýbajúcimi hodnotami v atribútoch a nevie pracovať s numerickými atribútmi.

Najväčšou výhodou ID3 je jeho jednoduchosť. Zvyčajne je kvôli tomu využívaný pre vzdelávacie účely. 

Nevýhody tohoto algoritmu presahujú spomenuté výhody. Stromy vytvorené týmto algoritmom sa nachádzajú v nejakom lokálnom optime (hladný algoritmus).
Vytvorené stromy bývajú väčšinou malé, ale keď algoritmus vytvorí väčší strom tak dochádza veľmi často k preučeniu. Preto pri tomto druhu algoritmu sú preferovanejšie menšie stromy. Algoritmus nepodporuje numerické hodnoty a pre prácu s nimi je nutné predspracovať dáta do nejakých diskrétnych skupín (hodnoty menšie ako $x$ budú v skupine $a$, ostatné v skupine $b$).
\subsection{C4.5}\label{kap1:2.5:2.5.4:C4.5}
C4.5 je vylepšením algoritmu ID3. Obidva majú rovnakého autora. Algoritmus používa koeficient zisku (\ref{kap1:2.5:2.5.1:GainRatioDef}) za deliace kritérium. Vytváranie stromu končí, keď počet inštancií pri delení je menej ako určitý threshold. Zároveň využíva techniky orezávania a dokáže pracovať aj s numerickými atribútmi. C4.5 vie popri tomu  pracovať aj s chýbajúcimi hodnotami atribútov.

Indukcia pomocou C4.5 poskytuje oproti ID3 veľa zlepšení. Pri orezávaní odstraňuje vetvy, ktoré neprispievajú k presnosti a nahradzuje ich listami. Inštanciám môžu chýbať niektoré hodnoty atribútov. Numerické atribúty sú riešené rozdelením domény daného atribútu na dve podmnožiny podľa deliaceho bodu. Tento deliaci bod je zvolený tak, aby maximalizoval koeficient zisku.

Jedným z ďalších vylepšení C4.5 je ešte o niečo novší, komerčný algoritmus C5.0. Popri výhodám z C4.5 obsahuje ešte ďalšie zlepšenia, ako je rýchlosť výpočtu a zlepšenie práce s pamäťou. Ďalším zlepšením je vstavané využitie techniky boosting, ktorá dokáže zvýšiť prediktívne schopnosti modelu.

Implementovanú verziu algoritmu C4.5, nazývanú J48 môžeme nájsť už v spomínanom nástroji Weka.

\subsection{CART}\label{kap1:2.5:2.5.5:CART}
CART je hladový algoritmus, ktorý slúži na tvorbu klasifikačných aj regresných stromov. Autorom algoritmu je Breiman. Vytvorený strom je v tomto prípade vždy binárny (každý vnútorný vrchol ma dvoch potomkov). Algoritmus využíva Twoing kritérium (\ref{kap1:2.5:2.5.1:TwoingDef}) ako deliace kritérium. CART používa na orezávanie stromu metriku, ktorá berie do úvahy počet listov spolu s chybovosťou modelu (\cite{wiki-costcompprune}, \cite[s.382]{kap1-DecisionTree}).

Dôležitou vlastnosťou algoritmu je vytváranie regresných stromov. Vytvorené stromy obsahujú v listoch reálne hodnoty. V tomto prípade sú deliace body vyberané tak, aby minimalizovali hodnotu získanú metódou najmenších štvorcov. V listových vrcholoch je hodnota rovná váženému priemeru dát v tomto vrchole.

\section{Využitie rozhodovacích stromov pri predikcii}\label{kap1:2.6:DTUsage}
V tomto oddieli zhrnieme výhody a nevýhody rozhodovacích stromov, ktoré priamo ovplyvňujú využiteľnosť tohoto modelu v praxi. Zároveň predstavíme projekty, ktoré prakticky využívajú techniku rozhodovacích stromov. Knihy s informáciami, ktoré boli použité v tejto časti sú \cite{kap1-DataMiningAndAnalysis,kap1-DataMiningForTrees}. Informácie s praktickým použitím rozhodovacích stromov sme čerpali zo stránok príslušných projektov \cite{online-astronomy} a \cite{online-psychoterapy}.

V časti \ref{kap1:2.6:2.6.1:AdvAndDis} popíšeme výhody a nevýhody rozhodovacích stromov, ktoré priamo ovplyvňujú využitie v praxi. V ďalších častiach predstavíme dva projekty, ktoré využívajú rozhodovacie stromy pri predikcii. Filtrovanie šumu v obrázkoch získaných z Hubblovho vesmírneho ďalekohľadu bude popísané v časti \ref{kap1:2.6:2.6.2:Hubble}. Využitie rozhodovacích stromov pre klinickú prax predvedieme v poslednej časti \ref{kap1:2.6:2.6.3:Clinical}.

\subsection{Výhody a nevýhody stromov}\label{kap1:2.6:2.6.1:AdvAndDis}
V predchádzajúcich oddieloch sme spomínali rôzne výhody a nevýhody rozhodovacích stromov. Po zhrnutí môžeme medzi výhody zaraďovať:
\begin{itemize}
\item Rozhodovacie stromy majú jednoduchú reprezentáciu, z ktorej je vidieť chovanie stromu. Pri malom počte listov je reprezentácia ľahko spracovateľná aj neprofesionálnym užívateľom.
\item Model je dostatočne bohatý nato, aby existoval prevod s ľubovoľným iným diskrétnym klasifikátorom.
\item Cesty od koreňa k listom predstavujú jednotlivé pravidla. Existuje teda jednoduchý prevod medzi stromom a vytvorenými pravidlami.
\item Model rozhodovacích stromov je zaraďovaný medzi neparametrické techniky a teda neuvažuje žiadne predpoklady o štruktúre klasifikátora či rozdelení dát.
\item Rozhodovacie stromy dokážu pracovať s dátami, ktoré obsahujú chyby alebo ktorým chýbajú hodnoty niektorých atribútov.
\item Rozhodovacie stromy vedia pracovať s kategoriálnymi atribútmi, tak ako aj s numerickými.
\end{itemize}

K nevýhodám rozhodovacích stromov patrí:
\begin{itemize}
\item Hladové techniky tvorby stromov môžu uviaznuť v lokálnom optime. Stromy sú taktiež náchylné na preučenie.
\item Tvorba stromu hladovými technikami sa môže zamerať na nerelevantné atribúty a šum v dátach.
\item Krátkozrakosť algoritmu, kde delenia v jednotlivých vrcholoch pracujú iba s priamymi potomkami vrcholu.
\item Väčšina algoritmov požaduje, aby bol výstupný atribút kategoriálny (C4.5, ID3, \ldots).
\item Riešenie problémov chýbajúcich hodnôt atribútov. 
\item Problém fragmentácie pri rozdeľovaní dát vo vrcholoch. Pri rovnomernom rozdelení dát vo vrcholoch, na ceste z koreňa do listu, môžeme otestovať iba $\log n$ atribútov. Veľký problém pri existencii väčšieho počtu významných atribútov.
\end{itemize}

\subsection{Filtrovanie šumu z Hubblovho ďalekohľadu}\label{kap1:2.6:2.6.2:Hubble}
Projekt si dal za úlohu popísať nové metódy a poskytnúť nové nástroje na analýzu rozsiahlych, komplexných a mnoho atribútových dátových množín pre výskumnú komunitu astrofyziky. Článok popisuje spoločné úsilie výskumníkov z umelej inteligencie a astrofyziky pri vytváraní a používaní rôznych techník, ktoré riešia signifikantné problémy v astronómii. Konkrétne sa v práci snažia čo najlepšie využiť mnoho rôznorodých techník zo strojového učenia na identifikovanie a klasifikovanie objektov z astronomických obrázkov. Tieto poznatky sa následne snažia aplikovať pri identifikovaní šumu v obrázkoch získaných z Hubblovho vesmírneho ďalekohľadu (HVD). Šum, na ktorý sa zamerali bolo kozmické žiarenie, ktoré je veľmi častým úkazom pri týchto fotkách.

\begin{figure}[h]
\centering
\centerline{\mbox{\includegraphics{../img/kap1/DT-hubble.pdf}}}
\caption{Získaný model klasifikujúci vstupný vektor s 20 príznakmi (Čísla v elipsách znamenajú delenia vo vrcholoch. Percentá pod vrcholmi predstavujú relatívny počet hviezd ku kozmickému žiareniu). Obrázok modelu je prebraný a prepísaný z \cite[str. 284]{online-astronomy}}\label{fig:HubbleTree}
\end{figure}

Vytvorenie dát predchádzalo veľa práce a to hlavne získavaním obrázkov z ďalekohľadu. Následne bolo nutné obrázky kalibrovať. Prvý, experimentálny prístup rozdelil obrázky na skupiny veľkosti $3 \times 3$ pixele. Skupiny boli vycentrované na hviezdu, aby obsahovali čo najviac informácii o nej. Hodnoty v skupine boli intenzity svetla. Týmto prístupom previedli hviezdy a kozmické žiarenie do deväť rozmerného priestoru. Každá inštancia pre klasifikovanie teda obsahovala deväť príznakov. Druhý prístup je nadstavbou toho prvého a využíva pojmy ako odchýlka a difrakcia, ktoré popisujú rozdelenie intenzity svetla (tzv. funkcie šírenia z bodu FSB) pre objekty ako napr. hviezdy. Pre hviezdy a kozmické žiarenie by sa takto získané FSB malo rozlišovať v niektorých parametroch. Počet príznakov pri tomto prístupe je rovný 20. Príznaky použité pri klasifikácií boli intenzity pixelov skupín veľkosti $3 \times 3$ spolu s parametrami FSB a s ďalšími vedomosťami (rozdiely medzi hviezdami a kozmickým žiarením).
Pri riešení využili ich vlastný model rozhodovacích stromov nazývaný OC1. Tento ich model spočíva v tom, že uvažuje aj také rozdeľovacie nadroviny, ktoré nie sú rovnobežné s hlavnými osami. Viac o algoritme je priamo v článku \cite[str. 281]{online-astronomy}.

Modely, vytvorené metódou OC1 dosahujú presnosť až 95\%. Experimenty, nachádzajúce sa v článku uvádzajú, že zlepšením metód na eliminovanie šumu môže ďalej vylepšiť presnosť vytvorených klasifikátorov. Príklad takto vytvoreného stromu je na obrázku \ref{fig:HubbleTree}

\subsection{Rozhodovacie stromy pre klinickú prax}\label{kap1:2.6:2.6.3:Clinical}
Tento projekt sa snaží preklenúť priepasť medzi vedou a praktickým využitím v oblasti duševnej zdravotnej starostlivosti a priviesť novú metódu, ktorú bude možné využiť v klinickej praxi.  Existencia dokonalého štatistického modelu, ktorý by dokonale predikoval výsledok liečby, je ale nepravdepodobná. V tomto projekte sa ale snažia čo najlepšie identifikovať a popísať možný neúspech liečby a ako takému neúspechu predchádzať pomocou správnych rozhodnutí pri prebiehajúcej liečbe. Projekt odkazuje na existujúce štúdiá, ktoré síce vykazujú priemerné výsledky, no neposkytujú dostatočné predikčné schopnosti pre rozhodovanie v praxi.

Dáta, na účely vytvorenia modelu, boli získané z iného projektu (Project TR-EAT), ktorý skúmal vzťahy medzi dĺžkou/intenzitou liečby poruchy prijímania potravy a výsledkom liečby. Z dát bola vybratá vzorka ľudí diagnostikovaná bulímiou nervosa ($n$ = 630). Pri výbere dát boli zaujímavé hlavne údaje o prijatí, liečebná stratégia a odpoveď na liečbu. Na túto vzorku bol použitý algoritmus CART. Vytvorený model má podľa projektu potenciál, pretože popisuje riziko neúspechu podľa dostupných liečebných možností. Model je podľa projektu zároveň citlivý na unikátne vlastnosti pacienta a výsledok liečby tohoto pacienta.

\begin{figure}[h]
\centering
\centerline{\mbox{\includegraphics[width=385pt]{../img/kap1/DT-clinical.pdf}}}
\caption{Získaný model klasifikujúci výsledok liečby (Čísla v elipsách predstavujú veľkosti skupín získané po delení. Percentá pod vrcholmi predstavujú relatívny počet ľudí s lepším verzus horším výsledkom. SCLGSI = Symptom Checklist-90-R Global Severity Index). Obrázok modelu je prebraný a prepísaný z \cite[s. 454]{online-psychoterapy}}\label{fig:ClinicalTree}
\end{figure}

Model získaný algoritmom CART, ktorý bol ešte zredukovaný, je optimálnym kompromisom medzi zložitosťou a prediktívnymi schopnosťami modelu. Strom obsahoval desať listových vrcholov a jeho chybovosť bola 18\%. Zo 448 pacientov, ktorým bol predpovedaný zlý výsledok malo 86\% skutočne zlý výsledok. Zo 182 pacientov s predikovaným dobrým výsledkom malo 74\% dobrý výsledok.

Výsledný strom tohoto projektu, vytvorený metódou CART, je na obrázku \ref{fig:ClinicalTree}.
\chapter{Genetické algoritmy}
Hlavným cieľom práce je optimalizovanie tvorby rozhodovacích stromov pomocou genetických algoritmov (GA). Preto 
v tejto kapitole predstavíme genetické algoritmy a podrobnejšie popíšeme jeho jednotlivé časti, ktoré budeme ďalej potrebovať v ďalšej kapitole.

Táto kapitola je delená do viacerých oddielov. V oddieli \ref{kap2:2.1:Info} popíšeme základné informácie o genetických algoritmoch. Genetické operátory používané v genetických algoritmoch zavedieme v oddieli \ref{kap2:2.2:Operators}. Rôzne druhy funkcií, používaných na vyhodnotenie jedincov, zadefinujeme v oddieli \ref{kap2:2.3:Fitnesses}. Ďalej v časti \ref{kap2:2.4:Selection} predvedieme najzákladnejšie metódy na výber jedincov. Nakoniec v časti \ref{kap2:2.5:Parameters} popíšeme problém výberu správnych parametrov v genetických algoritmoch.

\section{Základný popis genetických algoritmov}\label{kap2:2.1:Info}
Genetické algoritmy sú optimalizačné techniky a prehľadávacie heuristiky, ktoré napodobujú princíp prirodzeného výberu vychádzajúceho z darwinovskej teórie. V tomto oddieli ponúkneme úvod ku genetickým algoritmom s popisom, ako by mal taký genetický algoritmus vyzerať. Informácie o genetických algoritmoch sme čerpali z knihy \cite{kap2-evolution} a úvod do oblasti genetiky zase z voľne dostupnej učebnice biológie \cite{online-biology}. Ďalšie doplňujúce poznatky boli získané z online zdrojov \cite{wiki-evolution,wiki-genetics}.

V časti \ref{kap2:2.1:2.1.1:Genetics} najprv z pohľadu biológie vysvetlíme základné pojmy z genetiky. Ďalej v časti \ref{kap2:2.2:2.1.2:}.

\subsection{Genetika z pohľadu Biológie}\label{kap2:2.1:2.1.1:Genetics}
Na popísanie a zadefinovanie ďalších súčastí genetických algoritmov je dobré pripomenúť niektoré základné pojmy z oblasti genetiky. 

Bunka je základnou stavebnou jednotkou každého organizmu. Každá bunka má uložený popis toho ako sa má správať pri typických bunečných procesoch. Tento popis sa nazýva genetická informácia a v organizmoch je uskladnená v DNA, čo je obrovská molekula, polymér zložený z veľkého množstva pospájaných nukleotidov (kyselina fosforečná s deoxyribózou a naviazanou dusíkatou bázou). Tieto nukleotidy a konkrétne ich báze sú zodpovedné zato, ako vyzerá genetický kód tejto genetickej informácie. Malý kúsok vybraný zo sekvencie DNA, ktorý kóduje nejakú funkčnosť bunky alebo proces v bunke, sa nazýva gén. Presnejšie vyjadrenie by bolo, že gén je zodpovedný za tvorbu proteínov, ktoré sú ovplyvňujú funkčnosť bunky. Gén si je teda možné predstaviť ako významnú črtu organizmu (napríklad farba očí). Rôzne nastavenia génu nazývame alely (modré, hnedé, zelené oči).

DNA je kvôli jej veľkosti ďalej komprimovaná do štruktúry nazývaná chromozóm. Chromozóm sa nachádza v jadre bunky. Organizmy môžu mať v jadre viacero rôznych chromozómov. Podľa typu organizmu môžu byť tieto chromozómy spárované (diploidné organizmy) alebo nespárované (haploidné organizmy). Genóm organizmu potom nazývame genetický materiál získaný zo všetkých chromozómov dokopy. Určitá množina génov z genómu organizmu sa nazýva genotyp. Vo vývojovom štádiu jedinca ovplyvňuje genotyp výzor a ďalšie charakteristiky organizmu (výška, farba očí, IQ, ...), nazývané aj fenotyp.

\begin{figure}[h]
\centering
\centerline{\mbox{\includegraphics[width=250pt]{../img/kap2/DNA.pdf}}}
\caption{Na obrázku je príklad bunky s jadrom, v ktorom sú viditeľné chromozómy. Jeden z chromozómov je v spodnej časti roztiahnutý do DNA, ktorej jednotlivé časti sú popísané. Značky A,G,T predstavujú nukleotidové báze adenín, guanín a cytozín.}\label{fig:DNA}
\end{figure}

Úlohou každého živého organizmu je rozšíriť svoju genetickú informáciu a tým pádom aj jeho gény. Sexuálna reprodukcia dvoch rodičovských organizmov je komplikovaný proces (zdroj \cite{online-shuffling} podáva rozumné vysvetlenie celého procesu). Konkrétne nás budú zaujímať dve štádia reprodukcie, ktoré funguje na úrovni chromozómov.
\begin{itemize}
\item Kríženie jednoduchých buniek alebo komplexných živočíchov, kde každá dvojica chromozómov si medzi sebou vymieňa gény a vytvára nové chromozómy.
\item Mutácia vytvorených chromozómov, pri ktorej sú niektoré nukleotidy pozmenené kvôli chýbam pri krížení.
\end{itemize}
Po reprodukcii dostávame nové chromozómy, ktoré tvoria genóm novo vytvoreného organizmu. Na obrázku \ref{fig:Chromosomes} je príklad reprodukcie nespárovaných chromozómov pre haploidné organizmy.

\begin{figure}[h]
\centering
\centerline{\mbox{\includegraphics[width=350pt]{../img/kap2/chromosomes.pdf}}}
\caption{Príklad zjednodušeného procesu reprodukcie chromozómov pre nového jedinca. Na obrázku a) sú chromozómy z rodičov. Gény v chromozómoch na obrázku b) sú vyznačené čiernymi plnými krúžkami. Kríženie génov v chromozómoch (vyznačené šípkami) a mutácia jedného z génov (šedý krúžok) je na obrázku c). Na obrázku d) sú vzniknuté chromozómy nového jedinca.}\label{fig:Chromosomes}
\end{figure}

Z biologického hľadiska je nutné popísať kvalitu nových organizmov. Tú je možné vyhodnotiť pomocou pravdepodobnosti, že sa jedinec dožije ďalšieho párenia. Ďalším kritériom môže byť plodnosť potomka.

\subsection{Popis genetických algoritmov}
V tejto časti zadefinujeme, čo sú to genetické algoritmy, pričom jednotlivé pojmy budú podobné tým, ktoré sme predstavili v predchádzajúcej časti. 

Genetické algoritmy patria medzi metaheuristiky, ktorých úlohou je prechádzať priestor riešení a z nich nájsť to najlepšie. Kvalita vytvorených jedincov záleží na tom, ako dobre tento jedinec rieši zadanú úlohu. Väčšina implementácií pracuje s jedincami, ktoré by sa dali popísať ako haploidné (nespárované chromozómy).

V prípade genetických algoritmov sú chromozómy novo vytvorených organizmov jednotlivé riešenia. Chromozóm môže byť kódovaný rôzne. Medzi najznámejšie kódovania jedincov patria:
\begin{itemize}
\item binárne, pri ktorom reprezentujeme jedinca bitovým reťazcom/poľom určitej dĺžky,
\item znakové, kde je jedinec reprezentovaný reťazcom, ktorého hodnoty sú z konečnej množiny symbolov,
\item celočíselné, kde je jedinec reprezentovaný reťazcom celočíselných hodnôt,
\item kódovanie s reálnymi číslami, pri ktorom reprezentujeme jedinca reťazcom, ktorého hodnoty sú reálne čísla,
\item stromové, kde sú jedince reprezentované ako stromy z teórie grafov,
\item a mnoho ďalších ako sú permutácie, matice, konečné automaty, neurónové siete,...
\end{itemize}
Pri použití binárneho kódovania sú jedinci reprezentovaní ako bitové reťazce. V tomto prípade sú jednotlivé prvky alebo skupiny prvkov v tomto poli génmi jedinca. Alela je v prípade bitovej hodnoty 0 alebo 1. Pri iných reprezentáciách sú alely komplikovanejšie.

Kríženie v genetických algoritmoch je realizované výmenou hodnôt medzi rodičmi. To o aké hodnoty sa jedná záleží na zvolenej reprezentácií (hodnoty v reťazci, stromy, ...). Typ kríženia, ktorý môžeme použiť, závisí od zvoleného kódovania a reprezentácie jedinca. 
Mutácia je realizovaná zmenou hodnoty už v kríženom jedincovi. Podobne ako pri krížení závisí na typu kríženia aj na zvolenej reprezentácií. 

Napríklad v prípade kódovania, pri ktorom sú jedinci reprezentovaní reťazcom je typ kríženia obmedzený na bodové a uniformné kríženie. Rozumnou mutáciou je v tomto prípade invertovanie bitu.

\subsection{Jednoduchý genetický algoritmus}

\section{Genetické operátory}\label{kap2:2.2:Operators}
V prípade kódovania, ktoré využíva reprezentáciu reťazcom môžeme použiť
\begin{itemize}
\item bodové kríženie, ktoré je založené na zvolení náhodných $n$ bodov. Takto zvolené body rozdelia reťazec na $n+1$ častí, pričom sa odpovedajúce časti vymenia s určitou pravdepodobnosťou. Tento typ kríženia sa ďalej pomenováva podľa počtu vybraných bodov (jednobodové, dvojbodové, ...).
\item uniformné kríženie, ktoré vypĺňa jednotlivé hodnoty z určitého rodiča. U každej položky hádžeme mincou, ktorá rozhoduje z ktorého rodiča hodnotu vyberieme.
\end{itemize}

\section{Vyhodnotenie jedincov}\label{kap2:2.3:Fitnesses}

\section{Výber jedincov}\label{kap2:2.4:Selection}

\section{Nastavenie správnych parametrov}\label{kap2:2.5:Parameters}


\chapter{Genetická tvorba rozhodovacích stromov}\label{kap3:DTGA}
V časti \ref{kap1:2.7:DTTechniques} sme popísali vytváranie rozhodovacích stromov pomocou indukčných algoritmov. Vieme, že stromy vytvorené týmito algoritmami bývajú kvalitné, ale nemusia byť optimálne. To je kvôli tomu, že tieto algoritmy patria medzi tzv. hladné techniky. Ďalším spôsobom, ako vytvárať stromy je použiť techniku, ktorá prehľadáva priestor všetkých riešení (napríklad prehľadávaciu heuristiku). Pri použití niektorej z týchto techník môžeme dostať neoptimálne riešenie, ktoré môže byť dokonca horšie, než to vytvorené indukčným algoritmom. Vhodne zvolená a nastavená prehľadávacia heuristika ale môže nájsť riešenie, ktoré je významne lepšie než to vytvorené indukčným algoritmom. V kapitole \ref{kap2:GA} sme predviedli jednu z prehľadávacích heuristík nazývanú genetické algoritmy.

V tejto kapitole teda popíšeme genetický algoritmus, ktorý sa dá použiť na tvorbu rozhodovacích stromov, a ktorý sme implementovali v nástroji Weka. Popíšeme, ako kódovať jedincov v populácií genetického algoritmu tak, aby títo jedinci čo najlepšie popisovali rozhodovacie stromy. Operátory zavedené v predchádzajúcej kapitole rozšírime o nové genetické operátory špeciálne navrhnuté na tvorbu rozhodovacích stromov. Ostatné súčasti genetického algoritmu iba pripomenieme, prípadne ich spojíme s pojmami z rozhodovacích stromov.

Vytváranie stromov genetickými algoritmami nie je novou technikou. Existujú práce, ktoré sa venujú tejto problematike, no tie väčšinou vytvárajú stromy od úplného začiatku. Niektoré z týchto prác spomenieme v kapitole \ref{kap6:SimilarWorks}. Taktiež treba uviesť, že voľne dostupných implementácií je málo a zatiaľ sa zdá, že neexistuje žiadna do nástroja Weka.

Najskôr v oddieli \ref{kap3:3.1:Intro} popíšeme úlohu tvorby stromov genetickými algoritmami. Ďalej v oddieli \ref{kap3:3.2:Encoding} pripomenieme kódovanie jedincov pomocou stromov a ako sa dá toto kódovanie pozmeniť na reprezentovanie rozhodovacích stromov. V oddieli \ref{kap3:3.3:Fitness} uvedieme fitness funkcie používané v takomto genetickom algoritme. Na konci, v oddieli \ref{kap3:3.4:Operators}, pripomenieme operátor kríženia stromov a zadefinujeme dve nové mutácie stromov.
\section{Základný popis}\label{kap3:3.1:Intro}
V tomto oddieli popíšeme genetický algoritmus, ktorý sa dá použiť na vytvorenie rozhodovacích stromov. Techniky tvorby rozhodovacích stromov pomocou genetických algoritmov môžeme rozdeliť na dve skupiny:
\begin{enumerate}
\item vytváranie stromov úplne od začiatku,
\item optimalizovanie už vytvorených stromov a zlepšenie ich prediktívnych schopností,
\end{enumerate}
pričom obidve skupiny majú svoje výhody aj nevýhody.

Vytváranie stromov pomocou genetických algoritmov od začiatku je pomalé a náj\-denie vhodného riešenia môže trvať príliš dlho. Výhodou je, že stromy vytvárame od začiatku, a teda nie sú ovplyvňované takými stromami, ktorými by sme rýchlo skonvergovali k lokálnemu optimu. 

Pri optimalizovaní vytvoreného stromu môžeme využiť to, že počiatočná generácia je už na počiatku kvalitná. V niektorých prípadoch by stačila iba malá zmena na týchto kvalitných stromoch a generalizačné schopnosti pozmeneného stromu by sa významne zlepšili. V iných prípadoch, kedy genetický algoritmus nedokáže vylepšiť generalizačné schopnosti, môžeme vylepšovať iné vlastnosti stromu (napr. veľkosť stromu, výšku stromu) pričom významne nezhoršíme generalizačné schopnosti stromu. Na úkor dodatočného času teda môžeme dostať rozhodovacie stromy, ktoré svojou kvalitou prevyšujú alebo sa aspoň vyrovnávajú tým vytvorenými indukčnými algoritmami. Pre predstavu uvádzame dva obrázky rozhodovacích stromov vytvorených pre tú istú úlohu. Úloha pozostávala z klasifikácie náhodných dát, vygenerovaných generátorom \verb|RDG1|, ktorý ponúka nástroj \verb|Weka|. Prvý zo stromov bol vytvorený indukčným algoritmom (Obrázok \ref{fig:treeIA}) a druhý bol optimalizovaný genetickým algoritmom (Obrázok \ref{fig:treeGA}). Ani jeden z týchto modelov nemusí byť významne lepší v predikovaní, no veľkosť stromu optimalizovaného genetickým algoritmom je výrazne menšia než veľkosť stromu vytvoreného indukčným algoritmom.

\begin{figure}[h]
\centering
\centerline{\mbox{\includegraphics[width=400pt]{../img/kap3/tree1.pdf}}}
\caption{Príklad stromu vytvoreného nejakým indukčným algoritmom.}\label{fig:treeIA}
\end{figure}

V tejto práci sa kvôli jeho výhodám zameriame na optimalizovanie vytvorených stromov genetickými algoritmami. Počiatočné stromy nebudú náhodné, ale do určitej miery kvalitné. Kvalitu stromu vynútime použitím niektorého z algoritmov spomínaných v časti \ref{kap1:2.6:DTEvaluation}. 

Celým procesom optimalizácie stromov sa snažíme čo najviac zlepšiť (tj. zmenšiť) generalizačnú chybu, pričom môžeme vylepšovať aj ďalšie kritéria stromu, ako napríklad veľkosť stromu. Z časti \ref{kap1:2.6:2.6.1:Generalize} vieme, že generalizačnú chybu môžeme aproximovať empirickou chybou na nezávislej testovacej množine. Dátovú množinu $D$ teda rozdelíme na trénovaciu podmnožinu $T$ a validačnú podmnožinu $V$. Na ešte lepšiu aproximáciu generalizačnej chyby môžeme použiť $n$-násobnú krížovú validáciu viz. časť \ref{kap1:2.6:2.6.1:Generalize}. Obidva prístupy sú založené na ohodnotení kvality vytvorených stromov. Algoritmus by mal kvôli tomu poskytovať nejakú formu aproximácie generalizačnej chyby.

\begin{figure}[h]
\centering
\centerline{\mbox{\includegraphics[width=300pt]{../img/kap3/tree2.pdf}}}
\caption{Príklad stromu vytvoreného indukčným algoritmom a optimalizovaný genetickým algoritmom.}\label{fig:treeGA}
\end{figure}

Počiatočné rozhodovacie stromy, ktoré budeme optimalizovať, teda budú vybudované z trénovacej množiny $T$. Operátory budú slúžiť na prehľadávanie priestoru riešení a vyhľadanie nových rozhodovacích stromov. Typy operátorov sú zvolené tak, aby čo najviac diverzifikovali populáciu. Na vyberanie jedincov je vhodné použiť elitizmus (kvôli udržaniu najlepšieho riešenia) a turnajovú selekciu hlavne kvôli jej výhodám spomínaným v časti \ref{kap2:2.4:2.4.2:Tournament}. Genetickým algoritmom vytvárame nové stromy, ktorých kvalitu určíme podľa zvolených kritérií. Na výpočet fitness z jedného alebo viacerých kritérií použijeme niektorú techniku z časti \ref{kap2:2.3:Fitnesses}. Validačná množina $V$ bude slúžiť na otestovanie modelu a na odhad generalizačnej chyby.

Pri tomto genetickom algoritme použijeme selekciu prostredia z časti \ref{kap2:2.4:2.4.4:EnvironmentalSelection}, čo spolu s elitizmom zabráni strate tých najlepších jedincov a zaručí, že populácia zostane aj naďalej dostatočne diverzifikovaná.

V časti \ref{kap2:2.1:2.1.2:AboutGeneticAlgo}, konkrétne na Obrázku \ref{fig:GAdiagram}, sme zaviedli diagram genetického algoritmu. Podľa tohoto diagramu sa riadi aj chovanie genetického algoritmu na optimalizovanie stromov.
V algoritme \ref{fig:DTGeneticAlgo} uvádzame pseudokód tohoto genetického optimalizovania stromov a popisujeme jeho jednotlivé súčasti.
Pseudokód je veľmi podobný tomu spomenutému v časti \ref{kap2:2.1:2.1.3:SimpleGeneticAlgo} v algoritme \ref{fig:geneticAlgoritm}.

\begin{algorithm}
\floatname{algorithm}{Algoritmus}
\caption{Kroky genetického algoritmu, ktorý optimalizuje rozhodovacie stromy vytvorené indukčným algoritmom.}\label{fig:DTGeneticAlgo}
$A$ - aktuálna populácia rozhodovacích stromov \\
$O$ - potomkovia \\
$e$ - hodnota elitizmu, $e \in [0,1]$ \\
$s$ - zvolená metóda selekcie (turnaj, ruleta) \\
$sp$ - zvolená metóda selekcie prostredia (turnaj, ruleta). \\
$f_1,\ldots,f_p$ - $p$ kritérií na vyhodnotenie rozhodovacích stromov pre problém $P$\\
$k_{1},\ldots,k_{q}$ - pravdepodobnosti $q$ operátorov kríženia\\
$m_{1},\ldots,m_{r}$ - pravdepodobnosti $r$ operátorov mutácie\\
$max\_gen$ - maximálny počet generácií \\
$max\_pop$ - maximálny počet jedincov v populácií \\
\bigskip
\begin{algorithmic}[1]
\State \parbox[t]{375pt}{$A \gets $ populácia $max\_pop$ stromov, vytvorených nejakým indukčným algoritmom.}
\For{$i = 1$ \textbf{to} $max\_gen$} 
	\State \textsc{Krok($A$)}
\EndFor
\\
\Procedure{Krok}{$A$}
	\State $O \gets \{\}$
	\State \textsc{VyhodnoťPopuláciu($A$)}
	\While{$\lvert O \rvert < max\_pop$}
		\State \parbox[t]{350pt}{$J_1,J_2 \gets $ \textsc{VyberJedincov($A$)}}
		\State \parbox[t]{350pt}{$K_1,K_2 \gets $ \textsc{Kríženie($J_1$,$J_2$)}}
		\State \parbox[t]{350pt}{$O_1 \gets $ \textsc{Mutácia($K_1$)}}
		\State \parbox[t]{350pt}{$O_2 \gets $ \textsc{Mutácia($K_2$)}}		
		\State $O \gets \{O_1,O_2\} \cup O$.
	\EndWhile
	\State $E \gets$ \textsc{Elitizmus($A$,$e$)}.
	\State $A \gets  E$ $\cup$ \textsc{VyberJedincovProstredia($O$)} 
\EndProcedure
\\
\Procedure{VyhodnoťPopuláciu}{$A$}
\State \parbox[t]{350pt}{každý rozhodovací strom $x$ z aktuálnej populácie $A$ ohodnoť kritériami $f_i, 1 \leq i \leq p$.}
\EndProcedure
\\
\Function{VyberJedincov}{$A$}
\State \Return \parbox[t]{300pt}{$O_1$,$O_2$ vybraných metódou selekcie $s$ z populácie $A$.}
\EndFunction
\\
\Function{Kríženie}{$J_1$,$J_2$}
\State $K_1, K_2 \gets J_1, J_2$
\For{$i = 1$ \textbf{to} $q$} 
\State $r \gets [0,1]$
\If{$r < \sum_{j=1}^{i}k_j$}
	\State \parbox[t]{300pt}{kríženie $i$ aplikuj na stromy $K_1, K_2$.}
\EndIf
\State \Return $K_1$,$K_2$
\EndFor 
\EndFunction
\algstore{myalg}
\end{algorithmic}
\end{algorithm}

\begin{algorithm}
\ContinuedFloat
\caption{pokračovanie...}
\begin{algorithmic}[1]
\algrestore{myalg}
\Function{Mutácia}{$K$}
\State $M \gets K$
\For{$i = 1$ \textbf{to} $r$} 
\State $r \gets [0,1]$
\If{$r < m_i$}
	\State \parbox[t]{300pt}{mutáciu $i$ aplikuj na strom $M$ (na jedinca môže byť použitých viacero mutácií).}
\EndIf
\EndFor 
\State \Return $M$
\EndFunction
\\
\Function{Elitizmus}{$A$,$e$}
\State \Return \parbox[t]{300pt}{$e |A|$ najlepších jedincov}
\EndFunction
\\
\Function{VyberJedincovProstredia}{$O$}
\State \Return \parbox[t]{300pt}{rozhodovacie stromy vybrané z $O$ selekciou prostredia $sp$. Počet vybraných stromov bude rovný $(1-e)max\_pop$.}
\EndFunction
\end{algorithmic}
\end{algorithm}

\section{Kódovanie jedincov}\label{kap3:3.2:Encoding}
Jedinci v tomto genetickom algoritme sú rozhodovacie stromy. Z časti \ref{kap1:2.3:2.3.2} vieme, že rozhodovací strom ma štruktúru stromu. Najrozumnejším kódovaním je preto kódovanie pomocou stromov, ktoré bolo spomínané v predchádzajúcej kapitole, v časti \ref{kap2:2.2:2.2.3:Tree}. Toto kódovanie je ale nutné upraviť tak, aby bralo do úvahy zvyšok definície rozhodovacieho stromu spolu s kritériami delenia zadefinovanými v oddieli \ref{kap1:2.5:DTSplitCriterias}.

Vnútorný uzol $u$ rozhodovacieho stromu je kódovaný trojicou $(a,z,h)$, kde $a$ predstavuje atribút delenia, $z$ porovnávací operátor použitý pri delení a $v$ hodnotu, v ktorej delíme dátovú množinu. Vieme, že typ atribútu môže byť numerický alebo kategoriálny. Podľa typu atribútu $a$ v trojici $(a,z,h)$ sú $z$ a $h$ definované takto:
\begin{enumerate}
\item V prípade numerického atribútu $a$ je $z$ porovnávací operátor a $h$ reálna hodnota.
\item V prípade kategoriálneho atribútu $a$ sú $z$ a $h$ nedefinované. Každá hrana vychádzajúca z vrcholu zodpovedá jednej možnej hodnote atribútu $a$.
\end{enumerate}
Počet hrán vychádzajúcich z vrcholu $u$ sa riadi typom atribútu $a$.
\begin{enumerate}
\item V prípade numerického atribútu sú hrany dve.
\item V prípade kategoriálneho atribútu je počet hrán rovný veľkosti definičného oboru atribútu $a$.
\end{enumerate}
Listový uzol $l$ rozhodovacieho stromu obsahuje iba hodnotu $c$, ktorá predstavuje klasifikáciu.
Takto kódovaný rozhodovací strom je na Obrázku \ref{fig:DTEncode}.

\begin{figure}[h]
\centering
\centerline{\mbox{\includegraphics{../img/kap3/dtgen.pdf}}}
\caption{Príklad kódovania rozhodovacieho stromu v genetickom algoritme na tvorbu rozhodovacích stromov. Nech hodnoty výstupného atribútu sú z množiny $C=\{c_0,c_1\}$. Na obrázku sú tri vnútorné uzly s atribútami $x,y,z$. Atribúty $x$ a $y$ sú numerické a $z$ je kategoriálny. Nedefinovaná hodnota operátora a nedefinovaná hodnota atribútu (u kategoriálneho atribútu) sú označené *. Hrany predstavujú rozhodnutia a počet hrán závisí na type atribútu uzlu. Listové vrcholy obsahujú hodnoty z množiny $C$ (klasifikáciu).}\label{fig:DTEncode}
\end{figure}

\section{Fitness funkcie}\label{kap3:3.3:Fitness}
Fitness funkcie v genetickom algoritme určujú, ako často budú jedinci vyberaní k reprodukcii. Tieto funkcie volíme podľa toho, čo chceme v rozhodovacom strome optimalizovať. Voľba vhodnej fitness funkcie v genetickom algoritme ovplyvňuje
\begin{itemize}
\item prediktívne vlastnosti stromu, kam zaraďujeme trénovaciu, testovaciu a generalizačnú chybu, 
\item zložitosť stromu, meranú napríklad výškou stromu, počtom listov alebo počtom vrcholov.
\end{itemize} 

Na optimalizovanie prediktívnych vlastností stromu musíme voliť také fitness funkcie, ktoré s predikciou súvisia. Tieto fitness funkcie väčšinou počítajú určitú metriku z matice chybovosti. Na zlepšenie prediktívnych vlastností nie je každá metrika vhodná. Kvôli tomu je lepšie sa zamerať iba na niektoré z nich.
V časti \ref{kap1:2.6:2.6.2:Alternatives} sme definovali alternatívne metriky založené na matici chybovosti, ktoré môžeme použiť pri ohodnocovaní rozhodovacích stromov. Spomínané kriteriá citlivosti, precíznosti, špecificity a f-metriky sú vhodnými kandidátmi fitness funkcií. Na zlepšenie generalizačnej chyby by mal byť výpočet metrík vykonaný na validačnej množine.

Z časti \ref{kap2:2.2:2.2.3:Tree} vieme, že stromy v genetických algoritmoch majú tendenciu narastať, a je nutné ich veľkosť obmedziť. To sa dá dosiahnuť voľbou fitness funkcií, ktoré ohodnocujú zložitosť stromu. Takéto fitness funkcie využívajú metriky zadefinované v časti \ref{kap1:2.6:2.6.3:TreeLook}. K existujúcim funkciám môžeme definovať nové fitness funkcie, ktoré budeme nazývať \emph{relatívne}. Takéto funkcie sú počítané vzhľadom k nejakej hodnote \emph{x}, v ktorej zároveň dosahujú maximum. Na Obrázku \ref{fig:normalvsrelative} uvádzame grafy dvoch fitness funkcií (normálnej a relatívnej) pre veľkosť stromov.

Vzorec na výpočet relatívnej fitness funkcie musí pracovať s hodnotou $y$, od ktorej túto fitness funkciu chceme dopočítať. Nech $x$ je skutočná hodnota fitness funkcie a $b$ je základ tejto funkcie. Relatívnu fitness funkciu môžeme definovať jedným zo vzorcov
\begin{itemize}
\item $\dfrac{1}{b^{|x-y|}}$
\item $\dfrac{1}{b^{(x-y)^2}}$
\item pozmenenou gaussovskou funkciou $b^{-\dfrac{(x-y)^2}{2z^2}}$, kde $z$ je smerodajná odchýlka
\end{itemize}

\begin{figure}[h]
\centering
\centerline{\mbox{\includegraphics[width=400pt]{../img/kap3/normvsrel.pdf}}}
\caption{Príklad grafov normálnej a relatívnej fitness funkcie hodnotiace veľkosť stromov. Naľavo je normálna fitness funkcia, ktorá dosahuje maximum pre najmenšiu veľkosť stromu a vpravo je relatívna fitness funkcia s maximom v hodnote \emph{y}. Vzorec ľavej fitness funkcie je $\left(\dfrac{1}{x}\right)$ a vzorec pravej fitness funkcie je $\left(\dfrac{1}{2^{|x-y|}}\right)$.}\label{fig:normalvsrelative}
\end{figure}

Voľba správnych fitness funkcií je obtiažnou úlohou. Optimálne by mali zvolené funkcie zmenšovať generalizačnú chybu a zbytočne nekomplikovať stromy. Tým sa úloha stáva multi-kriteriálnym problémom a výpočet kvality jedinca sa robí jednou z metód spomínaných v časti \ref{kap2:2.3:Fitnesses}.
\section{Operátory na tvorbu rozhodovacích stromov}\label{kap3:3.4:Operators}
V oddieli \ref{kap2:2.5:Operators} sme zadefinovali genetické operátory a uviedli ich dôležitosť v genetických algoritmoch. V prípade genetickej tvorby rozhodovacích stromov sú operátory zodpovedné za vytváranie nových rozhodovacích stromov. Tieto operátory by mali pracovať v súlade s dátovými množinami. To znamená nevykonávať náhodné zmeny, ale iba také, ktoré nejako súvisia s dátami.

Z predchádzajúcich častí vieme, že v genetických algoritmoch obvykle najprv vykonávame kríženie a potom mutáciu. Taktiež vieme, že každý operátor aplikujeme iba s určitou pravdepodobnosťou, a že toto nastavenie je neľahkou úlohou. Jednou z možností je inšpirovať sa oddielom \ref{kap2:2.6:Parameters}.

V časti \ref{kap3:3.4:3.4.1:Crossover} pripomenieme kríženie podstromov. Potom v časti \ref{kap3:3.4:3.4.2:Mutation} predvedieme dva nové druhy mutácií.

\subsection{Kríženie}\label{kap3:3.4:3.4.1:Crossover}
V časti \ref{kap2:2.5:2.5.1:Crossover} sme definovali operátor kríženia podstromov. V prípade stromov je toto jediné kríženie, ktoré dáva zmysel, a preto ho budeme používať aj pri genetickej tvorbe rozhodovacích stromov.

Toto kríženie ale musíme upraviť tak, aby v prípade kategoriálneho atribútu kontrolovalo prekrížené stromy. Krížený podstrom môže obsahovať vrchol, ktorého atribút sa už vyskytuje na ceste do koreňa a hodnoty atribútu sú nekonzistentné. V takomto prípade ho nahradím správnym potomkom (s konzistentnou hodnotou atribútu). Po zmene znova vykonám kontrolu na tomto potomkovi a opakujem tento postup, dokým nie sú tieto atribúty konzistentné.

\subsection{Mutácie}\label{kap3:3.4:3.4.2:Mutation}
V časti \ref{kap2:2.5:2.5.2:Mutation} sme uviedli rôzne druhy mutácií jedincov kódovaných stromami. Po menších úpravách sa dajú tieto mutácie použiť v genetickom algoritme na tvorbu rozhodovacích stromov. Mutácie je väčšinou nutné upraviť tak, aby brali do úvahy kódovanie uvedené v časti \ref{kap3:3.2:Encoding}. Jednotlivé mutácie by tiež mali vykonávať zmeny, ktoré sú v súlade s danou dátovou množinou $D$ (zabránenie náhodným zmenám).

Jednou zo spomínaných mutácií, ktorú použijeme v genetickom algoritme, je zmena náhodného vnútorného uzlu na list. Táto mutácia teda odstráni všetky odchádzajúce hrany zo zvoleného vrcholu a nastaví hodnotu klasifikácie.

Nech $S$ je strom, na ktorom vykonávame mutáciu, $u$ je \emph{vnútorný} uzol stromu $S$. Ďalej nech $D_u$ je podmnožina množiny $D$ získaná delením $D$ podľa vrcholov (konkrétne atribútov a hodnôt vo vrchole) na ceste z koreňa do vrcholu $u$ a $\max_c u$ je najčastejšia hodnota výstupného atribútu v $D_u$. Potom jednotlivé kroky mutácie sú
\begin{enumerate}
\item zvoľ vnútorný uzol $u$,
\item nahraď $u$ listom s klasifikáciou $\max_c u$.
\end{enumerate}
Príklad takejto mutácie, nahradzujúcej podstrom listom, je znázornený na Obrázku \ref{fig:mutnodetoleaf}.

\begin{figure}[h]
\centering
\centerline{\mbox{\includegraphics[width=400pt]{../img/kap3/mutnodetoleaf.pdf}}}
\caption{Príklad mutácie rozhodovacieho stromu, ktorá mení vnútorný uzol na list. Zvolený vnútorný uzol $u$ (nech $\max_c u = c_0$) je označený čiarkovaným obdĺžnikom. Na obrázku vpravo je vnútorný uzol nahradení listom s hodnotou $c_0$ (klasifikáciou).}\label{fig:mutnodetoleaf}
\end{figure}

Druhou použitou mutáciou je priradenie nového podstromu do nejakého uzlu stromu. Podstromy budú generované z podmnožín dátovej množiny $D$ a uložené v predom definovanej množine podstromov. Počet podmnožín bude rovný počtu vytváraných podstromov. Tieto podmnožiny môžeme vytvárať:
\begin{enumerate}
\item delením dátovej množiny $D$ na toľko častí, koľko podstromov budeme vytvárať,
\item výberom s opakovaním z dátovej množiny $D$.
\end{enumerate}
Nové podstromy na priradenie by nemali byť náhodné a ani zbytočne veľké. Mutácia preto používa rozhodovacie korene (definícia \ref{kap1:2.3:2.3.2:stumpDT}).

Nech $K=\{K_1,\ldots,K_r\}$ je množina podstromov (rozhodovacích koreňov). Ďalej nech $S$ je strom, na ktorom vykonávame mutáciu, $u$ ľubovoľný uzol stromu $S$. Potom jednotlivé kroky mutácie sú 
\begin{enumerate}
\item vyber náhodne podstrom $K_i$ z $K$,
\item vyber náhodne vrchol $u$ stromu $S$,
\item nahraď $u$ podstromom $K_i$.
\end{enumerate}
Príklad mutácie nahradzujúcej podstrom rozhodovacím koreňom je na Obrázku \ref{fig:mutdecisionstump}.

Podobne ako pri krížení je treba túto mutáciu upraviť, aby v prípade kategoriálneho atribútu kontrolovala, či sa atribút vrcholu rozhodovacieho koreňa nenachádza na ceste do koreňa stromu a hodnoty atribútu by boli nekonzistentné. V takom prípade nahradíme vrchol jeho potomkom  (to je určite list) pre správny atribút.

\begin{figure}[h]
\centering
\centerline{\mbox{\includegraphics[width=400pt]{../img/kap3/mutdecisionstump.pdf}}}
\caption{Príklad mutácie rozhodovacieho stromu pomocou rozhodovacích koreňov. Na obrázku hore je množina vygenerovaných rozhodovacích koreňov $K$. Na obrázku dole je samotná mutácia stromu. Uzol určený na mutáciu je označený čiarkovaným štvorcom. Na obrázku (dole vpravo) je zvolený vrchol $u$ (list) nahradený rozhodovacím koreňom $K_1$. }\label{fig:mutdecisionstump}
\end{figure}
\chapter{Implementácia}
V tejto kapitole popíšeme implementáciu a technické detaily aplikácie \verb|GenDTLib|, ktorá vytvára rozhodovacie stromy pomocou genetických algoritmov tak, ako sme uviedli v predchádzajúcej kapitole. Táto aplikácia zároveň slúži ako zásuvný modul do nástroja Weka. Najskôr v časti \ref{kap4:4.1:Info} popíšeme základné technické informácie o aplikácii \verb|GenDTLib|. Ďalej v časti \ref{kap4:4.2:About} uvedieme štruktúru aplikácie z pohľadu kódu. Ďalšími časťami sa zameriame na jednotlivé súčasti genetického algoritmu z predchádzajúcej kapitoly. V časti \ref{kap4:4.3:Implementation} predvedieme implementáciu genetického algoritmu použitého v aplikácii. V časti \ref{kap4:4.4:Components} sa budeme zaoberať implementáciou jednotlivých komponent genetického algoritmu. V poslednej časti \ref{kap4:4.5:Plugin} popíšeme možnosti rozšírenia aplikácie o nové komponenty a využitie aplikácie aj bez nástroja Weka.
\section{Technický popis}\label{kap4:4.1:Info}
V jednotlivých častiach popíšeme základné technické detaily aplikácie. Najprv v časti \ref{kap4:4.1:4.1.1:Requirements} zadefinujeme tieto požiadavky. Ďalej v časti \ref{kap4:4.1:4.1.2:Environment} uvedieme platformu aplikácie. V časti \ref{kap4:4.1:4.1.3:SystemRequirements} uvedieme systémové požiadavky na spustenie aplikácie. Nakoniec v časti \ref{kap4:4.1:4.1.4:Libs} predstavíme knižnice, ktoré boli použité v implementácií aplikácie GenLib.

\subsection{Požiadavky na aplikáciu}\label{kap4:4.1:4.1.1:Requirements}
Pri návrhu aplikácie sme kládli určité nároky na funkcionalitu aplikácie. Tie najzákladnejšie z nich sú:
\begin{itemize}
\item viacplatformovosť, t.j. schopnosť spustiť aplikáciu na viacerých operačných systémoch, čo je pre každú aplikáciu len pozitívom,
\item jednoduché užívateľské prostredie aplikácie, čím myslíme priamočiare ovládanie aplikácie a jednoduché nastavenie parametrov.
\item jednoduchá konfigurácia použitých komponent (viď. \ref{kap4:4.4:Components}) v genetickom algoritme, t.j. nefixujeme sa na pevne zadané operátory, fitness funkcie, selektory ale ponechávame možnosť ľubovoľne kombinovať tieto komponenty.
\item jednoduchá rozšíriteľnosť o nové komponenty genetického algoritmu, t.j. do aplikácie sa dajú doimplementovať nové druhy komponent, na ktoré sme prípadne v našej implementácií zabudli. 
\item aplikácia zároveň slúži aj ako zásuvný modul klasifikácie do nástroja Weka, t.j. aplikáciu budeme môcť použiť z vnútra nástroja Weka, ako jeden z klasifikátorov. 
\end{itemize}

Dôvody, prečo sme vybrali akúrát týchto 5 kritérií sú jednoduché.
Genetické algoritmy majú množstvo parametrov, ktoré treba nastaviť a zbytočne zložité užívateľské prostredie by komplikovalo prácu s aplikáciou. Jednoduchá konfigurácia komponent v genetickom algoritme zabezpečí, že bude stačiť jedna verzia aplikácie, ktorej chovanie je závislé na zmene jedného súboru. Možnosť rozšíriť aplikáciu o nové komponenty sme vybrali preto, aby sme minimalizovali počet zásahov do zdrojového kódu aplikácie, ktoré by mohli viesť k jej poškodeniu. Výhodou je, že užívateľ nepotrebuje vedieť to, ako celá aplikácia funguje z vnútra. Voľba vytvoriť zásuvný modul nám uľahčila prácu pri implementovaní tejto aplikácie kvôli tomu, že nástroj Weka poskytuje množstvo jestvujúcich a kvalitných funkcií na použitie (nie je ich potreba vytvárať na novo).

Na dosiahnutie viacplatformovosti je najľahšou možnosťou zvoliť taký programovací jazyk, ktorý je nezávislý na platforme.  Najrozumnejšou voľbou je jazyk Java, keďže chceme, aby aplikácia zároveň slúžila aj ako zásuvný model do nástroja Weka (ktorý je napísaný v jazyku Java). Užívateľské rozhranie aplikácie je grafické, no umožňuje aj ovládanie z príkazového riadku vhodné napríklad pre hromadné spracovávanie.

Na dosiahnutie ľubovoľnej konfigurovateľnosti sme navrhli programové rozhranie (API) tak, aby podporovalo aj budúce nové knižnice rozšírené o nové komponenty genetického algoritmu.

\subsection{Platforma aplikácie}\label{kap4:4.1:4.1.2:Environment}
Z časti \ref{kap1:2.2:2.2.5:Tools} vieme, že nástroj Weka je napísaný v jazyku Java.
Aplikáciu GenLib sme teda vyvíjali na platforme Java SE v jazyku Java, pre ktorý sme sa rozhodli kvôli požiadavke viacplatformovosti a požiadavke vytvorenia zásuvného modulu do nástroja Weka.
\subsection{Systémové požiadavky}\label{kap4:4.1:4.1.3:SystemRequirements}
Verzia behového prostredia Javy (Java Runtime Environment -- JRE) je závislá od verzie nástroja Weka. Najnovšia verzia nástroja Weka (v. >3.7.0) vyžaduje JRE verzie 6 alebo vyššej.

Na spustenie aplikácie GenLib vyžadujeme o verziu novšie JRE (JRE 7), a to kvôli jeho zmenám uvedených v \cite{online-java}, ktoré zjednodušili prácu pri programovaní. Nainštalovaná verzia Javy musí byť teda minimálne vo verzii 1.7. 

JRE 7 vyžaduje minimálne 128 MB pamäte RAM, no aplikácia sama o sebe vyžaduje minimálne 256MB pamäte RAM. Množstvo požadovanej pamäte záleží na veľkosti generovaných populácií a veľkosti klasifikovaných dát. Odporučané je mať aspoň 1GB pamäte RAM.

Rýchlosť procesoru je pri tejto aplikácii veľmi dôležitá. Za to môžu hlavne vyššie výpočtové požiadavky genetického algoritmu, ktorý pracuje s veľkým množstvom rozhodovacích stromov. Minimálna rýchlosť procesoru je Pentium 2 266 MHz, ktorá je uvedená v požiadavkách na používanie JRE 7.
\subsection{Použité knižnice}\label{kap4:4.1:4.1.4:Libs}
Aplikácia má slúžiť ako zásuvný modul do nástroja Weka, takže jedinou použitou knižnicou bude práve nástroj Weka. Weka uľahčuje veľké množstvo úloh, ktoré by sme inak museli implementovať sami. Všetky funkcie Weky, ako je klasifikácia dát, testovanie a porovnávanie modelov získame jej použitím automaticky. V aplikácii ďalej využívame iba základné balíčky z jazyka Java. 

Veľkou výhodou, pri použití knižnice Weka, je jednoduchá migrácia aplikácie GenLib do ďalšieho, spomínaného klasifikačného nástroja, RapidMiner (viz. \ref{kap1:2.2:2.2.5:Tools}).

\section{Popis aplikácie}\label{kap4:4.2:About}
Úlohou aplikácie \verb|GenDTLib| je vytvárať rozhodovacie stromy pomocou genetických algoritmov, tak ako sme toto vytváranie popísali v predchádzajúcej kapitole a umožniť ich použitie na klasifikáciu. Snahou bolo vytvoriť také rozhodovacie stromy, ktorých generalizačné schopnosti by boli lepšie než v prípade rozhodovacích stromov vytvorených indukčnými algoritmami. Popri tom sme kládli určité požiadavky na aplikáciu, aby bola z pohľadu užívateľa aj programátora jednoduchá na použitie.

V aplikácii vykonávame pri jednej inštancii behu algoritmu tri základné kroky:
\begin{enumerate}
\item vytváranie počiatočnej populácie podľa nejakého indukčného algoritmu z nástroja Weka,
\item použitie genetického algoritmu na vylepšenie modelu podľa daných kritérií a
\item klasifikovanie inštancií, prípadne porovnanie modelov medzi sebou.
\end{enumerate}

Priebeh algoritmu a vytváranie počiatočnej populácie sú ovládané pomocou konfiguračného súboru, ktorý obsahuje definície komponent (viz \ref{kap4:4.4:Components}). Komponenty ďalej obsahujú parametre, ktoré riadia ich chovanie. Tieto komponenty sú základné stavebné bloky algoritmu a nastavenie ich parametrov je závislé na type riešenej úlohy. Existujúce parametre sú jednoduché (nastavenie nejakej množiny hodnôt) aj zložené (kombinácia parametru s hodnotou). Pri niektorých komponentách je dôležité aj poradie parametrov. Návod na nastavenie komponent bude bližšie popísaný v časti xx.

V konfiguračnom súbore môžeme, na zrýchlenie aplikácie, definovať počet vlákien použitých pri počítaní fitness funkcií a generovaní počiatočnej generácie.

Keďže aplikácia slúži ako zásuvný modul do nástroja Weka, tak klasifikovanie inštancií a porovnanie modelov môžeme vykonávať priamo z nástroja Weka. Vytvorený algoritmus musí kvôli tomu implementovať niektoré rozhrania z WekaAPI.

\subsection{Platforma Weka}\label{kap4:4.2:4.2.1:Weka}
Z časti \ref{kap1:2.2:2.2.5:Tools} vieme, že Weka je široko uznávaný nástroj strojového učenia napísaný komunitou výskumníkov v Jave. Weka implementuje veľké množstvo nástrojov na klasifikáciu, štatistickú analýzu, testovania a porovnávanie modelov, ktoré sa dajú použiť ako z príkazovej riadky, tak i z grafického rozhrania. 

\begin{figure}[h]
\centering
\centerline{\mbox{\includegraphics{../img/kap4/wekalibs.pdf}}}
\caption{Pozmenený súbor, ktorý registruje existujúce a nové moduly v nástroji Weka. Na obrázku sú dve výrazné skupiny modulov: moduly na vyhodnotenie deliacich kritérií a klasifikačné moduly. Žlto označený riadok popisuje cestu ku novo vytvorenému klasifikátoru.}\label{fig:wekalibs}
\end{figure}

Weka, od verzie 3.5.8, podporuje jednoduché vytváranie nových externých modulov (napr. klasifikačné a vizualizačné moduly). Na vloženie nového klasifikačného modulu do nástroja Weka stačí pozmeniť konfiguračný súbor (v balíčku weka.jar) a pridať nový riadok s cestou k nami vytvorenému klasifikátoru, ako na Obrázku \ref{fig:wekalibs}

Nástroj Weka má svoje definované WekaAPI, ktoré uľahčuje tvorbu nových modulov. Každý modul má v ňom svoje definované rozhranie. Pri vytváraní nového modulu musíme implementovať toto rozhranie. Vytvorený modul môže potom zúžitkovať existujúce funkcie Weky. Na Obrázku \ref{fig:wekalibs} sú vypísané dve skupiny modulov, pričom jedna z nich popisuje klasifikátory (weka.classifiers.Classifier). Názvy týchto skupín zároveň určujú rozhranie modulu, ktoré musíme implementovať pri vytváraní nového modulu (nový klasifikátor musí implementovať weka.classifiers.Classifier).
Interakcie v kóde medzi nástrojom Weka a aplikáciou GenLib sú popísané v nasledujúcej časti \ref{kap4:4.2:4.2.2:CodeOrganization}

\pagebreak

\subsection{Organizácia kódu}\label{kap4:4.2:4.2.2:CodeOrganization}
Kód je v aplikácií GenLib delení do dvoch významných častí:
\begin{itemize}
\item hlavný genetický algoritmus na tvorbu stromov a
\item prepojenie genetického algoritmu s nástrojom Weka.
\end{itemize}



Prehľad tých najdôležitejších balíčkov a rozhraní (v našom vytvorenom API aj vo WekaAPI) je zobrazený v uml diagrame na Obrázku \ref{fig:GenLibPackage}. Interakcie medzi balíčkami sú znázornené šípkami.
\section{Implementácia genetického algoritmu}\label{kap4:4.3:Implementation}
\section{Popis komponent aplikácie}\label{kap4:4.4:Components}
\subsection{Generovanie počiatočnej generácie}
\subsection{Selekcie}
\subsection{Fitness funkcie}
\subsection{Operátory}
\section{Rozšíriteľnosť}\label{kap4:4.5:Plugin}
\subsection{Nové komponenty}
\subsection{Použitie bez Weky}
\chapter{Súhrnné testy}

\section{Metodika testovania}
\chapter{Podobné práce}\label{kap6:SimilarWorks}
V tejto kapitole popíšeme dve práce, ktoré sa zaoberajú vytváraním rozhodovacích stromov pomocou genetických algoritmov. Ani jedna z týchto prác neposkytuje program, ktorý by sa dal otestovať a porovnať s mojim riešením.

\section{LEGAL-Tree: A lexicographic multi-objective genetic algorithm for decision tree induction}\label{kap6: 6.1:Legal}
Cieľom práce a jej autorov \cite{kap6-legal} bolo vytvoriť taký genetický algoritmus na tvorbu rozhodovacích stromov, ktorý by dokázal nahradiť existujúce indukčné algoritmy -- vytvorené modely by nekonvergovali k lokálnemu optimu, ako to je pri indukčných algoritmoch. 

Genetický algoritmus vytvára jedincov z rozhodovacích koreňov (viď. \ref{kap1:2.3:2.3.2:stumpDT}). Mutácie a kríženia sú v algoritme rovnaké ako v našej implementácii, pričom na porovnávanie jedincov používa tzv. lexikografické multi-kriteriálne funkcie, ktoré fungujú podobne ako prioritná fitness z časti \ref{kap2:2.3:2.3.2:Priority}.

Vytvorené modely testovali na známych dátových množinách a porovnávali s indukčným algoritmom C4.5. Pri testovaní dostali stromy, ktoré pri porovnaní s C4.5 neboli presnosťou signifikantne lepšie, ale tieto stromy boli v niektorých prípadoch jednoduchšie.

\section{A new genetic programming algorithm for building decision tree}\label{kap6:6.2:Hybrid}
Cieľom autorov \cite{kap6-group} tejto práce bolo vytvárať rozhodovacie stromy genetickými algoritmami, pričom sa snažili zabrániť tomu, aby boli pri reprodukcii použité také stromy, ktorých rozdiel veľkostí je príliš veľký. 

Algoritmus uvažuje populácie stromov, ktoré rozdeľuje do skupín, pričom v každej skupine sú iba stromy rovnakých veľkostí. Kríženia a mutácia sú vykonané iba na jedincoch v rámci jednej skupiny. V práci uvádzajú, že takýto prístup dokáže vhodnejšie prehľadávať priestor riešení a vývoj kvality nájdených riešení nie je až tak skokový, ako v prípade obyčajných genetických algoritmov.

Vytvorené modely znova testovali na známych dátových množinách, porovnávali s indukčným algoritmom C4.5 a obyčajným genetickým algoritmom na vytváranie stromov. Autori uvádzajú presnosti modelov, ktoré sú vo veľa prípadoch signifikantne lepšie. Rýchlosť implementácie je v porovnaní s normálnym genetickým algoritmom menšia a stabilnejšia (pomalšie stúpa so zvyšujúcou sa veľkosťou dát).


% Ukázka použití některých konstrukcí LateXu (odkomentujte, chcete-li)
% \include{ukazka}

\chapter{Záver}\label{kap:fin}
\renewcommand{\figurename}{Obrázok}
Všetky stanovené ciele práce sa podarilo splniť. Genetický algoritmus na vytváranie rozhodovacích stromov, ktorý sme navrhli a naprogramovali v tejto práci dokáže skonštruovať rozhodovacie stromy. Kvalitou sú tieto stromy porovnateľné so stromami vytvorenými algoritmom C4.5 a zároveň sú od nich výrazne menšie.

Implementovaný algoritmus obsahuje značné množstvo nastaviteľných parametrov, ktoré je možné meniť pri každom spustení programu. Na druhú stranu sme sa v práci hlbšie nezaoberali hľadaním optimálneho nastavenia parametrov pre konkrétnu úlohu, pretože veľkosť priestoru parametrov je príliš veľký. Preskúmanie vhodných parametrov by vyžadovalo rozsiahle experimenty skúmajúce dopad rôznych nastavení na výsledné stromy.

V práci sme porovnali nový genetický algoritmus tvorby rozhodovacích stromov s indukčným algoritmom C4.5, pričom dosiahnuté výsledky môžeme považovať za úspech.
Pretože sme od algoritmu vyžadovali, aby vytváral menšie stromy, tak sme ich získali bez výrazného zhoršenia generalizačných schopností vytvorených stromov a pravdaže za cenu dlhšieho čakania na výsledky.

Aplikácia \verb|GenDTLib| je spustiteľná a použiteľná z nástroja Weka. Implementácia je spoľahlivá a rýchlosť tvorby modelu je postačujúca, keď berieme do úvahy, že používame genetické algoritmy. Na implementovanie sme použili jazyk Java, ktorá poskytuje nezávislosť aplikácie od operačného systému. S aplikáciou sme navrhli aj aplikačné rozhranie, pri vytváraní ktorého sme sa snažili o to, aby bolo zrozumiteľné a v prípade ďalšieho rozšírenia sa s ním pracovalo čo najľahšie. Zároveň sme museli zohľadniť to, že aplikácia má byť kompatibilná s nástrojom Weka. Parametre genetického algoritmu je možné nastaviť z konfiguračného súboru a existujúce komponenty je taktiež možné rozšíriť o nové pomocou zásuvných modulov.

Túto aplikáciu je možné spustiť na bežne dostupných počítačoch, na ktorých je spustiteľný aj nástroj Weka. 
Voľba vytvoriť aplikáciu ako zásuvný modul do nástroja Weka uľahčila prácu pri testovaní a umožňuje použiť klasifikátor z grafického rozhrania. Skonštruované stromy môžeme navyše vďaka tomu zobraziť zo vstavaného vizualizačného nástroja (obr. \ref{fig:wekatree}).

\begin{figure}[h]
\centering
\centerline{\mbox{\includegraphics[width=300pt]{../img/zaver/visualize.pdf}}}
\caption{Obrazovka z nástroja Weka, na ktorej je vizualizačný nástroj na zobrazovanie rozhodovacích stromov. Štruktúra stromov je podľa Definície \ref{kap1:2.3:2.3.2:DT}}\label{fig:wekatree}
\end{figure}

\section{Možné rozšírenia práce}
Aplikácia \verb|GenDTLib| je v súčasnej dobe plnohodnotný klasifikátor v nástroji Weka, ktorý vytvára rozhodovacie stromy pomocou genetických algoritmov. Možnosť nastaviť parametre genetického algoritmu v konfiguračnom súbore umožňuje užívateľovi meniť a experimentovať s existujúcimi komponentami a ich nastaveniami. Aplikáciu je ale možné rozšíriť aj o nové komponenty, o ktorých sme v práci neuvažovali. Pri ďalšom vývoji alebo v budúcich verziách aplikácie by bolo vhodné pridať niektoré funkcie:

\begin{itemize}
\item \textbf{Nový kontajner pre populáciu}. V pokročilých genetických algoritmoch sa problém správnej voľby veľkosti populácie rieši dynamicky sa meniacou veľkosťou populácie (na začiatku väčšia a postupne sa zmenšuje). Aplikáciu je možné rozšíriť o nové druhy populácií, a preto by bolo v budúcnosti vhodné takýto druh dynamickej populácie implementovať.
\item \textbf{Lepšia paralelizácia}. Aplikácia v aktuálnej verzii umožňuje nastaviť počet vlákien, ktoré počítajú fitness funkcie a generujú počiatočnú populáciu, no do budúcna by bolo možno vhodné paralelizovať aj niektoré komplikovanejšie operátory. Ďalšou možnosťou je implementovať niektorú z techník na paralelizovanie genetických algoritmov (napr. ostrovný model \cite{online-islandmodel}, hybridný paralelný genetický algoritmus).
\item \textbf{Vylepšiť zastavovacie kritéria}. Najpoužívanejším kritériom zastavenia je pri genetických algoritmoch počet generácií. V aktuálnej verzii je algoritmus zastavený po pevne danom počte generácií. Kritériá zastavenia môžu byť ale vhodnejšie. Pri ďalšom vývoji aplikácie by bolo vhodné vyskúšať a otestovať kvalitu riešení a rýchlosť genetických algoritmov s upravenými kritériami zastavenia. Takáto zmena by takisto mohla urýchliť celkový beh genetických algoritmov.
\item \textbf{Pareto fitness}. Pri niektorých úlohách riešených genetickými algoritmami sa používa pareto fitness. V aktuálnej verzii aplikácie je tento druh fitness nepodporovaný, a preto by bolo vhodné pareto fitness v budúcich verziách pridať a otestovať.
\end{itemize}

%%% Seznam použité literatury
%%% Seznam použité literatury je zpracován podle platných standardů. Povinnou citační
%%% normou pro diplomovou práci je ISO 690. Jména časopisů lze uvádět zkráceně, ale jen
%%% v kodifikované podobě. Všechny použité zdroje a prameny musí být řádně citoványy

\def\bibname{Zoznam použitej literatúry}
\addcontentsline{toc}{chapter}{\bibname}
\printbibliography
% http://en.wikipedia.org/wiki/Data_mining %



%%% Použité zkratky v diplomové práci, existují-li, včetně jejich vysvětlení.
\chapwithtoc{Zoznam použitých skratiek}
\textbf{KDD} - Knowledge Discovery in Databases \\
\textbf{DM} - Data mining / Dobývanie znalostí \\
\textbf{CRISP-DM} - CRoss-Industry Standard Process for Data Mining \\
\textbf{SVM} - Support vector machine \\
\textbf{AID} - Automatic Interaction Detection \\
\textbf{THAID} - THeta Automatic Interaction Detection \\
\textbf{CHAID} - CHi-squared Automatic Interaction Detection \\
\textbf{ID3} - Iterative Dichotomiser 3 \\
\textbf{CART} - Classification And Regression Tree \\
\textbf{TDIDT} - Top down induction of decision trees \\
\textbf{SCLGSI} - Symptom Checklist-90-R Global Severity Index \\
\textbf{GA} - Genetické algoritmy \\
\textbf{GP} - Genetické programovanie \\

\setcounter{chapter}{0} 
\renewcommand{\thechapter}{\Alph{chapter}}%
\chapter{Popis priloženého média}\label{kapI}
\chapter{Užívateľská príručka}\label{kapII}
\renewcommand{\figurename}{Obrázok}
\section{Grafické rozhranie aplikácie}
Nástroj Weka má implementované grafické rozhranie.
Tým, že sme vytvorili zásuvný modul do nástroja Weka je možné spustiť klasifikátor z grafického rozhrania. 

Uvítacia obrazovka, vykreslená na Obrázku \ref{fig:gui1}, umožňuje vytvárať modely kliknutímm na tlačítko \emph{Explorer} a testovať klasifikátory kliknutím na tlačítko \emph{Experimenter}. 

\begin{figure}[h]
\centering
\centerline{\mbox{\includegraphics[width=200pt]{../img/kapII/weka1.pdf}}}
\caption{Uvítacia obrazovka nástroja Weka.}\label{fig:gui1}
\end{figure}

\subsection*{Vytváranie modelov}
Po stlačení tlačítka \emph{Explorer} sa otvorí obrazovka z Obrázku \ref{fig:gui2}. V nej môžeme načítať (tlačítko \emph{Open file}) alebo vygenerovať (tlačítko Generate) dátovú množinu a tú ďalej predspracovať. Obrazovka obsahuje informácie o dátovej množine, ako je napríklad rozdelenie hodnôt pre atribúty. Predspracovanie dát môžeme vykonať pomocou tlačítka \emph{Edit}. Taktiež je možné vybrať iba určité atribúty, ktoré chceme použiť na klasifikáciu.

\begin{figure}[h]
\centering
\centerline{\mbox{\includegraphics[width=300pt]{../img/kapII/weka2.pdf}}}
\caption{Obrazovka, v ktorej načítame dáta a ďalej ich spracovávame.}\label{fig:gui2}
\end{figure}

Pre vytvorenie modelu pokračujeme zvolením záložky \emph{Classify}. Obrazovka na vytvorenie modelu vyzerá ako na Obrázku \ref{fig:gui3}. Tu môžeme zvoliť 
\begin{itemize}
\item typ klasifikátoru, ktorý chceme použiť. Výber klasifikátoru prevedieme kliknutím na tlačítko \emph{Choose}, ktoré otvorí sťahovacie menu rovnaké tomu na Obrázku \ref{fig:gui4},
\item akým spôsobom vytvárame a testujeme kvalitu vytvoreného modelu a akú dátovú množinu pritom použijeme. Medzi možnosti patrí:
\begin{itemize}
\item použiť trénovaciu množinu na vytvorenie modelu aj jeho otestovanie alebo
\item použiť trénovaciu množinu na vytvorenie modelu a dodatočne zadanú testovaciu množinu na jeho otestovanie, alebo
\item použiť $n$-násobnú krížovú validáciu, alebo
\item rozdeliť dátovú množinu na trénovaciu a testovaciu podmnožinu.
\end{itemize}
\item ktorý atribút bude výstupný.
\end{itemize}  

\begin{figure}[h!]
\centering
\centerline{\mbox{\includegraphics[width=300pt]{../img/kapII/weka3.pdf}}}
\caption{Obrazovka pre vytvorenie modelu, v ktorej volíme klasifikátor, ktorú množinu použijeme na otestovanie a ktorý atribút je výstupný.}\label{fig:gui3}
\end{figure}

\begin{figure}[h!]
\centering
\centerline{\mbox{\includegraphics[width=300pt]{../img/kapII/weka4.pdf}}}
\caption{Obrazovka s rozbaľovacím menu, v ktorej vyberáme klasifikátor.}\label{fig:gui4}
\end{figure}

Po zvolení algoritmu na vytvorenie modelu môžeme nastaviť jeho parametre. Po kliknutí na zvolený klasifikátor, vedľa tlačítka \emph{Choose}, sa objaví nová obrazovká rovnaká tej na Obrázku \ref{fig:gui5}. Nastavenie parametrov je bližšie spomenutý v oddieli \ref{kapII:config}.

Pre vytvorenie modelu stlačíme tlačítko \emph{Start}.

\begin{figure}[h!]
\centering
\centerline{\mbox{\includegraphics[width=300pt]{../img/kapII/weka5.pdf}}}
\caption{Príklad obrazovky, v ktorej nastavujeme parametre klasifikátoru -- a teda aj genetického algoritmu}\label{fig:gui5}
\end{figure}

\pagebreak

\subsection*{Testovanie modelov}
Po stlačení tlačítka \emph{Experimenter} dostaneme obrazovku z Obrázku \ref{fig:guie1}. Táto obrazovka je určená na testovanie modelov a ich porovnávanie. Na obrazovke môžeme zvoliť jednoduchý alebo pokročilý mód\footnote{Pokročilý mód poskytuje množstvo ďalších nastavení, ako napríklad rozloženie práce medzi viaceré počítače}. 
Pre vykonanie experimentu môžeme voliť z dvoch možností:
\begin{enumerate}
\item vytvoriť nový experiment (tlačítko \emph{New}) alebo
\item načítať existujúci experiment (tlačítko \emph{Open})
\end{enumerate} 

Podobne ako pri vytváraní modelov musíme
\begin{enumerate}
\item zvoliť algoritmy, ktoré budú použité na vytvorenie modelov (tlačítko \emph{Add new} v časti \emph{Algorithms}) a
\item načítať dátové množiny (tlačítko \emph{Add new} v časti \emph{Datasets}), na ktorých chceme vytvorené modely otestovať 
\end{enumerate}

Ďalej môžeme nastaviť typ experimentu, koľko krát bude experiment zopakovaný a akým spôsobom bude vykonaný (najprv množiny alebo najprv algoritmy). Medzi povolené typy experimentu patrí
\begin{enumerate}
\item $n$-násobná krížová validácia,
\item náhodné rozdelenie na trénovaciu a testovaciu množinu,
\item rozdelenie na trénovaciu a testovaciu množiny s rovnakým poradím
\end{enumerate}

\begin{figure}[h!]
\centering
\centerline{\mbox{\includegraphics[width=300pt]{../img/kapII/wekae1.pdf}}}
\caption{Obrazovka, ktorá slúži na testovanie algoritmov medzi sebou.}\label{fig:guie1}
\end{figure}

Príklad správne nastaveného experimentu je zobrazený na Obrázku \ref{fig:guie2}. Experiment vykoná 10-násobnú krížovú validáciu, pričom každá z nich je zopakovaná 10-krát. V experimente testujeme dva algoritmy (\verb|J48| a \verb|WekaEvolutionTreeClassifier|) na jednej dátovej množine (\verb|breast-cancer.arff|).

\begin{figure}[h!]
\centering
\centerline{\mbox{\includegraphics[width=300pt]{../img/kapII/wekae2.pdf}}}
\caption{Príklad obrazovky s nastavenými algoritmami a dátovou množinou na otestovanie.}\label{fig:guie2}
\end{figure}

Po nastavení experimentu stlačíme záložku \emph{Run}, ktorá zobrazí obrazovku, na ktorej spustíme experiment tlačítkom \emph{Start}. Po dokončení experimentu bez nejakých chýb dostaneme obrazovka, ktorá vyzerá ako na Obrázku \ref{fig:guie3}.

\begin{figure}[h!]
\centering
\centerline{\mbox{\includegraphics[width=300pt]{../img/kapII/wekae3.pdf}}}
\caption{Obrazovka, ktorá slúži na spustenia alebo zastavenie experimentu.}\label{fig:guie3}
\end{figure}

Nakoniec stlačíme záložku \emph{Analyse}, ktorá otvorí obrazovku, na ktorej môžeme zanalyzovať výsledky experimentu. Najprv je ale nutné načítať výsledky stlačením tlačítka \emph{Experiment}. Následne môžeme nastaviť, ktorú vlastnosť by sme chceli otestovať (napr. presnosť, veľkosť stromov), typ testu, jeho hladinu významnosti a iné. Výsledok zobrazíme pomocou tlačítka \emph{Perform test}. Typická obrazovka s vykonaným testom vyzerá ako na Obrázku \ref{fig:guie4}.

\begin{figure}[h!]
\centering
\centerline{\mbox{\includegraphics[width=300pt]{../img/kapII/wekae4.pdf}}}
\caption{Príklad obrazovky s vykonaným testom, ktorý porovnáva presnosť modelov, pričom na testovanie používame opravený párový t-test s hladinou významnosti 0.05.}\label{fig:guie4}
\end{figure}

\pagebreak

\section{Konzolové rozhranie aplikácie}
\begin{observation}
Povinné príkazy budeme v nasledujúcich častiach uvádzať bez zmeny. Nepovinné súčasti príkazu budeme uvádzať v hranatých zátvorkách.
\end{observation}

Nástroj Weka je možné spustiť s našou aplikáciou \verb|GenDTLib| z príkazového riadku pomocou príkazu

\begin{figure}[h!]
\centering
\begin{tabular}{|l|}
\hline
\texttt{java [parametre JVM] -cp "weka.jar;GenLib.jar"}  \\
\texttt{klasifikátor [parametre klasifikátoru]} \\
\texttt{[-t *.arff] [-T *.arff] [-h]} \\
\hline
\end{tabular}
\caption{}
\end{figure}

Nastavenie parametrov JVM nie je povinné a vo väčšine prípadov stačí to štandardné. V prípade, keď program skončí s výnimkou OutOfMemoryError, je potrebné zväčšiť hodnoty minimálnej a maximálnej hodnoty pamäte (parametre \verb|-Xms| a \verb|-Xmx|).

Povolené hodnoty argumentu \emph{klasifikátor} závisia od množiny klasifikátorov, ktoré sú registrované v nástroji Weka (súbor \verb|GenericPropertiesCreator.props|). Hodnoty na nastavenie majú štruktúru balíčkov z jazyka Java. Na spustenie nášho genetického algoritmu na tvorbu stromov volíme hodnotu klasifikátora ako \verb|genlib.classifier.weka.WekaEvolutionTreeClassifier|.

Parametre klasifikátoru je možné zobraziť pomocou argumentu \verb|-h|. Nastavenie parametrov uvádzame bližšie v oddieli \ref{kapII:config}. Tieto parametre sú ale nepovinné, pretože štandardne sa berú hodnoty zo súboru \verb|config.properties|.

Argumentom \verb|-t| definujeme akú trénovaciu množinu použijeme. Parametre tohoto argumentu sú relatívne cesty k súborom (s príponou \verb|arff|) s dátovými množinami.

Argumentom \verb|-T| definujeme akú testovaciu množinu použijeme. Parametre tohoto argumentu sú relatívne cesty k súborom (s príponou \verb|arff|) s dátovými množinami.

\section{Nastavenie argumentov genetického algoritmu}\label{kapII:config}
\begin{observation}
Pri nastavení argumentov genetického algoritmu budú hodnoty na vyplnenie v zložených zátvorkách.
\end{observation}

V implementačnej časti sme zaviedli požiadavky, ktoré si na aplikáciu kladieme. Jednou z nich bola možnosť nastaviť argumenty (parametre) genetického algoritmu. V tejto kapitole popíšeme argumenty algoritmu a aké hodnoty sú pre ne povolené. Názvy argumentov budú zodpovedať grafickému rozhraniu, no pre každý parameter uvedieme aj jeho ekvivalent v konzolovom rozhraní.

Nastavenie argumentov genetického algoritmu môžeme rozdeliť do  4 skupín:
\begin{enumerate}
\item Nastavenie číselného argumentu algoritmu (napr. počet generácii, veľkosť populácie)
\begin{center}
\texttt{\{argument\_algoritmu\}=\{číslo\}}
\end{center} 
\item Nastavenie objektového argumentu algoritmu (dátová množina)
\begin{center}
\texttt{\{argumentmeter\_algoritmu\}=\{parametre\}}
\end{center} 
\item Nastavenie argumentu, ktorý pracuje s komponentami
\begin{center}
\texttt{\{argument\_algoritmu\}=\{komponenta\} \{parametre\}}
\end{center}
\item Nastavenie argumentu, ktorý podporuje viac než jednu komponentu
\begin{center}
\texttt{\{argument\_algoritmu\}=\{komponenta1\} \{parametre1\}};
\texttt{\{komponenta2\} \{parametre2\};...}
\end{center}
\end{enumerate}

Hodnota na vyplnenie \texttt{\{parametre\}} môže byť nastavená ako:
\begin{itemize}
\item n-tica dvojíc parameter,hodnota
\begin{center}
\texttt{parametre := \{parameter1\},\{hodnota\_parametru1\},
\{parameter2\},\{hodnota\_parametru2\},...} 
\end{center}
\item n-tica hodnôt
\begin{center}
\texttt{parametre := \{hodnota1\},\{hodnota2\},...} 
\end{center}
\item bez parametru
\begin{center}
\texttt{parametre := x} 
\end{center}
\end{itemize}

\subsection{Argumenty}
Argumenty môžeme deliť podľa skupín spomínaných vyššie. Do skupiny 1 patria tieto argumenty:
\begin{enumerate}
\item \texttt{number-of-generations} -- počet generácií,
\item \texttt{pop-size} -- veľkosť populácie,
\item \texttt{elitism} -- predstavuje hodnotu elitizmu,
\item \texttt{fit-threads} -- počet vlákien použitých na počítanie fitness funkcií,
\item \texttt{gen-threads} -- počet vlákien použitých na vytvorenie počiatočnej populácie,
\item \texttt{classify} -- počet najlepších rozhodovacích stromov použitých na klasifikáciu. V tomto prípade bude genetický algoritmus fungovať podobne ako technika \emph{bagging}.
\end{enumerate}
Komplikovanejšie argumenty popíšeme v nasledujúcich úsekoch.

\subsubsection*{Dátová množina}
Argumentom \verb|data| (skupina 2.) definujeme ako bude rozdelená dátová množina na trénovaciu a validačnú podmnožinu. 
Dovolené parametre sú:
\begin{itemize}
\item \verb|TRAINRATIO| Podiel trénovacej množiny v dátach. Jeho hodnota je číslo z intervalu [0,1],
\item \verb|RESAMPLE| Podmnožiny sú vytvorené výberom s opakovaním. Jeho hodnota je typu boolean.
\end{itemize}

\subsubsection*{Fitness funkcie}
Argumentom \verb|fit-functions| (skupina 4.), definujeme fitness funkcie, ktoré budú použité na vyhodnotenie jedincov.
Dovolené komponenty sú 
\begin{itemize}
\item \verb|tAcc| -- presnosť rozhodovacieho stromu,
\item \verb|tSize|  -- počet vrcholov rozhodovacieho stromu,
\item \verb|tHeight| -- výška rozhodovacieho stromu,
\item \verb|tFMsr| -- f-miera rozhodovacieho stromu,
\item \verb|tTP| -- počet pravdivo pozitívnych inštancií,
\item \verb|tTN| -- počet pravdivo negatívnych inštancií,
\item \verb|tFP| -- počet nepravdivo pozitívnych inštancií,
\item \verb|tFN| -- počet nepravdivo negatívnych inštancií,
\item \verb|tPrecision| -- precíznosť rozhodovacieho stromu,
\item \verb|tPreval| -- hodnota `'prevalence'' z matice chybovosti rozhodovacieho stromu,
\item \verb|tRecall| -- citlivosť rozhodovacieho stromu,
\item \verb|tSpecificity| -- špecificita rozhodovacieho stromu.
\end{itemize}

Dovolené parametre sú:
\begin{itemize}
\item \verb|INDEX| -- index fitness funkcie v jedincovi,
\item \verb|DATA| -- dátová množina použitá na výpočet fitness funkcie. Jeho hodnota je buď -1 (celá množina), 0 (trénovacia množina), 1 (validačná množina).
\end{itemize}

\subsubsection*{Vyhodnotenie fitness funkcií}
Argumentom \verb|fit-comparator| (skupina 4.) definujeme to, ako budeme určovať kvalitu jedincov.
Dovolené komponenty sú 
\begin{itemize}
\item \verb|SINGLE| Kvalitu jedincov určujeme pomocou jednej vybranej fitness funkcie,
\item \verb|PRIORITY| Kvalitu jedincov určujeme fitness funkciami branými prioritne \ref{kap2:2.3:2.3.2:Priority},
\item \verb|WEIGHT| Kvalitu jedincov určujeme pomocou váženej hodnoty fitness funkcií \ref{kap2:2.3:2.3.1:Weighted}.
\end{itemize}

Parameter komponenty pozostáva z $n$ hodnôt, pričom $n = po\v cet funkcií$.

\subsubsection*{Kríženia a mutácie}
Argumentom \verb|xover-operators| (skupina 4.) definujeme operátory kríženia.
Dovolené komponenty (kríženia) sú: 
\begin{itemize}
\item \verb|dtX| -- žiadne kríženia (skopírovanie jedincov),
\item \verb|subTreeX|  -- kríženie podstromov.
\end{itemize}

Argumentom \verb|mut-operators| (skupina 4.) definujeme operátory mutácie.
Povolené komponenty (mutácie) sú:
\begin{itemize}
\item \verb|dtM| -- žiadna mutácia (skopírovanie jedincov),
\item \verb|ntlNomM|  -- mutácia podstromu na list (výstupný atribút je kategoriálny),
\item \verb|ntlNumM|  -- mutácia podstromu na list, (výstupný atribút je numerický),
\item \verb|wekaDSM|  -- mutácia uzlu na rozhodovací koreň,
\item \verb|valueMut| -- \textbf{POZOR!} experimentálna mutácia, ktorá náhodne zmení hodnotu klasifikácie v listu.
\end{itemize}

Dovolené parametre sú:
\begin{itemize}
\item \verb|PROB| Pravdepodobnosť operátora.
\item \verb|DATA| Dátová množina použitá pri práci s operátorom. Jeho hodnota je buď -1 (celá množina), 0 (trénovacia množina), 1 (validačná množina).
\end{itemize}

\subsection*{Selekcie}
Argumentom \verb|selectors| (skupina 4.) definujeme selekcie \ref{kap2:2.4:Selection} a argumentom \verb|env-selectors| (skupina 4.) definujeme selekcie prostredia \ref{kap2:2.4:2.4.4:EnvironmentalSelection}.
Dovolené komponenty sú 
\begin{itemize}
\item \verb|random| \textbf{POZOR!} experimentálny výber jedincov, ktorá vyberá jedincov náhodne.
\item \verb|Rw| Ruletová selekcia,
\item \verb|Tmt|  Turnajová selekcia.
\end{itemize}

Selekcie sú bez parametrov.

\subsubsection*{Inicializácia populácie}
Argumentom \verb|pop-init| (skupina 3.), definujeme inicializátory počiatočnej populácie.
Povolené komponenty sú 
\begin{itemize}
\item \verb|wCompTree| Rozhodovacie stromy sú zaradené do počiatočnej populácie bez zmeny,
\item \verb|wRanStump| Rozhodovacie korene sú medzi sebou spájané až dokým nedosiahnu určitú výšku.
\end{itemize}

Dovolené parametre sú:
\begin{itemize}
\item \verb|RESAMPLE| Podmnožiny, ktoré sú použité na vytvorenie jedincov, sú vytvorené výberom s opakovaním. Jeho hodnota je typu boolean,
\item \verb|AUTOHEIGHT| Na reprezentovanie rozhodovacích stromov sa použije trieda \verb|MultiWayDepthNode|. Jeho hodnota je typu boolean,
\item \verb|DIVIDEPARAM| Podmnožiny, ktoré sú použité na vytvorenie jedincov, sú získané rozdelením dátovej množiny na $n$ kúskov. Hodnota parametru je číslo $n$.
\item \verb|DATA| Dátová množina použitá pri práci s operátorom. Jeho hodnota je buď -1 (celá množina), 0 (trénovacia množina), 1 (validačná množina).
\end{itemize}

\subsubsection*{Generátor jedincov}
Argument s názvom \verb|ind-generator| (skupina 3.), ktorým definujeme fitness funkcie, ktoré budú použité na vyhodnotenie jedincov.
Dovolené komponenty sú 
\begin{itemize}
\item \verb|wJ48Gen| Rozhodovacie stromy sú generované pomocou algoritmu J48 z nástroja \verb|Weka|.
\end{itemize}

Parametre sú rovnaké ako pre algoritmus J48.
\chapter{Rozšíriteľnosť aplikácie}
V tejto kapitole popíšeme možnosti rozšírenia aplikácie o nové komponenty (operátory, selekcie, fitness funkcie, populácie) a využitie aplikácie aj bez nástroja Weka.

\section{Rozšíriteľnosť}\label{kap4:4.4:Plugin}
Na rozšírenie aplikácie \verb|GenDTLib| môžeme použiť existujúce rozhrania z balíčku \verb|genlib.plugins|. Rozhrania môžu byť určené buď na definovanie nových komponent genetického algoritmu, alebo na definovanie nových klasifikátorov a deliacich kritérií. Toto je možné kvôli zvolenej abstrakcii genetického algoritmu.
\subsection{Nové komponenty}
Existujúce komponenty môžeme rozšíriť pridaním nových knižníc do predom určených zložiek. Knižnice musia dediť od niektorého rozhrania z \verb|genlib.plugins|. Typ knižnice závisí od tohoto rozhrania. Definované typy knižníc sú uvádzané v dodatku XX.

\subsection{Použitie bez Weky}

\openright
\end{document}
